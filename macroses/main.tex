\documentclass[nohyper,nobib,justified]{tufte-book}
\usepackage{nameref}

\usepackage{url}

\usepackage{xargs}

\usepackage[LGR,T1]{fontenc}

\usepackage{amsmath}
\usepackage{amsfonts}
\usepackage{amsthm}
\usepackage{amssymb}

\usepackage{mathtools}

\usepackage{algorithm}
\usepackage{algpseudocode}

\usepackage{tikz}
\usetikzlibrary{shapes,backgrounds}

\usepackage{subfig}

\usepackage{ifthen}

\usepackage{multicol}
\usepackage[inline]{enumitem}

\usepackage{anyfontsize}

\usepackage{fitch}

\usepackage{qrcode}
\usepackage{wrapfig}

\usepackage{epigraph}

\usepackage{multirow}
\usepackage{booktabs}
\renewcommand{\arraystretch}{1.2}

\usepackage{xspace}

\usepackage{units}

\usepackage{csquotes}

\usepackage[refpage]{nomencl}
\makenomenclature


% Generates the index
\usepackage{makeidx}
\makeindex

%%%% Kevin Godny's code for title page and contents from https://groups.google.com/forum/#!topic/tufte-latex/ujdzrktC1BQ
\makeatletter
\renewcommand{\maketitlepage}{%
\begingroup%
\setlength{\parindent}{0pt}

{\fontsize{24}{24}\selectfont\textit{\@author}\par}

\vspace{1.75in}{\begin{minipage}{1.5\linewidth}\fontsize{36}{54}\selectfont\@title\par\end{minipage}}

\vspace{0.5in}{\fontsize{14}{14}\selectfont\textsf{\smallcaps{\@date}}\par}

\vfill{\fontsize{14}{14}\selectfont\textit{\@publisher}\par}

\thispagestyle{empty}
\endgroup
}
\makeatother

\usepackage[pageanchor,hidelinks,linktoc=all]{hyperref}

\titlecontents{part}%
    [0pt]% distance from left margin
    {\addvspace{0.25\baselineskip}}% above (global formatting of entry)
    {\uppercase{Part~\thecontentslabel}\uppercase}% before w/ label (label = ``Part I'')
    {\uppercase{Part~\thecontentslabel}\uppercase}% before w/o label
    {}% filler and page (leaders and page num)
    [\vspace*{0.5\baselineskip}]% after

\titlecontents{chapter}%
    [4em]% distance from left margin
    {}% above (global formatting of entry)
    {\contentslabel{2em}\textit}% before w/ label (label = ``Chapter 1'')
    {\hspace{0em}\textit}% before w/o label
    {\qquad\thecontentspage}% filler and page (leaders and page num)
    [\vspace*{0.5\baselineskip}]% after
%%%% End additional code by Kevin Godby

\setcounter{secnumdepth}{1}

\titleformat{\chapter}%
[block]% shape
{\relax\ifthenelse{\NOT\boolean{@tufte@symmetric}}{\begin{minipage}{\linewidth}}{}}% format applied to label+text
{\itshape\huge\thechapter.\hspace{10pt}}% label
{0pt}% horizontal separation between label and title body
{\huge\rmfamily\itshape}% before the title body
[\ifthenelse{\NOT\boolean{@tufte@symmetric}}{\end{minipage}}{}]% after the title body

\renewcommand{\nomname}{List of Symbols}

\usepackage{ifthen}
\renewcommand{\nomgroup}[1]{%
  \item[\bfseries%
  \ifthenelse{\equal{#1}{S}}{Set Notation}{%
  \ifthenelse{\equal{#1}{C}}{Counting}{%
  \ifthenelse{\equal{#1}{L}}{Logical Notation}{%
  \ifthenelse{\equal{#1}{R}}{Relations}{%
  \ifthenelse{\equal{#1}{F}}{Functions}{%
  \ifthenelse{\equal{#1}{G}}{Graphs}{%
  }}}}}}%
]}

\def\pagedeclaration#1{, \hyperlink{page.#1}{page\nobreakspace#1}}

\setlength\nomlabelwidth{2.5cm}

\renewcommand{\nompreamble}{%
  The letters $A$, $B$, $X$, $Y$, and $Z$ denote sets, the letters $x$, $y$,
  and $z$ denote the elements of $X$, $Y$, and $Z$ respectively, $P$ and $Q$
  denote propositions and predicates, the lower case latin letters $f$ and $g$
  denote functions from $X$ to $Y$ and from $Y$ to $Z$ respectively,
  the letters $a$, $b$, $n$, and $k$ denote integer numbers, and the greek
  letter $\alpha$ and $\beta$ denote real numbers.
}

\usepackage{environ}

\usepackage{youngtab}


\newtheorem{exercise}{Exercise}[chapter]
\newtheorem{principle}{Principle}[chapter]
\newtheorem{definition}{Definition}[chapter]
\newtheorem{theorem}{Theorem}[chapter]
\newtheorem{lemma}{Lemma}[chapter]
\newtheorem{remark}{Remark}[chapter]
\newtheorem{corollary}{Corollary}[chapter]

\newsavebox{\mybox}
\newenvironment{warning}
{
    \vskip 5pt
    \begin{lrbox}{\mybox}\begin{minipage}{0.9\textwidth}
        \textbf{Warning:}
}
{
    \end{minipage}\end{lrbox}\fbox{\usebox{\mybox}}
    \vskip 5pt
}

\newenvironment{template}
{
    \vskip 5pt
    \begin{lrbox}{\mybox}\begin{minipage}{0.9\textwidth}
}
{
    \end{minipage}\end{lrbox}\fbox{\usebox{\mybox}}
    \vskip 5pt
}

\newcounter{solutionnumber}

\newenvironment{solution}
{
    \begingroup\edef\x{\endgroup\noexpand\solutione{\theexercise}}\x}
{
    \endsolutione
}

\definecollection{solutions}

\newenvironment{solutione}[1]{
    \collect{solutions}{
        \protect \item[\textup{\textbf{#1}}]
    }{}
}{
    \endcollect
    \stepcounter{solutionnumber}
}

\newcommand{\hint}[1]{
    \textit{Hint: #1}
}

\newcommand{\labitem}[2]{%
\def\@itemlabel{\textbf{#1}}
\item
\def\@currentlabel{#1}\label{#2}}
\makeatother

\newenvironment{chapterendexercises}
{
    \renewcommand{\exercise}[1][ ]
    {
        \stepcounter{exercise}
        \item[\textup{\textbf{\theexercise}}]
            \ifthenelse{\equal{##1}{ }}{}{\textit{(##1)}}
    }
    \section*{End of The Chapter Exercises}
    \begin{description}
}
{
    \end{description}
    \ifnum\thesolutionnumber>0
        \section*{Solutions to The Exercises}
        \begin{description}
            \includecollection{solutions}
        \end{description}
    \fi
    \setcounter{solutionnumber}{0}
    \begin{collect}{solutions}{}{}
        % We need to clean the file

    \end{collect}
}

\makeatother

\newcommand{\floor}[1]{\left\lfloor {#1} \right\rfloor}
\newcommand{\ceil}[1]{\left\lceil {#1} \right\rceil}
\newcommand{\textgreek}[1]{\begingroup\fontencoding{LGR}\selectfont#1\endgroup}
\newcommand{\divides}{\mid}
\newcommand{\notdivides}{\nmid}

% set theory
\renewcommand{\Im}{\mathrm{Im}}
\newcommand{\set}[2][ ]{\left\{#2 \ifthenelse{\equal{#1}{ }}{ }{~:~#1}\right\}}
\newcommand{\N}{\mathbb{N}}
\newcommand{\Z}{\mathbb{Z}}
\newcommand{\Q}{\mathbb{Q}}
\newcommand{\R}{\mathbb{R}}
\newcommand{\C}{\mathbb{C}}
\newcommand{\range}[1]{[{#1}]}

% math logic
\newcommand{\bigland}{\bigwedge}
\newcommand{\biglor}{\bigvee}
\newcommand{\ltrue}{\text{T}}
\newcommand{\lfalse}{\text{F}}

\newcommand{\fpm}{\mathcal{G}}

\newcommand{\marginurl}[2]{
  \begin{marginfigure}
    {\scriptsize {#1}}
    \vskip 0.25cm
    \noindent
    \qrset{link, height=3cm}
    \qrcode{https://#2}
    \vskip 0.25cm
    \noindent
    \url{https://#2}
  \end{marginfigure}
}

