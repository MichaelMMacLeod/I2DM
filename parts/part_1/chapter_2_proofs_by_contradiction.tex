\chapter{Proofs by Contradiction}
\begin{marginfigure}
  {\scriptsize Proofs by Contradiction:\\\noindent
  Introduction to Mathematical Reasoning \#3}
  \vskip 0.25cm
  \noindent
  \qrset{link, height=3cm}
  \qrcode{https://youtu.be/bWP0VYx75DI}
  \vskip 0.25cm
  \noindent
  \url{https://youtu.be/bWP0VYx75DI}
\end{marginfigure}
\section{Proving Negative Statements}
The direct method is not very convenient when we need to prove a negation of
some statement.

For example, we may try to prove that $78 n + 102 m = 11$ does not have integer
solutions. It is not clear how to prove it directly since we can not consider
all possible $n$ and $m$. Hence, we need another approach. Let us assume that
such a solution $n, m$ exists. Note that $78 n + 102 m$ is even, but $11$ is
odd. In other words, an odd number is equal to an even number, it is impossible.
Thus, the assumption was false.

Let us consider a more useful example, let us prove that if $p^2$ is even, then
$p$ is also even ($p$ is an integer). Assume the opposite i.e. that $p^2$ is
even but $p$ is not. Let $p = 2b + 1$\footnote{Note that we use here the
statement that an integer $n$ is not even iff it is odd, which, formally
speaking, should be proven.}. Note that $p^2 = (2b + 1)^2 = 2(2b^2 + 2b) + 1$.
Hence, $p^2$ is odd which contradicts to the assumption that $p^2$ is even.

Using this idea we may prove much more complicated results e.g. one may show
that $\sqrt{2}$ is irrational. For the sake of contradiction, let us assume
that it is not true. In other words there are $p$ and $q$ such that
$\sqrt{2} = \frac{p}{q}$ and $\frac{p}{q}$ is an irreducible fraction.

Note that $\sqrt{2} q = p$, so $2q^2 = p^2$. Which implies that $p$ is even
and $4$ devises $p^2$. Therefore $4$ devises $2q^2$ and $q$ is also even. As
a result, we get a contradiction with the assumption that $\frac{p}{q}$ is an
irreducible fraction.

\begin{template}
  \textbf{Template for proving a statement by contradiction.} \\

  Assume, for the sake of contradiction, that \textit{the statement} is false.
  Then \textit{present some argument that leads to a contradiction}. Hence, the
  assumption is false and \textit{the statement} is true.
\end{template}

\begin{exercise}
  Show that $\sqrt{3}$ is irrational.
\end{exercise}

\section{Proving Implications by Contradiction}
This method works especially well when we need to prove an implication.
Since the implication $A \implies B$ is false only when $A$ is true but $B$ is
false. Hence, you need to derive a contradiction from the fact that $A$ is true
and $B$ is false.

We have already seen such examples in the previous section, we proved that
$p^2$ is even implies $p$ is even for any integer $p$. Let us consider another
example. Let $a$ and $b$ be reals such that $a > b$. We need to show that
$(ac < bc) \implies c < 0$. So we may assume that $ac < bc$ but $c \ge 0$. By
the multiplicativity of the inequalities we know that if $(a > b)$ and $c > 0$,
then $ac > bc$ which contradicts to $ac < bc$.

A special case of such a proof is when we need to prove the implication
$A \implies B$, assume that $B$ is false and derive that $A$ is false which
contradicts to  $A$ (such proofs are called proofs by contraposition); note
that the previous proof is a proof of this form.

\section{Proof of ``OR'' Statements}
Another important case is when we need to prove that at least one of two
statements is true. For example, let us prove that $ab = 0$ iff $a = 0$ or
$b = 0$. We start from the implication from the right to the left. Since if
$a = 0$, then $ab = 0$ and the same is true for $b = 0$ this implication is
obvious.

The second part of the proof is the proof by contradiction. Assume $ab = 0$,
$a \neq 0$, and $b \neq 0$. Note that $b = \frac{ab}{a} = 0$,
hence $b = 0$ which is a contradiction to the assumption.

\section*{End of The Chapter Exercises}
\begin{exercises}
  \exerciseitem Prove that if $n^2$ is odd, then $n$ is odd.
  \exerciseitem  In Euclidean (standard) geometry, prove: If two lines share a
    common perpendicular, then the lines are parallel.
  \exerciseitem Let us consider four-lines geometry, it is a theory with
    undefined terms: point, line, is on, and axioms:
    \begin{enumerate}
        \item there exist exactly four lines,
        \item any two distinct lines have exactly one point on both of them, and
        \item each point is on exactly two lines.
    \end{enumerate}

    Show that every line has exactly three points on it.
  \exerciseitem
    Let us consider group theory, it is a theory with undefined
    terms: group-element and times (if $a$ and $b$ are group elements,
    we denote $a$ times $b$ by $a \cdot b$), and axioms:
    \begin{enumerate}
      \item $(a \cdot b) \cdot c = a \cdot (b \cdot c)$
        for every group-elements $a$, $b$, and $c$;
      \item there is a unique group-element $e$ such that
        $e \cdot a = a = a \cdot e$ for every group-element $a$ (we say that such
        an element is the identity element);
      \item for every group-element $a$ there is a group-element $b$
        such that $a \cdot b = e$, where $e$ is the identity element;
      \item for every group-element $a$ there is a group-element $b$
        such that $b \cdot a = e$, where $e$ is the identity element.
    \end{enumerate}

    Let $e$ be the identity element. Show the following statements
    \begin{itemize}
      \item if $b_0 \cdot a = b_1 \cdot a = e$, then $b_0 = b_1$, for every
        group-elements $a$, $b_0$, and $b_1$.
      \item if $a \cdot b_0 = a \cdot b_1 = e$, then $b_0 = b_1$, for every
        group-elements $a$, $b_0$, and $b_1$.
      \item if $a \cdot b_0 = b_1 \cdot a = e$, then $b_0 = b_1$, for every
        group-elements $a$, $b_0$, and $b_1$.
    \end{itemize}
  \exerciseitem Let us consider three-points geometry, it is a theory with
    undefined terms: point, line, is on, and axioms:
    \begin{enumerate}
        \item There exist exactly three points.
        \item Two distinct points are on exactly one line.
        \item Not all the three points are collinear i.e. they do not lay on the
            same line.
        \item Two distinct lines are on at least one point i.e. there is at
          least one point such that it is on both lines.
    \end{enumerate}

    Show that there are exactly three lines.
  \exerciseitem Show that there are irrational numbers $a$ and $b$ such that
    $a^b$ is rational.
  \exerciseitem Show that there does not exist the largest integer.
\end{exercises}
