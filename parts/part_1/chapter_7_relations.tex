\chapter{Relations}
Nonetheless that function are used almost everywhere in mathematics, many
relations are not functional by their nature. For example, could never define a
function $r(a)$ that gives the solution of $x^2 = a$ because there are two
solutions for $a > 0$ and there are zero solutions for $a < 0$. A relation is a
more general mathematical object.

In order to define a relation we need to relax the definition of the graph of a
function (Section~\ref{section:graph}) by allowing more than one ``result'' and
by allowing zero ``results''. In other words we just say that any set $R
\subseteq X_1 \times \dots \times X_k$ is a $k$-ary \emph{relation} on $X_1$,
\dots, $X_k$. We also say that $x_1 \in X_1$, \dots, $x_k \in X_k$ are in the
relation $R$ iff $(x_1, \dots, x_k) \in R$. If $k = 2$ such a relation is
called a \emph{binary relation} and we write $x R y$ if $x$ and $y$ are in
the relation $R$. If $X_1 = \dots = X_k = X$, we say that $R$ is a $k$-ary
relation on $X$.

Note that $=$, $\le$, $\ge$, $<$, and $>$ define relations on $\R$ (or any
subset $S$ of $\R$). For example, if $S = \set{0, 1, 2}$, then $<$ defines
the relation $R = \set{(0, 1), (0, 2), (1, 2)}$.

Probably the most popular relation in mathematics is the following relation on
$\Z$. Let $a, b \in \Z$. If $n$ divides $a - b$ for some $n \in Z$, we say that
``$a$ equivalent to $b$ modulo $n$'' and denote it as $a \equiv b \pmod{n}$.
For example, $1$ and $4$ are equivalent modulo $3$ since $3$ divides
$1 - 4 = -3$.

\section{Equivalence Relations}
The definition of a relation is way to broad. Hence, quite often we consider
some types of relation. Probably the most interesting type of the relations is
equivalence relations.

\begin{definition}
  Let $R$ be a relation on a set $X$. We say that $R$ is an \emph{equivalence
  relation} if it satisfies the following conditions:
  \begin{description}
    \item[reflexivity:]  $x R x$ for any $x \in X$;
    \item[symmetry:]  $x R y$ iff $y R x$ for any $x, y \in X$;
    \item[transitivity:]  for any $x, y, z \in X$, if $x R y$ and $y R Z$, then
      $x R z$;
  \end{description}
\end{definition}
One may guess that the equivalence relation are mimicking $=$, so it is not a
surprise that $=$ is an equivalence relation.

The definition seems quite bizarre, however, all of you are already familiar
with an important example: you know that equivalent fractions represent the
same number. For example $\frac{2}{4}$ is the same as $\frac{1}{2}$.
Let us consider this example more thorough, let $S$ be a set of symbols of the
form $\frac{x}{y}$ (note that it is not a set of numbers) where $x, y \in Z$
and $y \neq 0$. We define a binary relation $R$ on $S$ such that
$\frac{x}{y}$ and $\frac{z}{w}$ are in the relation $R$ iff $xw = zy$.
It is easy to prove that this relation is an equivalence relation.
\begin{description}
  \item[reflexivity:] Let $\frac{a}{b} \in S$. Since $ab = ab$, we have that
    $\frac{a}{b} R \frac{a}{b}$.
  \item[symmetry:] Let $\frac{a}{b}, \frac{c}{d} \in S$. Suppose that
    $\frac{a}{b} R \frac{c}{d}$, by the definition of $R$, it implies that
    $ac = db$. As a result, $\frac{c}{d} R \frac{a}{b}$.
  \item[transitivity:] Let $\frac{a}{b}, \frac{c}{d}, \frac{e}{f} \in S$ with
    $\frac{a}{b} R \frac{c}{d}$ and $\frac{c}{d} R \frac{e}{f}$. Then
    $ad = cb$ and $cf = ed$. The first equality can be rewritten as $c = ad / b$.
    Hence, $adf / b = ed$ and $af = eb$ since $d \neq 0$. So $\frac{a}{b} R
    \frac{e}{f}$.
\end{description}

\subsection{Partitions}
Let $S$ be some set. We say that $\set{P_1, \dots, P_k}$ form a partition of
$S$ iff $P_1$, \dots, $P_k$ are pairwise disjoint and
$P_1 \cup \dots \cup P_k = S$; in other words, a partition is a way of dividing
a set into overlapping pieces.

\begin{exercise}
  Let $\set{P_1, \dots, P_k}$ be a partition of a set $S$ and $R$ be a binary
  relation of $S$ such that $a R b$ iff $a, b \in P_i$ for some $i \in [k]$.
  Show that $R$ is an equivalence relation.
\end{exercise}

This exercise shows that one may transform a partition of the set $S$ into an
equivalence relation on $S$. However, it is possible to do the opposite.
\begin{theorem}
\label{theorem:partition}
  Let $R$ be a binary equivalence relation on a set $S$. For any element $x \in
  S$, define $R_x = \set[x R y]{y \in S}$ (the set of all the elements of $S$
  related to $x$) we call such a set the equivalence class of $x$.
  Then $\set[x \in S]{R_x}$ is a partition of $S$.
\end{theorem}

\begin{exercise}
  Prove Theorem~\ref{theorem:partition}.
\end{exercise}

\subsection{Modular Arithmetic}
The relation ``$\equiv \pmod{n}$'' is actively used in the number theory. One
of the important properties of this relation is that it is an equivalence
relation.
\nomenclature[R]{$a \equiv b \pmod{n}$}{says that $n$ divides $a - b$}
\begin{theorem}
  The relation $\equiv \pmod{n}$ is an equivalence relation.
\end{theorem}
\begin{proof}
  To prove this statement we need to prove all three properties: reflexivity,
  symmetry, and transitivity.
  \begin{description}
    \item[reflexivity:] Note that for any integer $x$, $x - x = 0$ is divisible
      by any integer including $n$. Hence, $x \equiv x \pmod{n}$.
    \item[symmetry:] Let us assume that $x \equiv y \pmod{n}$; i.e. $x - y = kn$
      for some integer $k$. Note that $y - x = (-k) n$, so
      $y \equiv x \pmod{n}$.
    \item[transitivity:] finally, assume that $x \equiv y \pmod{n}$ and
      $y \equiv z \pmod{n}$; i.e. $x - y = k n$ and $y - z = \ell n$
      for some integers $k$ and $\ell$. It is easy to note that
      $x - z = (x - y) + (y - z) = (k + \ell) n$. As a result,
      $x \equiv z \pmod{n}$.
  \end{description}
  Thus, we proved that $\equiv \pmod{n}$ is an equivalence relation.
\end{proof}

Let $x \in \Z$; we denote by $r_{x, n}$ the equivalence class of $x$ with
respect to the relation $\equiv \pmod{n}$, we also denote by $\Z / n\Z$ the set
of all the equivalence classes with respect to the relation $\equiv \pmod{n}$.

Another important property of these relations is that they behave well with
respect to the arithmetic operations.
\begin{theorem}
  Let $x, y \in Z$ and $n \in \N$. Suppose that $a \in r_{x, n}$ and $b \in
  r_{y, n}$, then $(a + b) \in r_{x + y, n}$ and $ab \in r_{xy, n}$.
\end{theorem}

Using this theorem we may define arithmetic operations on the equivalence
classes with respect to the relation $\equiv \pmod{n}$.
Let $x, y \in \Z$ and $n \in \N$. Then
$r_{x, n} + r_{y, n} = \set[a \in r_{x, n}, b \in r_{y, n}]{a + b} =
r_{x + y, n}$ and
$r_{x, n} r_{y, n} = \set[a \in r_{x, n}, b \in r_{y, n}]{ab} = r_{xy, n}$.
Moreover, these operations have plenty of good properties.
\begin{exercise}
  Let $a, b, c \in \Z / n\Z$.
  Show that the following equalities are true:
  \begin{itemize}
    \item $a + (b + c) = (a + b) + c$,
    \item $a + r_{0, n} = a$ (thus we denote $r_{0, n}$ as $0$),
    \item $a r_{1, n} = a$ (thus we denote $r_{1, n}$ as $1$),
    \item there is a class $d \in \Z / n\Z$ such that $a + d = r_{0, n}$
      (thus we denote this $d$ as $-a$),
    \item $a + b = b + a$,
    \item $ab = ba$,
    \item $a(b + c) = ab + ac$,
  \end{itemize}
\end{exercise}

\section{Partial Orderings}
In the previous section we discussed a mathematical way to express
the property being similar. In this section we are going to give a way to
analyze relation similar to comparisons.

\begin{definition}
  A binary relation $R$ on $S$ is a partial ordering if it satisfies the
  following
 constraints.
  \begin{description}
    \item[reflexivity:]  $x R x$ for any $x \in S$;
    \item[antisymmetry:] if $x R y$ and $y R x$, then $x = y$ for all $x, y \in
      S$;
    \item[transitivity:] for any $x, y, z \in S$, if $x R y$ and $y R Z$, then
      $x R z$;
  \end{description}

  We say that an order $R$ on a set $S$ is total iff for any $x, y \in S$,
  either $x R y$ or $y R x$.
\end{definition}
\noindent Note that if $S$ is a set of numbers, then $\le$ defines a partial
ordering on $S$; moreover, it defines a total order.

Typically we use symbols similar to $\preceq$ to denote partial orderings and we
write $a \prec b$ to express that $a \preceq b$ and $a \neq b$.

Let $\divides$ be the relation on $\Z$ such that $d \divides n$ iff $d$
divides $n$.
\nomenclature[R]{$a \divides b$}{says that $a$ divides $b$}
\begin{theorem}
  The relation $\divides$ is a partial ordering of the set $\N$.
\end{theorem}
\begin{proof}
  To prove that this relation is a partial ordering we need to check all three
  properties.
  \begin{description}
    \item[reflexivity:] Note that $x = 1 \cdot x$ for any integer $x$;
      hence, $x \divides x$ for any integer $x$.
    \item[antisymmetry:] Assume that $x \divides y$ and $y \divides x$. Note
      that it means that $k x = y$ and $\ell y = x$ for some integers $k$ and
      $\ell$. Hence, $y = (k \cdot \ell) y$ which implies that
      $k \cdot \ell = 1$ and $k = \ell = 1$. Thus, $x = y$.
    \item[transitivity:] finally, assume that $x \divides y$ and $y \divides z$;
    i.e. $k x = y$ and $\ell y = z$. As a result, $(k \cdot \ell) x = z$ and
    $x \divides z$.
  \end{description}
\end{proof}

\begin{exercise}
  Let $S$ be some set, show that $\subseteq$ defines a partial ordering on the
  set $2^S$.
\end{exercise}

\subsection{Topological Sorting}
Partial orderings are very useful for describing complex processes. Suppose
that some process consists of several tasks, $T$ denotes the set of these tasks.
Some tasks can be done only after some others e.g. when you cooking a salad you
need to wash vegetables before you chop them. If $x, y \in T$ be some tasks, $x
\preceq y$ if $x$ should be done before $y$ and this is a partial ordering.

In the applications this order is not a total order because some steps do not
depend on other steps being done first (you can chop tomatoes and chop
cucumbers in any order). However, if we need to create a schedule in which the
tasks should be done, we need to create a total ordering on $T$. Moreover, this
order should be compatible with the partial ordering. In other words, if $x
\preceq y$, then $x \preceq_t y$ for all $x, y \in T$, where $\preceq_t$ is the
total order. The technique of finding such a total ordering is called
\emph{topological sorting}.

\begin{theorem}
\label{theorem:topological-order}
  Let $S$ be a finite set and $\preceq$ be a partial order on $S$. Then there is
  a total order $\preceq_t$ on $S$ such that if $x \preceq y$, then
  $x \preceq_t y$ for all $x, y \in S$
\end{theorem}

This sorting can be done using the following procedure.
\begin{itemize}
  \item Initiate the set $S$ being equal to $T$
  \item Choose the minimal element of the set $S$ with respect to the ordering
    $\preceq$ (such an element exists since $S$ is a finite set, see
    Chapter~\ref{chapter:bijections-surjections-injections}).
    Add this element to the list, remove it from the set $S$, and repeat this
    step if $S \neq \emptyset$.
\end{itemize}

Let us consider the following example. In the left column we list the classes
and in the right column the prerequisite.
\begin{center}
  \begin{tabular}{c | c}
    Courses & Prerequisite \\
    \hline
    Math 20A \\
    Math 20B & Math 20A \\
    Math 20C & Math 20B \\
    Math 18 \\
    Math 109 & Math 20C, Math 18 \\
    Math 184A & Math 109
  \end{tabular}
\end{center}
We need to find an order to take the courses.
\begin{enumerate}
  \item We start with
    \[
      S = \set{\text{Math 20A}, \text{Math 20B},
        \text{Math 20C}, \text{Math 18}, \text{Math 109}, \text{Math 184}}.
    \]
    There are two minimal elements: Math 20A and Math 18. Let us remove Math 18
    from $S$ and add it to the resulting list $R$.
  \item Now we have
    \[
      R = \text{Math 18}
    \]
    and
    \[
      S = \set{\text{Math 20A},
      \text{Math 20B}, \text{Math 20C}, \text{Math 109}, \text{Math 184}}.
    \]
    There is only one minimal element Math 20A. We remove it and add it to the
    list $R$.
  \item On this step
    \[
      R = \text{Math 18}, \text{Math 20A}
    \]
    and
    \[
      S = \set{
        \text{Math 20B}, \text{Math 20C}, \text{Math 109}, \text{Math 184}}.
    \]
    Again there is only one minimal element: Math 20B.
  \item
    \[
      R = \text{Math 18}, \text{Math 20A}, \text{Math 20B}
    \] and
    \[
      S = \set{\text{Math 20C}, \text{Math 109}, \text{Math 184}}.
    \]
    There is only one minimal element: Math 20C.
  \item
    \[
      R = \text{Math 18}, \text{Math 20A}, \text{Math 20B}, \text{Math 20C}
    \]
    and
    \[
      S = \set{\text{Math 109}, \text{Math 184}}.
    \]
    There is only one minimal element: Math 109.
  \item Finally,
    \[
      R = \text{Math 18}, \text{Math 20A}, \text{Math 20B}, \text{Math 20C},
      \text{Math 109}
    \]
    and
    \[
      S = \set{\text{Math 184}}.
    \]
    There is only one minimal element: Math 184A.
\end{enumerate}
As a result, the final list is
\[
  R = \text{Math 18}, \text{Math 20A}, \text{Math 20B}, \text{Math 20C},
  \text{Math 109}, \text{Math 184A}.
\]


\begin{chapterendexercises}
  \exercise[recommended] Show that the relation $\divides$ does not define a partial
    ordering on $\Z$.
  \exercise Let a relation $R$ be defined on the set of real numbers as
    follows: $x R y$ iff $2x + y = 3$. Show that it is antisymmetric.
  \exercise Are there any minimal elements in $\N$ with respect to
    $\divides$? Are there any maximal elements?
\end{chapterendexercises}
