\chapter{Structural Induction}
To illustrate this chapter let us consider the following game:
Alice have chosen a number from $1$ to $1000$. Bob want to guess the number
so he can ask ``yes'' or ``no'' questions to Alice.
How many questions Bob needs to ask to guess the number?

The following simple algorithm allows Bob to guess the number using $10$
questions.
\begin{enumerate}
  \item Bob start with two numbers $\ell = 1$ and $u = 1000$.
  \item \label{algorithm-step:binary-search-decision-guess-number}
    Bob asks whether the Alice's number is smaller than $(\ell + u) / 2$.
    If the answer is yes, then Bob replace $u$ by $(\ell + u) / 2$;
    otherwise Bob replace $\ell$ by $(\ell + u) / 2$.
  \item \label{algorithm-step:binary-search-result-guess-number}
    If there is only one integer $x$ between $\ell$ and $u$, Bob
    says that Alice's number is $x$. Otherwise go to
    step~\ref{algorithm-step:binary-search-decision-guess-number}
\end{enumerate}

Now we have two problems:
\begin{enumerate}
  \item we need to prove that the algorithm is correct and
  \item we would like to prove that it is impossible to guess the number
    using less questions.
\end{enumerate}

To study both these problems we need a formal definition of an algorithm that
Bob may use. However, the first problem is somewhat less demading and
the solution can be explained without a precise definition.

First note that the Alice's number always between $\ell$ and $u$. Hence,
the answer given at step~\ref{algorithm-step:binary-search-result-guess-number}
is always correct. You may also notice that on each step $u - \ell$
decreases by $2$, hence, after $\ceil{\log 1000} = 10$ questions $u - \ell < 1$.
As a result, after $10$ questions, there is only one integer between $\ell$ and
$u$.

\begin{exercise}
  Give a nonadaptive algorithm for Bob that allows him to guess the number
  using $10$ queries. In other words, write $10$ questions such that
  answers to these questions allow Bob to guess the number.
\end{exercise}

\section{Recursive Definitions}
First note that any question for Alice can be formulated as follows:
``Is the value of a function $f$ at your number equal to $\ltrue$?'', where
$f$ is a function from $\Z$ to $\set{0, 1}$ (here and in the sequel
we interpret $1$ as ``yes'' and $0$ as ``no'').

Hence, there are two possible behaviours of any algorithm for Bob.
\begin{itemize}
  \item The algorithm prescribe Bob to just say that the answer is some
    number $x \in \Z$, or
  \item The algorithm prescribe Bob to ask whether $f$ at Alice's number is
    equal to $\ltrue$. If the answer is yes, then prescribe Bob to proceed
    according to an algorithm $A_1$, otherwise it prescribe Bob to proceed
    according to an algorithm $A_0$.
\end{itemize}

Hence, any algorithm for Bob can be described using the following object.
\begin{definition}
  We say that $T$ is a $B$-decision tree if
  \begin{description}
    \item [(base case)] either $T$ is equal to an integer, or
    \item [(recursion step)] $T$ is equal to $(f, T_0, T_1)$,
      where $f : \Z \to \set{0, 1}$, and $T_0$ and $T_1$
      are $B$-decision trees.
  \end{description}
\end{definition}
Note that this definition is not quite formal since it is recursive and
we usually do not allow recursive definitions. So we will need to give a more
formal way to define $B$-decision trees. However, this definition allows us to
prove that $(f_1, (f_2, 1, 2), (f_3, 3, 4))$
is a $B$-decision tree, where
\begin{gather*}
  f_1(x) =
    \begin{cases}
      1 & \text{if } x \le 2 \\
      0 & \text{otherwise}
    \end{cases}, \\
  f_2(x) =
    \begin{cases}
      1 & \text{if } x = 1 \\
      0 & \text{otherwise}
    \end{cases}, \\
  \text{and} \\
  f_3(x) =
    \begin{cases}
      1 & \text{if } x = 3 \\
      0 & \text{otherwise}
    \end{cases}.
\end{gather*}
This can be explained as follows.
\begin{itemize}
  \item It is clear that $1$, $2$, $3$, and $4$ are $B$-decision trees by the
    base case.
  \item Hence, by the recurstion step case, $(f_2, 1, 2)$ and $(f_3, 3, 4)$
    are $B$-decision trees.
  \item Finally, by the recurstion step case, $(f_1, (f_2, 1, 2), (f_3, 3, 4))$
    is a $B$-decision tree.
\end{itemize}
In other words we proved that $(f_1, (f_2, 1, 2), (f_3, 3, 4))$
is a $B$-decision tree by providing $T_1$, \dots, $T_7$ such that
$T_7 = (f_1, (f_2, 1, 2), (f_3, 3, 4))$ and for each $i \in [7]$,
$T_i$ is a number from $\Z$ or $T_i = (f, T_j, T_k)$ for $j, k < i$ and
$f : \Z \to \set{0, 1}$.

This idea leads to the framework that would allows us to give a formal
definition of $B$-decision trees.
\begin{definition}
  Let $U$ be a set, let $U_0 \subseteq U$, and let
  \[
    \mathcal{F} =
    \set{F_1 : U^{\ell_1} \to U, \dots, F_n : U^{\ell_n} \to U, \dots}.
  \]
  The we say that the set $S$ generate by $\mathcal{F}$ from $U_0$ is
  the set of all $u \in U$ such that there is a sequence
  $u_1$, \dots, $u_m$ satisfying the following constraints: $u_m = u$ and
  for each $i \in \range{m}$,
  \begin{itemize}
    \item either $u_i \in U_0$, or
    \item $u_i = F(u_{k_1}, \dots, u_{k_\ell})$
      for $F \in \mathcal{F}$ and $k_1, \dots, k_\ell < i$.
  \end{itemize}
\end{definition}
In case of $B$-decision trees $U$ is the set of all sequences of numbers,
parentheses, commas, and functions from $\Z$ to $\set{0, 1}$,
$U_0$ is the set of all sequences consisting of one number, and
\[
  \mathcal{F} =
  \set[
    {
      f \text{ is a function from } \Z
        \text{ to } \set{0, 1}
    }
  ]{F_f},
\]
where $F_f(T_0, T_1) = (f, T_0, T_1)$.

Using similar ideas we may define some functions on the objects defined
recursively.
\begin{definition}
  Let $T$ be a $B$-decision tree. Then the height $h(T)$ of $T$ can be
  defined as follows.
  \begin{description}
      \item [(base case)] If $T$ is equal to a number $x \in \Z$,
        then $h(T) = 0$.
      \item[(recursion step)] If $T_1$ and $T_2$ are $B$-decision trees, then
        $h\big((f, T_1, T_2)\big) = \max(h(T_1), h(T_2)) + 1$.
  \end{description}
\end{definition}
Note that $h(T)$ correpsonds to the worst-case number of queries made by Bob if
we interpret $T$ as a description of an algorithm.

However, before we explain how to formalize such a definition, we need to note
that in the general case such definition may be contradictory. Consider
$U = \R$, $U_0 = \set{0}$, and $\mathcal{F} = \set{f, g}$, where $f(x, y) = xy$
and $g(x) = x + 1$. We define $v : U \to \R$ as follows.
\begin{description}
    \item [(base case)] $v(0) = 0$.
    \item[(recursion step)] $v(f(x, y)) = f(v(x), v(y))$ and
      $v(g(x)) = v(x) + 1$.
\end{description}
Note that $v(f(g(0), g(0))) = f(v(g(0)), v(g(0))) = \big(v(g(0))\big)^2 = 4$
and $v(g(0)) = v(0) + 2 = 2$.
However, $g(0) = 1$ and $f(g(0), g(0)) = 1$.

Therefore handle such an issue, we consider $S$ that is \emph{freely} generated
from $U_0$ by by $\mathcal{F}$.
\begin{definition}
  The set $S$ is freely generated from $U_0$ by by $\mathcal{F}$ iff it is
  generated by $\mathcal{F}$ from $B$,
  $U_0 \cap \Im F = \emptyset$, and $\Im F \cap \Im G = \emptyset$ for any
  $F, G \in \mathcal{F}$.
\end{definition}

The following theorem claims existence of functions defined recursively.
\begin{theorem}
\label{theorem:recursion-principle}
    Let $S \subseteq U$ be the set freely generated from $U_0 \subseteq U$ by
    $\mathcal{F} =
      \set{F_1 : U^{\ell_1} \to U, \dots, F_n : U^{\ell_n} \to U, \dots}$.
    In addition, let $G_0 : U_0 \to V$ and
    $G_1 : V^{\ell_1} \to V, \dots, G_n : V^{\ell_n} \to V$, \dots
    be some functions.

    Then there is a function $h : S \to V$ such that
    \begin{description}
        \item [(base case)] $h(u) = G_0(u)$ for any $u \in U_0$.
        \item[(recursion step)] $h(F_i(u_1, \dots, u_{\ell_i})) =
            G_i(h(u_1), \dots, h(u_{\ell_i}))$ for any $i$ and
            $u_1, \dots, u_{\ell_i} \in S$.
    \end{description}
\end{theorem}

\begin{exercise}
    Prove \Cref{theorem:recursion-principle}.
\end{exercise}

$B$-decision tree are used in this chapter to represent Bob's algorithms;
hence, we need to define values of $B$-decision trees.
\begin{definition}
  The value
  $\mathrm{val}(T, n)$ of a $B$-decision tree $T$ at an integer $n$ can
  be defined as follows.
  \begin{description}
      \item [(base case)] If $T$ is just an number $x \in \range{1000}$,
        then the number the result $\mathrm{val}(T, n)$ is equal to $x$.
      \item[(recursion step)] If $T_1$ and $T_2$ are two representations of
        Bob's algorithms, then
        \[
          \mathrm{val}\big((f, T_0, T_1), n\big) =
          \begin{cases}
            \mathrm{val}(T_0, n) & \text{if } f(x) = 0 \\
            \mathrm{val}(T_1, n) & \text{otherwise}
          \end{cases}.
        \]
  \end{description}
\end{definition}

Using all these notions we can reformulate the results about Alice and Bob's
game.
\begin{theorem}
\label{theorem:guess-the-number}
  \begin{enumerate}
    \item There is a $B$-decision tree $T$ such that
      \begin{itemize}
        \item $h(T) \le 10$ and
        \item $\mathrm{val}(T, n) = n$ for any $n \in \range{1000}$.
      \end{itemize}
    \item Let $T$ be a $B$-decision tree such that $\mathrm{val}(T, n) = n$ for
      any $n \in \range{1000}$. Then $h(T) \ge 10$.
  \end{enumerate}
\end{theorem}

\begin{exercise}
  Prove the first part of \Cref{theorem:guess-the-number}.
\end{exercise}

\section{Structural Induction Theorem}
To prove the second part of \Cref{theorem:guess-the-number} we need
to introduce the notion of structural induction.

\begin{theorem}[The Structural Induction Principle]
\label{theorem:structural-induction}
    Let $S \subseteq U$ be the set freely generated from $U_0 \subseteq U$ by
    $\mathcal{F} =
      \set{F_1 : U^{\ell_1} \to U, \dots, F_n : U^{\ell_n} \to U, \dots }$.

    Assume that $S' \subseteq U$ is a set such that the following constraints
    are true.
    \begin{description}
        \item [(base case)] $U_0 \subseteq S'$
        \item[(induction step)]
          $F_i(u_1, \dots, u_{\ell_i}) \in S'$ for any
          $u_1, \dots, u_{\ell_i} \in S'$ and $i \in \N$.
    \end{description}
    Then $S \subseteq S'$.
\end{theorem}

Using this result, we may prove \Cref{theorem:guess-the-number}.
\begin{proof}[Proof of \Cref{theorem:guess-the-number}]
  We need to prove only the second part of the statement.
  Let $V(T) = \set[n \in \Z]{\mathrm{val}(T, n)}$, where $T$ is a $B$-decision
  tree. This proof is based on the following observation.
  \begin{claim}
  \label{claim:guess-the-number}
    For any $B$-decision tree $T$, size of $V(T)$ is at most $2^{h(T)}$.
  \end{claim}

  Assume that $T$ is a $B$-decision tree such that $\mathrm{val}(T, n) = n$ for
  any $n \in \range{1000}$. Whence $\range{1000} \subseteq V(T)$. Therefore 
  $V(T)$ has at least $1000$ elements. As a result, $2^{h(T)} \ge 1000$;
  which implies that $h(T) \ge 10$.

  So we just need to prove \Cref{claim:guess-the-number}. We prove it using
  structural induction. Let $S'$ be the set of decision trees $T$ such that size
  of $V(T)$ is at most $2^{h(T)}$.
  \begin{description}
    \item[(base case)] If $T$ is equal to an integer $x$, then 
      $\mathrm{val}(T, n) = x$ for any $n \in \Z$. Hence, the size of $V(T)$ is
      equal to $1 = 2^0 = 2^{h(T)}$.
    \item[(induction step)] Assume that $T = (f, T_0, T_1)$ for some 
      $T_0, T_1 \in S'$. We know that the size of $V(T_0)$ is at most
      $2^{h(T_0)}$ and the size of $V(T_0)$ is at most $2^{h(T_1)}$. In 
      addtion, it is clear that $V(T) \subseteq V(T_0) \cup V(T_1)$. Therefore,
      the size of $V(T)$ is at most\footnote{%
        Formally speaking, we use \Cref{theorem:additive-principle} to prove
        this; i.e. we use the fact that the size of a set $A \cup B$ is at most
        the size of $A$ plus the size of $B$.
      }
      \[
        2^{h(T_0)} + 2^{h(T_1)} \le 2^{\max(h(T_0), h(T_1)) + 1} = 2^{h(T)}.
      \]
  \end{description}
  Hence, by \Cref{theorem:structural-induction}, $S'$ is equal to the set of all
  $B$-decision trees.
\end{proof}

Now we are ready to prove \Cref{theorem:structural-induction}.
\begin{proof}[Proof of \Cref{theorem:structural-induction}]
    We prove the statement using induction. More precisely, we prove using
    induction by $m$ that if there is a sequence
    $u_1$, \dots, $u_m$ such that
    for each $i \in \range{m}$, $u_i \in U_0$ or $u_i = F(u_{k_1}, \dots, u_{k_\ell})$
    for $F \in \mathcal{F}$ and $k_1, \dots, k_\ell < i$, then $u_m \in S'$.

    The case when $m = 1$ is clear since in this case $u_1 \in U_0$ which implies
    that it is in $S'$.

    Let us now prove the induction step. Assume that the statement is true for any
    $k \le m$. Consider a sequence $u_1$, \dots, $u_{m + 1}$ such that
    for each $i \in [m + 1]$,
    $u_i \in U_0$ or $u_i = F(u_{k_1}, \dots, u_{k_\ell})$
    for $F \in \mathcal{F}$ and $k_1, \dots, k_\ell < i$. Let us consider
    $F \in \mathcal{F}$ and $k_1, \dots, k_\ell < m + 1$ such that
    $u_{m + 1} = f(u_{k_1}, \dots, u_{k_\ell})$. By the induction hypothesis,
    $u_{k_1}, \dots, u_{k_\ell} \in S'$. Therefore, by the properties of $S'$,
    $u_{m + 1} \in S'$.
\end{proof}

\begin{chapterendexercises}
    \exercise[recommended]
        Using recursive definition we can define an arithmetic formula on the
        variables $x_1$, \dots, $x_n$.
        \begin{description}
            \item [(base case)] $x_i$ is an arithmetic formula on the variables $x_1$,
                \dots, $x_n$ for all $i$; if $c$ is a real number, then $c$ is also
                an arithmetic formula on the variables $x_1$, \dots, $x_n$.
            \item[(recursion step)] If $P$ and $Q$ are arithmetic formulas on the variables
                $x_1$, \dots, $x_n$, then $(P + Q)$ and $P \cdot Q$ are arithmetic formulas
                on the variables $x_1$, \dots, $x_n$.
        \end{description}

        We can also define recursively the value of such a formula.
        Let $v_1$, \dots, $v_n$ be some integers.
        \begin{description}
            \item[(base cases)] $x_i\big\rvert_{x_1 = v_1, \dots, x_n = v_n} = v_i$; in
                other words, the value of the arithmetic formula $x_i$ is equal to $v_i$
                when $x_1 = v_1$, \dots, $x_n = v_n$; if $c$ is a real number, then
                $c\rvert_{x_1 = v_1, \dots, x_n = v_n} = c$.
            \item[(recursion steps)] If $P$ and $Q$ are arithmetic formulas on the
            variables $x_1$, \dots, $x_n$, then
            \[
                (P + Q)\big\rvert_{x_1 = v_1, \dots, x_n = v_n} =
                P\big\rvert_{x_1 = v_1, \dots, x_n = v_n} +
                Q\big\rvert_{x_1 = v_1, \dots, x_n = v_n}
            \]
            and
            \[
                (P \cdot Q)\big\rvert_{x_1 = v_1, \dots, x_n = v_n} =
                P\big\rvert_{x_1 = v_1, \dots, x_n = v_n} \cdot
                Q\big\rvert_{x_1 = v_1, \dots, x_n = v_n}.
            \]
        \end{description}

        Prove that for any arithmetic formula $A$ on $x$, there is a polynomial
        $p$ such that $p(v) = A\big\rvert_{x = v}$ for any $v \in \R$.
    \exercise
        \begin{itemize}
            \item Define arithmetic formulas with division and define their value (make
                sure that you handled divisions by $0$).

            \item Show that for any arithmetic formula with division $A$ on $x$,
                there are polynomials $p$ and $q$ such that $\frac{p(v)}{q(v)} =
                A\big\rvert_{x = v}$ or $A\big\rvert_{x = v}$ is not defined for
                any real value $v$.
        \end{itemize}
\end{chapterendexercises}
