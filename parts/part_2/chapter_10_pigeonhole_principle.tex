\chapter{The Pigeonhole Principle}
We know that if there is an injection from $[n]$ to $[m]$ then $n \le m$.
Contraposition of this statement says that if $n > m$, then any function from
$[n]$ to $[m]$ is not an injection. This simple statement is faymous in
mathematics and called \textit{the pigeonhole principle}.

\begin{theorem}
  Let $X$ and $Y$ be some sets such that $|X| > |Y|$. Then for any function
  $f : |X| \to |Y|$ there are $x_0 \neq x_1 \in X$ such that $f(x_0) = f(x_1)$.
\end{theorem}

This simple statement is very handy in combinatorics. For example, using this
statement one may prove that in this classroom there are two people who were
born in the same month.

Assume that there are $n$ people in the classroom, note that $n > 12$.
Consider the following function $f : [n] \to [12]$ such that $f(i) = j$ if the
$i$th person was born in $j$th month. Note that $f$ is not an injection since
$n > 12$ i.e. there are $i_0 \neq i_1$ such that $i_0$th and $i_1$th person are
born in the same month.

We may also prove that in this classroom there are two people who are
friends with the same number of people in the classroom.

Assume the number of people is $n$. It is easy to see that every person may
have at most $n - 1$ friends. Hence, we may define a function $f: [n] \to \set{0, \dots, n - 1}$ such that $f(i)$ is equal to the number of friends in this
classroom of the $i$th person in this classroom.
We need to consider two cases.
\begin{itemize}
  \item If $\mathrm{Im}f \subseteq [n - 1]$. In this case
    $|[n]| > |\mathrm{Im}f|$ and $f$ is not an injection.
  \item Otherwise, note that it is not possible that $(n - 1) \in \mathrm{Im}f$
    since it there is a friend of nobody it is not possible that there is a
    friend of everyone. Hence, $f : [n] \to \set{0, 1, \dots, n - 2}$ and $f$ is
    not an injection.
\end{itemize}

\section*{End of The Chapter Exercises}
\begin{exercises}
  \exerciseitem Prove that for every integers $a_1, \dots, a_n$ there are
    $k > 0$ and $\ell \ge 0$ such that $k + \ell \le n$ and
    $\sum\limits_{i = k}^{k + \ell} a_i$ is divisible by $n$.
  \exerciseitem Let $S \subseteq [20]$ be a set. Show that if
    $|S| \ge 13$, then there are $a, b \in S$ such that $a - b = 6$.
  \exerciseitem Sasha is training for a triathlon. Over a $30$ day period, he
    pledges to train at least once per day, and $45$ times in all. Then there
    will be a period of consecutive days where he trains exactly $14$ times.
\end{exercises}
