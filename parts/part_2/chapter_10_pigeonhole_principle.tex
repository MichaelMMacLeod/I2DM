\chapter{The Pigeonhole Principle}
\marginurl{%
  The Pigeonhole Principle:\\\noindent
  Introduction to Combinatorics \#3
}{youtu.be/1D1Fa7WIUO8}

The principle we are going to discuss in this chapter is very simple, it states
that if you have more objects than boxes, then you cannot put all the objects to
boxes without puting two objects in the same box.

More formally the principle can be formulated as follows: if $n > m$, then any
function from $[n]$ to $[m]$ is not an injection. This simple statement is
famous in mathematics and called \emph{the pigeonhole principle}\footnote{%
  The pigeonhole principle is also called the Dirichlet principle, after the
  German mathematician G. Lejeune Dirichlet, who demonstrated, using this
  principle, that there were at least two Parisians with the same number of
  hairs on their heads.
}.

\begin{theorem}[the pigeonhole principle]
  Let $X$ and $Y$ be some sets such that $|X| > |Y|$. Then for any function
  $f : |X| \to |Y|$ there are $x_0 \neq x_1 \in X$ such that $f(x_0) = f(x_1)$.
\end{theorem}
\begin{proof}
  The statement follows from
  Theorem~\ref{theorem:injections-surjections-inequalities}.
\end{proof}

This simple statement is very handy in combinatorics. For example, using this
statement one may prove that in any group of more than $12$ people there are
two people who were born in the same month.

Assume that there are $n$ people in the group and $n > 12$.
Consider the following function $f : [n] \to [12]$ such that $f(i) = j$ if the
$i$th person was born in $j$th month. Note that $f$ is not an injection since
$n > 12$ i.e. there are $i_0 \neq i_1$ such that $i_0$th and $i_1$th person are
born in the same month.

We may also prove that in any group of people there are two people who are
friends with the same number of people in the group.

Assume the number of people is $n$. It is easy to see that every person may
have at most $n - 1$ friends. Hence, we may define a function $f: [n] \to
\set{0, \dots, n - 1}$ such that $f(i)$ is equal to the number of friends in
this group of the $i$th person in this group.
We need to consider two cases.
\begin{itemize}
  \item If $\Im f \subseteq [n - 1]$. In this case
    $|[n]| > |\Im f|$ and $f$ is not an injection.
  \item Otherwise, note that it is not possible that $(n - 1) \in \Im f$
    since it there is a friend of nobody it is not possible that there is a
    friend of everyone. Hence, $f : [n] \to \set{0, 1, \dots, n - 2}$ and $f$ is
    not an injection.
\end{itemize}

\begin{theorem}[Erd\H{o}s–-Szekeres]
  Every sequence of $(r - 1)(s - 1) + 1$ distinct real numbers contains a
  subsequence of length $r$ that is increasing or a
  subsequence of length $s$ that is decreasing.
\end{theorem}
\begin{proof}
  Given a sequence of length $(r - 1)(s - 1) + 1$, label each number $x_i$ in
  the sequence with the pair $(a_i, b_i)$, where $a_i$ is the length of the
  longest increasing subsequence ending with $x_i$ and $b_i$ is
  the length of the longest decreasing subsequence ending with $x_i$.
  Each two numbers in the sequence are labeled with a different pair: if $i < j$
  and $x_i < x_j$ then $a_i < a_j$, and on the other hand if $x_i > x_j$ then
  $b_i < b_j$. But there are only $(r - 1)(s - 1)$ possible labels if $a_i$ is
  at most $r - 1$ and $b_i$ is at most $s - 1$, so by the pigeonhole principle
  there must exist a value of $i$ for which $a_i$ or $bi$ is outside this
  range. If ai is out of range then $x_i$ is part of an increasing sequence of
  length at least $r$, and if $b_i$ is out of range then $x_i$ is part of a
  decreasing sequence of length at least $s$.
\end{proof}

\section{The Generalized Pigeonhole Principle}
One may generalize the pigeonhole principle in the following way.
If $N$ objects are placed into $k$ boxes, then there is at least one box
containing at least $\ceil{N / k}$ objects.
\begin{theorem}[the generalized pigeonhole principle]
\label{theorem:generalized-pigeonhole-principle}
  Let $X$ and $Y$ be some sets. Then for any function $f : |X| \to |Y|$ there
  are $x_1, \dots, x_\ell \in X$ such that
  \begin{itemize}
    \item $f(x_i) = f(x_j)$,
    \item $x_i \neq x_j$ for any $i \neq j \in [\ell]$, and
    \item $\ell \ge \ceil{|X| / |Y|}$
  \end{itemize}
\end{theorem}

Now we illustrate applications of this principle on some examples and prove the
statement in the next section.

Using this theorem we can prove that if we draw $9$ cards out of a deck of
cards, we are guaranteed that at least three of them are of the same suit.
Indeed, there are $4$ suits and by pigeonhole principle if we put each card to
one out of four boxes according to their suit, one of the boxes should have
at least $\ceil{9 / 4} = 3$ cards.

Another example shows how the generalized pigeonhole principle can be applied
to an important part of combinatorics called Ramsey theory.

Assume that in a group of six people, each pair of individuals consists of two
friends or two enemies. One may prove that there are either three mutual
friends or three mutual enemies in the group.

Let $A$ be one of the six people; of the five other people in the group, there
are either three or more who are friends of $A$, or three or more who are
his enemies $A$. This statements follows from the generalized pigeonhole
principle since when five objects are divided into two sets, one of the sets
has at least $\ceil{5 / 2} = 3$ elements. Without loss of generality we may
suppose that $B$, $C$, and $D$ are friends of $A$. If any two of these three
individuals are friends, then these two and $A$ form a group of three mutual
friends. Otherwise, $B$, $C$, and $D$ form a set of three mutual enemies.

\section{The Averaging Principle}
Assume that we have a collection of $m$ objects, the $i$th of which has
``size'' $l_i$, and we wish to show that at least one of the objects is large.
In this situation we can argue that at least one of the objects has size
greater or equal to the average size ($\sum l_i / m$).
\begin{theorem}[the averaging principle]
\label{theorem:averaging-principle}
  Every sequence of numbers has a number at least as large as the average and a
  number at least as small as the average; i.e. for any sequence $a_1$, \dots,
  $a_m$ there are $i$ and $j$ such that
  \begin{gather*}
    a_i \ge \frac{1}{m} \sum_{i = 1}^n a_i \\
    \text{and} \\
    a_j \le \frac{1}{m} \sum_{i = 1}^n a_i.
  \end{gather*}
\end{theorem}
\begin{proof}
  We prove only the existence of $i$, proof of the existence of $j$ is almost
  the same.

  Assume the opposite, i.e. that $a_i < \sum_{i = 1}^n a_i / m$
  for any $i \in [n]$. Note that this implies that
  $\sum_{i = 1}^n a_i \le m \cdot \sum_{i = 1}^n a_i / m = \sum_{i = 1}^n a_i$.
  Which is a contradiction.
\end{proof}

\begin{exercise}
  Finish the proof of Theorem~\ref{theorem:averaging-principle}
\end{exercise}

Like the pigeonhole principle, this principle is very simple but the
applications of it are surprisingly intresting.

First, it allows to prove the generalized pigeonhole principle.
\begin{proof}[Proof of Theorem~\ref{theorem:generalized-pigeonhole-principle}]
  Let $Y = [m]$ (it is easy to see that the proof works for any other finite
  $Y$). Define the sequence $a_i = |f^{-1}(i)|$. %TODO add the definition of $f^{-1}$ in the case when $f$ is not a bijection.
  Note that we need to prove that $a_i \ge \ceil{|X| / m}$ for some
  $i \in [m]$

  It is clear that $\bigcup_{i = 1}^m f^{-1}(i) = X$ and that $f^{-1}(i) \cap
  f^{-1}(j) = \emptyset$ for any $i \neq j \in [m]$. Thus, by the additive
  principle, $\sum_{i = 1}^m a_i = |X|$. Hence, by the averaging principle,
  $a_i \ge |X| / m$ for some $i \in [m]$. However, $a_i$ is an integer, thus
  $a_i \ge \ceil{|X| / m}$.
\end{proof}

Another nice application of the averaging principle allows us to prove that if
in some group (with more than one person) the number of pairs of people who
know each other is less than $n - 1$, then we can split this group into two
subgroups such that people from different subgroups do not know each other.

Let us assume that there are $n$ people in the group. We prove the statement
using the induction by $n$.
\begin{description}
  \item [(the base case)] If $n = 2$, there are less than $n - 1 = 1$ pairs
    of people who know each other, in other words, these two people in the
    group do not know each other. Thus we can put each of them into a separate
    subgroup.
  \item [(the induction step)] Let $p_i$ ($i \in [n]$) be the number of
    acquaintances of the $i$th person. Note that
    $\sum_{i = 1}^n p_i \le 2(n - 2)$ since we count each pair twice.
    By the averaging principle, $p_i \le 2(n - 2) / n = 2 - 2 / n$ for some
    $i \in [n]$.  Thus $p_i$ is either $0$ or $1$.
    \begin{itemize}
      \item If $p_i = 0$, we can put the $i$th person into the first subgroup
        and everyone else into another.
      \item If $p_i = 1$ we consider the group of $n - 1$ people without the
        $i$th person, by the induction hypothesis, we can split everyone but
        $i$th person into two subgroups and since the $i$th person has only one
        acquaintance we can put them in the same subgroup.
      \end{itemize}
\end{description}



\begin{exercises}
  \exercise Show that among any group of five (not necessarily consecutive)
    integers, there are two with the same remainder when divided by $4$.
  \exercise Show that if there are 30 students in a class, then at least
    two have last names that begin with the same letter.
  \exercise Let $n$ be a positive integer. Show that in any set of $n$
    consecutive integers there is exactly one divisible by $n$.
  \exercise Prove that for every integers $a_1$, \dots, $a_n$ there are
    $k > 0$ and $\ell \ge 0$ such that $k + \ell \le n$ and
    $\sum_{i = k}^{k + \ell} a_i$ is divisible by $n$.
  \exercise Let $S \subseteq [20]$ be a set. Show that if
    $|S| \ge 13$, then there are $a, b \in S$ such that $a - b = 6$.
  \exercise How many numbers must be selected from the set $[6]$ to
    guarantee that at least one pair of these numbers add up to $7$?
  \exercise Sasha is training for a triathlon. Over a $30$ day period, he
    pledges to train at least once per day, and $45$ times in all. Then there
    will be a period of consecutive days where he trains exactly $14$ times.
  \exercise Show that among any $n + 1$ positive integers not exceeding $2n$
    there must be an integer that divides one of the other integers.
    \emph{Advise: consider the set of holes equal to the set of odd numbers
    from $1$ to $2n$.}
  \exercise Let $a_1$, $a_2$, \dots, $a_t$ be positive integers. Show that
    if $a_1 + a_2 + \dots + a_t - t + 1$ objects are placed into $t$ boxes,
    then for some $i \in [t]$, the $i$th box contains at least $a_i$ objects.
  \exercise Let $\set{(x_1,y_1), \dots, (x_5, y_5)} \subseteq \Z^2$ be a
    set of five distinct points with integer coordinates in the xy plane. Show
    that the midpoint of the line joining at least one pair of these points has
    integer coordinates.
\end{exercises}
