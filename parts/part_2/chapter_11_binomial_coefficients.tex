\chapter{Permutations and Binomial Coefficients}
\section{Counting Functions}
Assume we have two finite sets $X$ and $Y$. The first question we may ask is:
how many function exist from $X$ to $Y$.


\begin{theorem}
\label{theorem:number-of-functions}
  Let $X$ and $Y$ be some finite sets. We denote by $Y^X$ the set of all
  functions from $X$ to $Y$. Then $|Y^X| = |Y|^{|X|}$.
\end{theorem}
\nomenclature[S]{$B^A$}{denotes the set of functions from $A$ to $B$}
\begin{proof}
  For simplicity we prove the statement in the case when $X = [n]$. Fix some
  finite set $Y$. We prove the statement using induction by $n$. The base case
  for $n = 1$ is ovbious, since there are $|Y|$ different functions from $[1]$
  to $Y$. Let us prove the induction step, by the induction hypothesis,
  $|Y^{[n - 1]}| = |Y|^{n - 1}$. Note that
  \begin{multline*}
    |Y^{[n]}| = \left| \set[f \in Y^{[ n - 1]}, y \in Y]{(f, y)} \right| = \\
    |Y^{[n - 1]} \times Y| = |Y^{[n - 1]}| \cdot |Y| = |Y|^n.
  \end{multline*}
\end{proof}

\begin{exercise}
  Finish the proof of Theorem~\ref{theorem:number-of-functions} by proving that
  the statement holds for any set $X$.
\end{exercise}

However, what if we need to find size of a subset of $Y^X$ satisfying some
constraint. For example, we may try to find size of the set
\[
  (Y)_X = \set[f\text{ is an injection}]{f \in Y^X}.
\]
\nomenclature[S]{$(B)_A$}{denotes the set of infunctions from $A$ to $B$}
First, let us try to do this informally. Assume that $X = [n]$ and $|Y| = m$,
to define $f \in (Y)_X$ we need to choose images of $1$, $2$, \dots, $n$. There
are $m$ possible ways to select an image of $1$, $m - 1$ ways to define $f(2)$
since we can not use the value selected for $1$ etc. Hence,
$|(Y)_X| = m (m - 1) \dots (m - n + 1)$ (we denote this number as $(m)_n$).
\nomenclature[C]{$(m)_n$}{denotes the number of ways to choose an
subset of $n$ elements from a fixed set of $n$ elements}

\begin{theorem}
\label{theorem:number-of-injections}
  Let $X$ and $Y$ be some sets. Then $|(Y)_X| = (|Y|)_{|X|}$.
\end{theorem}
\begin{proof}
  Let us prove this statement for $X = [n]$. We prove this using induction by
  $n$. The base case, for $n = 1$, is clear. Now we need to prove the induction
  step from $n$ to $n + 1$. By the induction hypothesis, for any $m$, the
  number of injections from $[n]$ to $Y$ is equal to $(|Y|)_n$.

  Fix some $m$ and some set $Y$ of cardinality $m$. Note that
  \[
    |(Y)_X| =
    |\set[v \not\in \Im f]{(f, v) \in (Y)_{[n - 1]} \times [m]}|.
  \]
  It is easy to see that $|\set[v \not\in \Im f]{(f, v)}| = m - n + 1$
  for any $f \in (Y)_{[n - 1]}$ and
  \[
    \set[v \not\in \Im f]{(f, v) \in (Y)_{[n - 1]} \times [m]} =
    \bigcup\limits_{f \in (Y)_{[n - 1]}} \set[v \not\in \Im f]{(f, v)}.
  \]
  As a result, $|(Y)_X| = (m)_{n - 1} \cdot (m - n + 1) = (m)_n$.
\end{proof}

The special case of this result is that there are
$n \cdot (n - 1) \cdot  \ldots \cdot  1$ different
bijections from $[n]$ to $[n]$. Such bijections are called permutations and the
number is denoted by $n!$.

\begin{exercise}
  Finish the proof of Theorem~\ref{theorem:number-of-injections} by proving that
  the statement holds for any set $X$.
\end{exercise}

\section{Counting Subsets}
Recall that we denoted the set of all subsets of $X$ by $2^X$. The reason for
this notation is that $|2^X| = 2^{|X|}$.

A quite famous example of a subset of this set is the set
\[
  \binom{X}{k} = \set[|A| = k]{A \subseteq [n]}.
\]
\nomenclature[S]{$\binom{A}{k}$}{denotes the set of subsets of $A$ of
cardianlity $k$}
In other words, it is the set of all possible ways to select $k$ elements from
$X$. Size of this set we denote by $\binom{|X|}{k}$ and call it a binomial
coefficient.
\nomenclature[C]{$\binom{m}{n}$}{denotes the number of ways to choose an
unordered subset of $n$ elements from a fixed set of $n$ elements}

Let us find several useful properties of the binomial coefficients.
\begin{theorem}
  Let $n$, $m$, and $k$ be some integers such that $m > n > k$.
  \begin{itemize}
    \item $\binom{n}{k} = \binom{n}{n - k}$.
    \item $\binom{n}{k} = \binom{n - 1}{k - 1} + \binom{n - 1}{k}$.
    \item $\binom{n}{k} = \frac{(m)_n}{n!}$.
    \item $\sum\limits_{k = 0}^n \binom{n}{k} = 2^n$.
  \end{itemize}
\end{theorem}
\begin{proof}
  We prove the second property all the others are left as an exercise.

  Informally, the proof is the following. Assume we need to choose $k$ objects
  out of $n$, there are two possible ways:
  \begin{itemize}
    \item we may select $n$ and choose $k - 1$ objects from the rest,
    \item or we may decide to not select $n$ choose $k$ objects from the rest.
  \end{itemize}
  In the first case we have $\binom{n - 1}{k - 1}$ ways to select objects and
  in the second case we have $\binom{n - 1}{k}$ ways to select objects.

  Let us prove the statement a bit more formally. Note that
  \begin{multline*}
    \binom{[n]}{k} = \set[|A| = k \text{ and } n \in A]{A \subseteq [n]} \cup \\
    \set[|A| = k \text{ and } n \notin A]{A \subseteq [n]}.
  \end{multline*}
  Since these sets are disjoint and $\set[|A| = k \text{ and } n \notin A]{A
  \subseteq [n]} = \binom{[n - 1]}{k}$, we get the following equality
  \[
    \binom{n}{k} = |\set[|A| = k \text{ and } n \in A]{A \subseteq [n]}| +
    \binom{n - 1}{k}.
  \]

  Hence, to finish the proof we need to explain that
  \[
    |\set[|A| = k \text{ and } n \in A]{A \subseteq [n]}| =
    \binom{n - 1}{k - 1}.
  \]
  To prove this statement we construct a bijection $f$
  from \[
    \set[|A| = k \text{ and } n \in A]{A \subseteq [n]}
  \] to
  $\binom{[n - 1]}{k}$: $f(A) = A \setminus \set{n}$.
  It is clear that this is a bijection, thus we prove the statement.
\end{proof}

Now we are ready to prove the theorem which gave the name to binomial
coefficients.
\begin{theorem}[Binomial theorem]
  For any real numbers $x$ and $y$,
    \[
      \sum\limits_{k = 0}^n \binom{n}{k} x^k y^{n - k} = (x + y)^n.
    \]
\end{theorem}
\begin{proof}
  As always, we prove the statement using induction by $n$. The base case is
  true, since $\sum\limits_{k = 0}^1 \binom{1}{k} x^k y^{1 - k} = x + y =
  (x + y)^1$. Asume that
  \[
    \sum\limits_{k = 0}^n \binom{n}{k} x^k y^{n - k} = (x + y)^n,
  \]
  we wish to prove that
  \[
    \sum\limits_{k = 0}^{n + 1} \binom{n + 1}{k} x^k y^{n + 1 - k} =
      (x + y)^{n + 1}.
  \]
  Note that
  \begin{multline*}
    (x + y)^{n + 1} = (x + y)
      \left(
        \sum\limits_{k = 0}^n \binom{n}{k} x^k y^{n - k}
      \right) = \\
    \sum\limits_{k = 0}^n \binom{n}{k} x^{k + 1} y^{n - k} +
      \sum\limits_{k = 0}^n \binom{n}{k} x^{k} y^{n + 1 - k} = \\
    \sum\limits_{k = 0}^{n + 1}
      \left(
        \binom{n + 1}{k - 1} + \binom{n}{k}
      \right)
      x^k y^{n + 1 - k} =
    \sum\limits_{k = 0}^{n + 1} \binom{n + 1}{k} x^k y^{n + 1 - k}.
  \end{multline*}
\end{proof}

\section*{End of The Chapter Exercises}
\begin{exercises}
  \exerciseitem Show that
    $\sum\limits_{i = 1}^k \binom{n}{i} \binom{m}{k - i} = \binom{n + m}{k}$.
  \exerciseitem Shwo that $\sum\limits_{m = k}^n \binom{m}{k} =
    \binom{n + 1}{k + 1}$.
  \exerciseitem Using the previous formula, find the formulas for the following
    expressions:
    \begin{itemize}
      \item $\sum\limits_{k = 0}^n k$,
      \item $\sum\limits_{k = 0}^n k^2$, and
      \item $\sum\limits_{k = 0}^n k^3$.
    \end{itemize}
  \exerciseitem Show that $\sum\limits_{k = 0}^{n} \binom{2n}{2k} =
    \sum\limits_{k = 0}^{n - 1} \binom{2n}{2k + 1}$ and
    $\sum\limits_{k = 0}^{n} \binom{2n + 1}{2k} =
    \sum\limits_{k = 0}^n \binom{2n + 1}{2k + 1}$.
  \exerciseitem Show that $\sum\limits_{k = 0}^n \binom{m + k}{k} =
    \binom{m + n + 1}{n}$.
  \exerciseitem Show that $\sum\limits_{k = 0}^n \binom{n - k}{k} = f_{n + 1}$,
    where $f_1 = 1$, $f_2 = 1$, and $f_{n + 2} = f_{n + 1} + f_n$ for $n > 0$.
  \exerciseitem Show that $\binom{n}{m} \binom{m}{k} =
    \binom{n}{k} \binom{n - k}{m - k}$.
\end{exercises}
