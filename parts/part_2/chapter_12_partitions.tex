\chapter{Partitions}
The main question we study in this chapter is as follows:
``how many ways to put $n$ objects into $k$ boxes''.
Note that there are four modes for this question:
\begin{enumerate}
  \item the objects and boxes are identical,
  \item the objects are identical but boxes are different,
  \item the objects are different but boxes are identical,
  \item the objects and boxes are different.
\end{enumerate}
We are going to study the question in all these modes.
The Table~\ref{table:} summarizes the results we are going
to prove for the cases when all the boxes are not empty.
\begin{table}[h!]
  \centering
  \begin{tabular}{lll}
    \toprule
    Object's name & Parameters & Formula \\
    \midrule
    \multirow{4}{*}{Surjections}
                       & $n$ distinct objects & \multirow{2}{*}{$S(n, k) k!$} \\
                       & $k$ distinct boxes   &  \\
    \rule{0pt}{2ex}
                       & $n$ distinct objects & \multirow{2}{*}{$\sum_{k = i}^n S(n, k) k!$} \\
                       & any number of boxes  & \\
    \rule{0pt}{4ex}
    \multirow{4}{*}{Compositions}
                       & $n$ distinct objects & \multirow{2}{*}{$\binom{n - 1}{k - 1}$} \\
                       & $k$ distinct boxes   &  \\
   \rule{0pt}{2ex}
                       & $n$ distinct objects & \multirow{2}{*}{$2^{n - 1}$} \\
                       & any number of boxes  & \\
   \rule{0pt}{4ex}
   \multirow{4}{*}{Set partitions}
                      & $n$ distinct objects & \multirow{2}{*}{$S(n, k)$} \\
                      & $k$ distinct boxes   &  \\
    \rule{0pt}{2ex}
                      & $n$ distinct objects & \multirow{2}{*}{$B(n)$} \\
                      & any number of boxes  & \\
    \rule{0pt}{4ex}
    \multirow{4}{*}{Integer partitions}
                       & $n$ distinct objects & \multirow{2}{*}{$p_k(n)$} \\
                       & $k$ distinct boxes   &  \\
    \rule{0pt}{2ex}
                       & $n$ distinct objects & \multirow{2}{*}{$p(n)$} \\
                       & any number of boxes  & \\
    \bottomrule
  \end{tabular}
  \caption{Formulas for the cases when boxes are not empty}
\label{table:1}
\end{table}
\section{Set Partitions}
This section considers the case when objects are not identical.

First, we define a notion that allows us to compute the asnwer in case when
all the boxes are the same.
\begin{definition}
  A partition of the set $[n]$ is a collection of non-empty
  blocks so that each element of $[n]$ belongs to exactly
  one of these blocks. The number of partitions of $[n]$
  into $k$ nonempty blocks is denoted by $S(n, k)$.
  The numbers $S(n, k)$ are called the \emph{Stirling numbers
  of the second kind}.
\end{definition}
\nomenclature[C]{$S(n, k)$}{denotes the Stirling number
of the second kind; i.e. the number of partitions
of $[n]$ into $k$ nonempty blocks}

It is easy to see that $S(n, 1) = 1$ and $S(n, n) = 1$.
Moreover, $S(n, k) = 0$ if $k > n$ or $k \le 0$.

Let us find the value in a more complicated setting, we
claim that $S(n, n - 2) = \binom{n}{2}$. Indeed, any
partition of $[n]$ into $n - 1$ blocks consists of
$n - 1$ singletons and one set with two elements, thus we
just need to select these two elements.

Using double counting, one may prove a recursive formula
for Stirling numbers of the second kind.
\begin{theorem}
  For any $n > k > 0$,
  \[
    S(n, k) = S(n - 1, k - 1) + k \cdot S(n - 1, k).
  \]
\end{theorem}
\begin{proof}
  Let us consider $n$, note that there are two cases either $n$ forms a
  singleton in a partition or it is not the only element in the part.

  It is easy to see that there are $S(n - 1, k - 1)$ partitions where $n$ is
  a singleton and $k \cdot S(n - 1, k)$ partitions when $n$ is not a singleton
  (we multiply by $k$ since there are $k$ possible ways to add $n$ to a
  partition of $[n - 1]$).
\end{proof}

Note that the number of surjections from $[n]$ to $[k]$ is equal to the
number of ways to put $n$ different objects into $k$ different boxes.
\begin{lemma}
  There are exactly $k! S(n, k)$ surjective functions from
  $[n]$ to $[k]$.
\end{lemma}
\begin{proof}
  Let $\mathcal{S}(n, k)$ be the set of surjections from $[n]$ to $[k]$,
  $\mathcal{P}(n, k)$ be the set of partitions with non-empty blocks, and
  $F : \mathcal{S}(n, k) \to \mathcal{P}(n, k)$ such that
  $F(f) = \set{f^{-1}(1), \dots, f^{-1}(k)}$.

  It is easy to see that $F(f) = F(g)$ iff there is $h : [k] \to [k]$ such that
  $f \circ h = g$. Hence, $F^{-1}(f) = k!$ for any $f \in \mathcal{S}(n, k)$.
  Thus $|\mathcal{S}(n, k)| = k! |\mathcal{P}(n, k)|$.
\end{proof}

Using this equality, we can prove a surprising result.
\begin{theorem}
\label{theorem:stirling-numbers-and-polynomials}
  For any real $x$ and positive integer $n$,
  \[
    x^n = \sum_{k = 0}^n S(n, k) (x)_k,
  \]
  where $(x)_k = \prod_{i = 0}^{k - 1} (x - i)$.
\end{theorem}

To prove the statement we need the following statement.
\begin{theorem}
  Let $p$ and $q$ be a real polynomials. If $p(\ell) = q(\ell)$ for all
  integers $\ell > 0$, then $p(x) = q(x)$ for all real numbers $x$.
\end{theorem}

\begin{proof}[Theorem~\ref{theorem:stirling-numbers-and-polynomials}]
  Using the previous result, it is enough to prove that
  for any integer $\ell > 0$,
  \[
    \ell^n = \sum_{k = 0}^n S(\ell, k) (\ell)_k.
  \]

  Clearly $\ell^n$ denotes the number of ways to put $n$ different
  objects into $\ell$ different boxes. Note that if we have $k$ nonempty
  boxes, then there are $\binom{n}{k}$ ways to select these boxes and
  $k! S(\ell, k)$ ways to put objects in these $k$ boxes. Thus formula in
  the left is equal to the fromula on the right.
\end{proof}

\begin{definition}
  The number of all set partitions of $[n]$ into nonempty parts
  is denoted by $B(n)$, and is called the \emph{$n$th Bell number}.
  (We deifine $B(0) = 0$).
\end{definition}
\nomenclature[C]{$B(n)$}{denotes the $n$th Bell number; i.e.
the number of partitions of $[n]$ into nonempty blocks}


It is easy to see that the following theorem holds.
\begin{theorem}
  For any $n \ge 0$,
  \[
    B(n + 1) = \sum_{i = 0}^n \binom{n}{i} B(i).
  \]
\end{theorem}
\begin{proof}
  Note that there are $B(n + 1)$ ways to split $[n]$ into
  non-empty blocks. At the same time there are $\binom{n}{n - i}$
  ways to select elements to put with $n + 1$ in the same block
  (if we know that there are $n - i$ elements with $n + 1$ in the block)
  and $B(i)$ ways to split the rest into blocks. As a result,
  there are $\sum_{i = 0}^n \binom{n}{i} B(i)$ to split $[n + 1]$
  into nonempty blocks.
\end{proof}

\section{Compoisiton}
This section answers the question in the case when the objects
are the same but boxes are different. Since all the objects
are identical, only the number of objects in each box matters.

\begin{definition}
  A sequence $(a_1, \dots, a_k)$ of nonnegative integers such
  that $a_1 + \dots + a_k = n$ is called a \emph{weak composition}
  of $n$ into $k$. If, in addition, all the numbers are positive,
  the sequence is called \emph{composition}.
\end{definition}

Using the binomial coefficients we can find the number of weak
compositions.
\begin{theorem}
  For all positive integers $n$ and $k$,
  the number of weak compositions of $n$ into $k$ is equal to
  $\binom{n + k - 1}{n}$.
\end{theorem}
\begin{proof}
  Let us consider $k$ boxes in line one after each other. Note that if we
  put balls inside of the boxes we see a line consisting of $n$ balls and
  $k - 1$ walls separating the $k$ boxes from each other. Note that simply
  knowing in which order the n identical balls and $k - 1$ separating walls
  follow each other is the same as knowing the number of balls in each box.
  So our problem is equivalent to counting the number of ways to put $k - 1$
  walls on one of $n + k - 1$ positions.
\end{proof}

As a result, we can count the number of compositions.
\begin{corollary}
  For all positive integers $n$ and $k$,
  the number of compositions of $n$ into $k$ is equal to
  $\binom{n}{k}$.
\end{corollary}

\begin{exercise}
  Let $\ell_1$, \dots, $\ell_k$ be some nonnegative numbers
  such that $\ell_1 + \dots + \ell_k$.
  Find the number of weak compositions $(a_1, \dots, a_k)$
  of $n$ into $k$ such that $a_i \ge \ell_i$.
\end{exercise}

\begin{corollary}
  The number of all compositions of $n$ is equal to $2^{n - 1}$.
\end{corollary}

\section{Integer Partitions}
Now consider the case when both objects and boxes are identical. In this case,
as in the previous we are only intrested in the numbers of objects in boxes, but
in addition, we are not intrested in the order of these numbers.

\begin{definition}
  Let $n$ and $a_1 \ge a_2 \ge \dots \ge a_k \ge 1$ be integers so that
  $a_1 + \dots a_k = n$. Then the sequence $(a_1, \dots, a_k)$ is called
  a \emph{partition}\footnote{%
    Note that we used the word partition in two different meanings: one to
    denote a partition of a set $[n]$ and another to denote the partition of
    an integer $n$. In most of the cases the meaning is clear from the context;
    however, if it is necessary to emphasize that we mean partition of a set,
    we say set-partition. Note that in some languages there are two different
    words for these two notions; e.g in French ``partition'' is used for
    set-partitions, and ``partage'' for partitions of the integer n).
  } of the integer $n$.

  The number of all the partitions is denoted by $p(n)$ and the number of
  partitions of $n$ into $k$ parts is denoted by $p_k(n)$.
\end{definition}

\nomenclature[C]{$p_k(n)$}{denotes the number of all the partitions of $n$ into
$k$ blocks}
\nomenclature[C]{$p(n)$}{denotes the number of all the partitions of $n$}


There is no good formula allowing to find the value of $p(n)$. However, we will
prove some properties of $p(n)$. The main tool to explain proofs we are going to
discuss are Young diagrams\footnote{%
  A small variation of these diagrams is called Ferrers shapes after an
  american mathematician Norman Macleod Ferrers.
}.
A Young diagram for a partition $(a_1, \dots, a_k)$ consists of $k$ columns of
squares called ``boxes'' such that in the $i$th column there are $a_i$ boxes
(an example of such a diagram is depicted on \ref{figure:young-diagram-example}).
\begin{figure}
  \centering
  \subfloat[The Young diagram for the partition $(4, 3, 1, 1)$.]{
    \qquad\qquad$\yng(4,2,2,1)$\qquad\qquad
  }
  \qquad\qquad
  \subfloat[The conjugate of the Young diagram for the partition $(4, 3, 1, 1)$.]{
    \qquad\qquad$\yng(4,3,1,1)$\qquad\qquad
  }
  \caption[][1cm]{Young diagrams.}
  \label{figure:young-diagram-example}
\end{figure}
We can reflect a Young diagram of a partition of $n$ with respect to its main
diagonal, we get another shape, representing the \emph{conjugate} partition of
$n$ (an example of such transformation is depicted on
\ref{figure:young-diagram-example}).

Using these diagrams it is easy to show the following theorems.
\begin{theorem}
  The number of partitions of $n$ into at most $k$ parts is equal to that of
  partitions of $n$ into parts not larger than $k$.
\end{theorem}
\begin{proof}
  Note that if a partition has at most $k$ parts, then the conjugate of this
  partition has all the parts of size at most $k$. As, a result, the number
  of partitions of $n$ into at most $k$ parts is equal to that of
  partitions of $n$ into parts not larger than $k$.
\end{proof}

\begin{chapterendexercises}
  \exercise
    Let $q(n)$ be the number of partitions of $n$ in which each part is at least
    two. Then $q(n) = p(n) - p(n - 1)$, for all positive integers $n \ge 2$.
  \exercise Find a formula for $S(n, 2)$.
  \exercise Show that $B(n) \le n!$.
  \exercise Let $m \ge n$ be positive integers. Show that
    \[
      S(m, n) = \sum_{i = 1}^m S(m - i, n - 1) n^{i - 1}.
    \]
  \exercise Prove that the number of partitions of $n$ into exactly $k$
    parts is equal to the number of partitions of $n$ in which the
    largest part is exactly $k$.
  \exercise Prove that the number of partitions of $n$ into at most $k$
    parts is equal to that of partitions of $n + k$ into exactly $k$ parts.
\end{chapterendexercises}
