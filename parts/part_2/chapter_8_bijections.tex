\chapter{Bijections, Surjections, and Injections}
\label{chapter:bijections-surjections-injections}
\marginurl{%
  Bijections, Surjections, and Injections:\\\noindent
  Introduction to Combinatorics \#1
}{youtu.be/fW5Zxg0TMDc}

In the previous chapters we used the property that the set is finite. However,
we have never defined formally what it means. In this chapter we define
cardinality which is a formalization of the notion size of the set and explain
how to compare sizes of two sets.

\section{Bijections}
The simplest way to explain that one set has the same number of elements as
another is to show a correspondence between elements of these sets. For example,
in order to explain that the set $\set{0, \pi, 1 / 4}$ has the same number of
elements as $\set{1, 2, 3}$ we may just say that $0$ corresponds to $1$,
$\pi$ corresponds to $2$, and $1 / 4$ corresponds to $3$. More formally such a
correspondence is defined using the following definition.
\begin{definition}
  Let $f : X \to Y$ be a function. We say that $f$ is a bijection iff the
  following properties are satisfied.
  \begin{itemize}
    \item Every element of $Y$ is an image of some element of $X$. In other
    words,
      \[
          \forall y \in Y~\exists x \in X\ f(x) = y.
      \]
    \item Images of any two elements
      of $X$ are different. In other words,
      \[
          \forall x_1, x_2 \in X\ f(x_1) \neq f(x_2).
      \]
    \end{itemize}
\end{definition}

Let us consider the following example. Let $f : \mathbb{R} \to \mathbb{R}$ be a
function such that $f(x) = x + 1$; Note that it is a bijection:
\begin{itemize}
  \item For any $y \in \mathbb{R}$, $f(y - 1) = (y - 1) + 1 = y$.
  \item If $f(x_1) = f(x_2)$, then $x_1 + 1 = x_2 + 1$ i.e. $x_1 = x_2$.
\end{itemize}


\begin{exercise}
  Show that $x^3$ is a bijection.
\end{exercise}

One may notice that if we have a bijection $f$ from $[n]$ to a set $S$ we
enumerate all the elements of $S$: $f(1)$, \dots, $f(n)$.
This observation allows us to define the cardinality of a set.
\begin{definition}
  Let $S$ be a set, we say that cardinality of $S$ is equal to $n$ (we write
  that $|S| = n$) iff there is a bijection from $[n]$ to $S$.

  We also say that a set $T$ is finite if there is an integer $n$ such that
  $|T| = n$.
\end{definition}

Note that this definition does not guarantee that cardinality is unique so
we need the following theorem.
\begin{theorem}
\label{theorem:correctness-of-cardinality}
  For any set $S$, if there are bijections $f : [n] \to S$ and $g : [m] \to S$,
  then $n = m$.
\end{theorem}
Before we prove this theorem, let us study some properties of bijections.

One of the nicest properties of bijections is that composition of two bijections
is a bijection.
\begin{theorem}
\label{theorem:bijections-composition}
  Let $X$, $Y$, and $Z$ be some sets and $f : X \to Y$ and $g : Y \to Z$ be
  bijections. Then $(g \circ f) : X \to Z$ is also a bijection.
\end{theorem}
\begin{proof}
  We need to check two properties.
  \begin{itemize}
    \item Let $x_1 \neq x_2 \in X$. Note that $f(x_1) \neq f(x_2)$ since $f$
      is a bijection. Hence, $g(f(x_1)) \neq g(f(x_2))$ since $g$ is a bijection
      as well. As a result, $(g \circ f)(x_1) \neq (g \circ f)(x_2)$.
    \item Let $z \in Z$; we need to find $x \in X$ such that
      $(g \circ f)(x) = z$. Note that since $g$ is a bijection there is
      $y \in Y$ such that $g(y) = z$. Additionally, there is $x \in X$ such
      that $f(x) = y$ since $f$ is a bijection. Thus,
      $(g \circ f)(x) = g(f(x)) = z$.
  \end{itemize}
\end{proof}

Probably the most important property of a bijection is that we may invert it.
\begin{theorem}
\label{theorem:inverse-of-bijections}
  Let $f : X \to Y$ be a function. $f$ is invertible (i.e. there is a function
  $g : Y \to X$ such that $(f \circ g)(y) = y$ and $(g \circ f)(x) = x$ for all
  $x \in X$ and $y \in Y$) iff $f$ is a bijection.
\end{theorem}
\begin{proof}
  \begin{description}
    \item[$\Rightarrow$] Let's assume that $f$ is invertible. We need to prove
      that $f$ is a bijection.
      \begin{itemize}
        \item Let's assume that $f$ does not satisfy the first property in the
          definitions of bijections i.e. there are
          $x_1, x_2 \in X$ such that $f(x_1) = f(x_2)$ but $x_1 = g(f(x_1)) =
          g(f(x_2)) = x_2$, which is a contradiction.
        \item Let $y \in Y$. Note that $f(g(y)) = y$, hence, $\Im f = Y$.
      \end{itemize}

    \item[$\Leftarrow$] Let's assume that $f$ is bijective. We need to define a
      function $g : Y \to X$ which is an inverse of $f$. Let $y \in Y$, note
      that there is a unique $x$ such that $f(x) = y$, we define $g(y) = x$.
      Note that $f(g(y)) = y$ for every $y$ by the construction of $g$.
      Additionally, $g(f(x)) = x$ since $f(g(f(x))) = f(x)$ and $f$ is a
      bijection.
  \end{description}
\end{proof}
\noindent We denote $g$ from this theorem as $f^{-1}$ and in case when $f$ is
not a bijection $f^{-1}(y)$ denotes the set $\set[f(x) = y]{x \in X}$.
\nomenclature[F]{$f^{-1}$}{denotes the inverse of the function $f$ (it's
defined only when $f$ is a bijection)}
\nomenclature[F]{$f^{-1}(y)$}{depend on the context it either denotes the set
$\set[f(x) = y]{x \in X}$ if $f$ is not a bijection and it denotes the value of
$f^{-1}$ at $y$ if $f$ is a bijection}


\begin{proof}[Proof of Theorem~\ref{theorem:correctness-of-cardinality}]
  Let us consider the inverse $g^{-1}$ of $g$ (it exists by
  Theorem~\ref{theorem:inverse-of-bijections} since $g$ is a
  bijection). Note that $h = g^{-1} \circ f$
  is a bijection from $[n]$ to $[m]$.

  We prove using induction by $n$ that for any $n, m \in \N$,
  if there is a bijection $h'$ from $[n]$ to $[m]$, then $n = m$.
  The base case is for $n = 1$; if $m \ge 2$,
  then there are $x, y \in [1]$ such that $h'(x) = 1$ and $h'(y) = 2$, but
  $x \neq y$ and we have only one element in $[1]$.

  The induction step is also simple. Assume that there is a bijection $h'$ from
  $[n + 1]$ to $[m]$. We define a function $h'' : [n] \to [m - 1]$ as follows:
  \[
    h''(i) =
    \begin{cases}
      h'(i) & \text{if } h'(i) < h'(n + 1) \\
      h'(i) - 1 & \text{otherwise}
    \end{cases}.
  \]
  We prove that $h''$ is a bijection.
  \begin{itemize}
    \item Let $i_1 \neq i_2 \in [n]$. If $h'(i_1), h'(i_2) < h'(n + 1)$ or
      $h'(i_1), h'(i_2) \ge h'(n + 1)$, then $h''(i_1) \neq h''(i_2)$ since
      $h'(i_1) \neq h'(i_2)$. Otherwise, without loss of generality we
      may assume that $h'(i_1) < h'(n + 1) < h'(i_2)$ but it implies that
      $h''(i_1) = h'(i_1) < h'(n + 1) \le h'(i_2) - 1 = h''(i_2)$.
    \item Let $j \in [m - 1]$. We need to consider two cases.
      \begin{enumerate}
        \item Let $j < h(n + 1)$. There is $i \in [n + 1]$ such that
          $h'(i) = j$ since $h'$ is a bijection (note that $i \neq n + 1$).
          Thus $h''(i) = j$.
        \item Otherwise, there is $i \in [n + 1]$ such that
          $h'(i) = j + 1$ since $h'$ is a bijection (note that $i \neq n + 1$).
          Thus $h''(i) = j$.
      \end{enumerate}
  \end{itemize}
  Since $h''$ is a bijection, the induction hypothesis implies that $n = m - 1$.
  As a result, $n + 1 = m$.
\end{proof}

Using Theorem~\ref{theorem:inverse-of-bijections} we may notice that
nonetheless that $X \times (Y \times Z)$ is not the same as
$(X \times Y) \times Z$, there is a natural correspondence between the elements
of these sets.
\begin{theorem}
  Let $X$, $Y$, $Z$ be some sets.
  There are bijections from $X \times (Y \times Z)$ and
  $(X \times Y) \times Z$ to $\set[x \in X, y \in Y, z \in Z]{(x, y, z)}$.
\end{theorem}
\begin{proof}
  Since the statement is symmetric, it is enough to prove that there is
  a bijection $f : X \times (Y \times Z) \to
  \set[x \in X, y \in Y, z \in Z]{(x, y, z)}$. Define $f$ such that
  $f(x, (y, z)) = (x, y, z)$.
  Clearly, $f^{-1}(x, y, z) = (x, (y, z))$ is the inverse of $f$, so $f$ is
  indeed a bijection.
\end{proof}
Due to this correspondence we will think about elements $(x, (y, z))$,
$((x, y), z)$, and $(x, y, z)$ as they are equal to each other.

Also, using Theorem~\ref{theorem:inverse-of-bijections} we may finally prove
that if there is a bijection from a finite set $X$ to a finite set $Y$, then
they have the same cardinality (i.e. they have the same number of elements).
\begin{theorem}
  Let $X$ and $Y$ be two finite sets such that there is a bijection $f$ from
  $X$ to $Y$. Then $|X| = |Y|$.
\end{theorem}
\begin{proof}
  Let $|X| = n$, and $g : [n] \to X$ be a bijection.
  Note that $f \circ g : [n] \to Y$ is a bijection, hence $|Y| = n$.
\end{proof}

Using this result we can make prove the following equality.
\begin{corollary}
\label{corollary:power-set-and-set-of-binary-strings}
  Let $X$ be a finite set of cardinality $n$. Then $2^X$ has the same
  cardinality as $\set{0, 1}^{|X|}$.
\end{corollary}
\begin{proof}
  To prove this statement we need to construct a bijection from $2^X$ to
  $\set{0, 1}^{|X|}$. Let $|X| = n$ and $f : X \to [n]$ be a bijection.

  First we construct a bijection $g_1 : 2^X \to 2^{[n]}$:
  \[
    g_1(Y) = \set[x \in Y]{f(x)}.
  \] It is easy to see that the function
  \[
    g^{-1}_1(Y) = \set[{x \in [n]}]{f^{-1}(x)}
  \]
  is an inverse of $g_1$, so $g_1$ is indeed a bijection.

  Now we need to construct a bijection $g_2$ from $2^{[n]}$ to $\set{0, 1}^n$:
  $g_2(Y) = (u_1, \dots, u_n)$, where $u_i = 1$ iff $i \in Y$. It is clear
  that $g^{-1}_2(u_1, \dots, u_n) = \set[u_i = 1]{i \in [n]}$ is an inverse
  of $g_2$ so $g_2$ is indeed a bijection.

  As a result, by Theorem~\ref{theorem:bijections-composition}, the function
  $(g_2 \circ g_1) : 2^X \to \set{0, 1}^{|X|}$ is a bijection.
\end{proof}

\section{Surjections and Injections}

It is possible to note that the definition of the bijection consists of two part.
Both of these parts are interesting in their own regard, so they have their own
names.
\begin{definition}
  Let $f : X \to Y$ be a function.
  \begin{itemize}
    \item We say that $f$ is a surjection iff every element of $Y$ is an image
      of some element of $X$. In other words,
      \[
          \forall y \in Y~\exists x \in X\ f(x) = y.
      \]
    \item We say that $f$ is an injection iff images of any two elements
      of $X$ are different. In other words,
      \[
          \forall x_1, x_2 \in X\ f(x_1) \neq f(x_2).
      \]
  \end{itemize}
\end{definition}

\begin{remark}
  Let $f : X \to Y$ be an injection. Then $g : X \to \Im f$ such that
  $f(x) = g(x)$ is a bijection.
\end{remark}

\begin{exercise}
  Let $\R^+ = \set[x > 0]{x \in \R}$. Is $f : \R^+ \to \R^+$ such that
  $f(x) = x + 1$ a surjection/injection?
\end{exercise}

Like in the case of the bijection we may use surjections and injections to
compare sizes of sets.
\begin{theorem}
\label{theorem:injections-surjections-inequalities}
  Let $X$ and $Y$ be finite sets.
  \begin{itemize}
    \item If there is an injection from $X$ to $Y$, then $|X| \le |Y|$.
    \item If there is a surjection from $X$ to $Y$, then $|X| \ge |Y|$.
  \end{itemize}
\end{theorem}

\section{Generalized Commutative Operations}
Using the notation of cardinality we may generalize the summation operation in
the following way: $\sum_{i \in S ~:~ P(i)} f(i)$ is equal to the sum of
$f(i)$ for all the $i \in \set[P(i)]{i \in S}$; i.e.
\[
  \sum_{i \in S ~:~ P(i)} f(i) = \sum_{j = 1}^k f(i_j),
\] where $\set[P(i)]{i \in S} = \set{i_1, \dots, i_k}$. More formally,
\[
  \sum_{i \in S ~:~ P(i)} f(i) = \sum_{j = 1}^k f(g(j)),
\]
where $k = |\set[P(i)]{i \in S}|$ and $g : \set[P(i)]{i \in S} \to [k]$ is a
bijection.
\nomenclature[C]{$\sum_{i \in S ~:~ P(i)} \alpha_i$}{denotes $\alpha_{i_1} +
\dots + \alpha_{i_k}$, where $\set[P(i)]{i \in S} = \set{i_1, \dots, i_k}$}

\begin{theorem}
\label{theorem:sum-correctness}
  The definition of $\sum_{i \in S : P(i)} f(i)$ is correct;
  i.e.
  $\sum_{i = 1}^k f(g_1(i)) = \sum_{i = 1}^k f(g_2(i))$
  for any two bijections $g_1, g_2 : \set[P(i)]{i \in S} \to [k]$.
\end{theorem}
Before we prove this statement we need to give a couple of definitions.
We say that a function $h : [n] \to [n]$ is a \emph{permutation} of $[n]$
iff $h$ is a bijection. We also say that
$i, j \in [k]$ form the inversion in $h$ iff $h(i) > h(j)$ and $i < j$.
We denote by $I(h)$ the number of inversions in $h$; i.e. $I(h) =
|\set[i, j \text{ form an inversion in } h]{(i, j)}|$.
\nomenclature[C]{$I(h)$}{denotes the number of inversions in $h$}

Important examples of permutations are transposition: for any $i, j \in [n]$,
$\tau_{i, j} : [n] \to [n]$ such that
\[
  \tau_{i, j}(x) =
  \begin{cases}
    j & \text{if } x = i \\
    i & \text{if } x = j \\
    x & \text{otherwise}
  \end{cases}.
\]
is called a transposition of $i$ and $j$.
\nomenclature[F]{$\tau_{i, j}$}{denotes the transposition of $i$ and $j$}

It is easy to see that $I(h) = 0$ iff $h(i) = i$ for any $i \in [k]$.
It is also clear that if $i, j$ form an inversion in $h$, then $I(h) > I(h')$,
where $h' = h \circ \tau_{i, j}$, i.e.
\[
  h'(x) =
  \begin{cases}
    h(j) & \text{if } x = i \\
    h(i) & \text{if } x = j \\
    h(x) & \text{otherwise}
  \end{cases}.
\]


\begin{proof}[Proof of Theorem~\ref{theorem:sum-correctness}]
  Proof of this theorem consists of two parts.
  First, we prove that
  \begin{equation}
    \label{equation:summation-with-respect-to_bijection}
    \sum_{i = 1}^k f(g(i)) = \sum_{i = 1}^k f(g(h(i)))
  \end{equation}
  for any bijections $g : \set[P(i)]{i \in S} \to [k]$ and $h : [k] \to [k]$.

  We prove Equation~\ref{equation:summation-with-respect-to_bijection} using the
  induction by $I(h)$.
  \begin{description}
    \item[(the base case)] If $I(h) = 0$, then $h$ is the identity function and
      $g(i) = g(h(i))$. Hence,
      Equation~\ref{equation:summation-with-respect-to_bijection} is true.
    \item[(the induction step)] By the induction hypothesis, for any permutation
      $h' : [k] \to [k]$,
      if $I(h') < \ell$, then
      \[
        \sum_{i = 1}^k f(g(i)) = \sum_{i = 1}^k f(g(h'(i))).
      \]
      Let us consider a permutation $h : [k] \to [k]$ such that $I(h) = \ell$.
      Let $i$ and $j$ form an inversion in $h$ (such $i$ and $j$ exist since
      $I(h) \neq 0$). Let $h' = h \circ \tau_{i, j}$.
      Note that by the induction hypothesis,
      \[
        \sum_{i = 1}^k f(g(i)) = \sum_{i = 1}^k f(g(h'(i)))
      \]
      since $I(h') < I(h) = \ell$ and it is clear that
      \[
        \sum_{i = 1}^k f(g(h'(i))) = \sum_{i = 1}^k f(g(h(i))).
      \]
      As a result,
      Equation~\ref{equation:summation-with-respect-to_bijection} is true.
  \end{description}

  Now we are ready to finish proof of the theorem.
  Consider $g_1, g_2 : \set[P(i)]{i \in S} \to [k]$ and define
  $h = g_1^{-1} \circ g_2$. Note that $h : [k] \to [k]$ is a permutation and
  $g_1(h(i)) = g_2(i)$. Thus we proved that
  \[
    \sum_{i = 1}^k f(g_1(i)) = \sum_{i = 1}^k f(g(h(i))) =
    \sum_{i = 1}^k f(g_2(i)).
  \]
\end{proof}

Similarly one may define a generalized union and intersection of sets.
Let $\Omega$ and $S$ be some sets, $X : S \to 2^\Omega$ and $P(i)$ be a
predicate. Then
\begin{gather*}
  \bigcup_{i \in S ~:~ P(i)} X(i) = \bigcup_{i = 1}^k X(g(i)) \\
  \text{and}\\
  \bigcap_{i \in S ~:~ P(i)} X(i) = \bigcap_{i = 1}^k X(g(i)),
\end{gather*}
where $k = |\set[P(i)]{i \in S}|$ and $g : \set[P(i)]{i \in S} \to [k]$ is a
bijection.
\nomenclature[S]{$\bigcup_{i \in S ~:~ P(i)} A_i$}{denotes $A_{i_1} \cup \dots
\cup A_{i_k}$, where $\set[P(i)]{i \in S} = \set{i_1, \dots, i_k}$}
\nomenclature[S]{$\bigcap_{i \in S ~:~ P(i)} A_i$}{denotes $A_{i_1} \cap \dots
\cap A_{i_k}$, where $\set[P(i)]{i \in S} = \set{i_1, \dots, i_k}$}

\begin{exercise}
  Show that the definitions of $\bigcup_{i \in S : P(i)} X(i)$ and
  $\bigcap_{i \in S : P(i)} X(i)$ are correct,
  i.e. that they do not depend on the choice of $g$.
\end{exercise}


\begin{chapterendexercises}
  \exercise Construct a bijection from $\set{0, 1, 2}^n$ to
    \[
      \set[{A, B \subseteq [n] \text{ and } A, B \text{ are disjoint}}]{(A, B)}.
    \]
  \exercise[recommended] Construct a bijection from $\set{0, 1} \times [n]$ to $[2n]$.
  \exercise Prove Theorem~\ref{theorem:injections-surjections-inequalities}.
\end{chapterendexercises}
