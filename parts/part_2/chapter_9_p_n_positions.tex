\chapter{P-positions and N-positions}
\marginpar{%
  This part is based on Part I of
  ``Game Theory''
  by Ferguson.
}
In this part we use our knowledge abot basics of mathematical reasoning
to study games similar to chekers, chess, shogi, and tic tac toe. The games we
are going to study are called combinatorial games. In these games there are two
players, each know all the information, there are no chance moves, and when the
game ends there is always a winner\footnote{%
  The last condition implies that among the beforementioned games only chekers
  are combinatorial since all of them allow draws; however, we may change the
  rules to disallow the draws and this change would make all of them
  combinatorial.
}. Such a game is determined by a set of positions, and possible moves
from each position for each player. Usually, players are taking turns until
they reach a position such that no moves are possible and one of the player is
declared a winner.

\section{Take-Away Game}

Since chess, shogi and even tic tac toe are realtively complicated,
we are going to start from much simpler example of combinatorial games.
\begin{game}[Take-Away Game]
\label{game:take-away-21-3-2-1}
  In this game there are two players.
  \begin{itemize}
    \item They have a pile of $21$ chips.
    \item They make moves in turns with player I starting,
      each move consists of moving one, two or three chips out of the pile.
    \item The player that removes the last chip wins.
  \end{itemize}
\end{game}
The quesiton we would like to answer is there a strategy for one of the players
to always win? So in the rest of this part we assume that both players are
playing optimally; i.e., if there is a winning strtegy they follow the strategy.

To analyze this game we need the following two observations:
\begin{enumerate}
  \item the game is symmetric and the only difference between the players is
    who makes the first move, and
  \item if at some point the players have $n$ chips it does not matter how they
    achieved this, it will not affect the rest of the game.
\end{enumerate}
Using this remarks and induction (this style of induction is sometimes
refered as \emph{backward induction}) we are abel to analise the game.

Let us consider some certain states of the game.
Assume that they have at most $3$ chips left, in this case the player that make
the move wins. However, if there are $4$ chips, the
player that makes the first move should always take at least $1$ chip so it
loses since after its turn there are at most $3$ chip. Similarly, if there
are $5$ chips, the first player to move wins since it can take a chip and
make the second player to start with $4$.

So twe can formulate the following conjecture.
Assume that $n$ chips left in the pile. Let $r$ be the reminder of $n$ modulo
$4$. Then if $r = 0$, the first player to move loses, otherwise, the other
player loses.

Let us prove this using induction. We already proved the base case so
we need to prove the induction step from $n$ to $n + 1$.
\begin{itemize}
  \item If $n \equiv 0 \pmod{4}$, then the first player to move can remove one
    chip and the other player will start with $n$ chips so by the induction
    hypothesis he/she loses.
  \item If $n \equiv 1 \pmod$. then the first player to move can remove two
    chips and the other player will start with $n$ chips so by the induction
    hypothesis he/she loses.
  \item If $n \equiv 2 \pmod$. then the first player to move can remove three
    chips and the other player will start with $n$ chips so by the induction
    hypothesis he/she loses.
  \item If $n \equiv 3 \pmod$. then after the current player move the other
    player will start with either $n$, or $n - 1$, or $n - 2$ chips. But all
    these numbers have non-zero reminders modulo $4$. So the other player
    can win in any case.
\end{itemize}

To study combinatorial games we need to give a formul definition of them.
\begin{definition}
  A game is combinatorial if
  \begin{itemize}
    \item there are two players,
    \item there is a set of possible positions in the game,
    \item for each position and each player, there is a fixed set of possible
      legal moves,
    \item players alternate moving,
    \item the game ends when no moves are possible for the player whose
      turn is to move.
  \end{itemize}
  There are possible winning conditions,
  \begin{description}
    \item [normal play rule:] the player that made the last move wins, and
    \item [misere play rule:] the player that made the last move loses.
  \end{description}
  If the game never ends, we declare a draw. If the game always ends, we
  say that the game satisfies \emph{the ending condition}.

  If the possible moves are the same for both players the game is
  called \emph{impartial} otherwise it is called \emph{partizan}.
\end{definition}

Note that these games do not allow random moves, hidden information,
simultaneous moves, and a draw in a finite number of steps so
pocker, battleships, rock-paper-scissors, and tick tack toe are not
combinatorial games.

Since we gave a formal definition of combinatorial games we can give a framework
that allows to alalise these games.
\begin{definition}
  We say that a position in a combinatorial game is \emph{terminal} if there
  are no legal moves.

  All terminal positions are \emph{P-positions}. Every position that allows for
  the current player to move to a P-position is an \emph{N-position}. If all
  possible moves lead to N-positions, then the position is a P-position.

  For the game using the Mis\'ere rule, the definition is the smae except the
  terminal position is an N-position.
\end{definition}


\begin{table}[h!]
  \centering
  \begin{tabular}{l l l l l l l l l}
      \toprule
      0 & 1 & 2 & 3 & 4 & 5 & 6 & 7 & 8 \\
      \midrule
      P & N & N & N & P & N & N & N & P \\
      \bottomrule
  \end{tabular}
  \caption{P-positions and N-positions for \Cref{game:take-away-21-3-2-1}}
  \label{table:take-away-21-3-2-1}
\end{table}


So in \Cref{game:take-away-21-3-2-1} the only terminal position is $0$;
hence, $0$ is a P-position. Similarly we can go to $0$ from $1$, $2$, and
$3$ so they are N-positions. Hence, $3$ is a P-position,
since all the moves from $4$ lead to N-positions.
\begin{exercise}
  Show that a position $n$ is a P-position if $4$ divides $n$, and
  it is an N-position otherwise.
\end{exercise}

In other words, in this game, P-positions coincide with the positions where the
current player loses. However, it is not a coincidence.
\begin{theorem}
  If some position in a combinatorial game is an N-position, then the player to
  move have a winning strategy if we start from this position. If the position
  is a P-position, then the other player has a winning strategy.
\end{theorem}

\subsection{Subtractraction Games}
Let us define a big class of game that generalizes the taka-away gamewe
discussed at the begining of the chapter.
\begin{game}
  Let $S \subseteq \N$ be some set. The subtraction game with the subtraction
  set $S$ is the following combinatorial game.
  Two players start with a pile of $n$ chips.
  On each move they remove $s \in S$ chips out of the pile.
\end{game}

So \Cref{game:take-away-21-3-2-1} is the subtraction game with the subtraction
set $\set{1, 2, 3}$.

Let us analise the subtraction game with the subtraction set $\set{1, 3, 4}$.
\begin{table}[h!]
  \centering
  \begin{tabular}{l l l l l l l l l}
      \toprule
      0 & 1 & 2 & 3 & 4 & 5 & 6 & 7 & 8 \\
      \midrule
      P & N & P & N & N & N & N & P & N \\
      \bottomrule
  \end{tabular}
  \caption{P-positions and N-positions for \Cref{game:take-away-21-3-2-1}}
  \label{table:subtraction-4-3-1}
\end{table}
Clearly $0$ is a P-position since it is the only terminal position in the game.
We can go to $0$ from $1$ so $1$ is an N-position. The only possible move from
$2$ is to $1$ so $2$ is a P-position. From $3$ and $4$ we can go to $0$
so they are N-positions. From $5$ and $6$ one may go to $2$ so they are a
N-positions as well. Hence, $7$ is a P-position.

Now we may notice the pattern: $n$ is a P-position iff $n \equiv 0 \pmod{7}$
or $n \equiv 2 \pmod{7}$. We prove is using induction. The base case for $n < 8$
we already proved. Let us now prove the induction step. Assume that the
statement is true for all $k < n$. Consider the following cases.
\begin{enumerate}
  \item If $n \equiv 0 \pmod{7}$, the current player can move to
    $n - 1 \equiv 5 \pmod {7}$, $n - 3 \equiv 4 \pmod {7}$, or
    $n - 3 \equiv 5 \pmod {7}$ which are all N-positions so $n$ is a P-position.
  \item If $n \equiv 1 \pmod{7}$, the current player can move to $n - 1$ which
    is a P-position so $n$ is an N-position.
  \item If $n \equiv 2 \pmod{7}$, the current player can move to
    $n - 1 \equiv 1 \pmod {7}$, $n - 3 \equiv 6 \pmod {7}$, or
    $n - 4 \equiv 5 \pmod {7}$ which are all N-positions so $n$ is a P-position.
  \item If $n \equiv 3 \pmod{7}$, the current player can move to $n - 1$ which
    is a P-position so $n$ is an N-position.
  \item If $n \equiv 4 \pmod{7}$, the current player can move to $n - 4$ which
    is a P-position so $n$ is an N-position.
  \item If $n \equiv 5 \pmod{7}$, the current player can move to $n - 3$ which
    is a P-position so $n$ is an N-position.
  \item If $n \equiv 6 \pmod{7}$, the current player can move to $n - 4$ which
    is a P-position so $n$ is an N-position.
\end{enumerate}


\begin{chapterendexercises}
  \exercise
    Two players I and II are playing the following game.
    \begin{itemize}
      \item They start with a number $0$ written on a blackboard.
      \item On each step one of the players replace a number $n$ on the
        blackboard by either $n + 1$ or by $n + 2$.
      \item Player I makes the first move and players do moves one
        after another.
      \item The player who writes $20$ wins.
    \end{itemize}
    Who has a winning strategy?
    (Note that the game is not a combinatorial game).
  \exercise
    Two players I and II are playing the following game.
    \begin{itemize}
      \item Initially, there are $20$ numbers written on a blackboard:
        $10$ numbers $1$ and $10$ numbers $2$.
      \item On each step one of the players select two numbers;
        and if they were the same, replace them by $2$;
        otherwise, replace them by $1$.
      \item Player I makes the first move and players do moves one
        after another.
    \end{itemize}
    Who is the winner? (Note that the game is not a combinatorial game).
  \exercise Consider the subtraction game where players may subtract $2$ and $3$
    chips on their turn, is $5$ an N-position?
  \exercise Consider the Mis\'ere subtraction game where players may subtract
    $1$, $2$ or $5$ chips on their turn, identify N-positions and
    P-positions.
  \exercise Consider the Mis\'ere subtraction game where players may subtract
    $1$, $5$ or $6$ chips on their turn, identify N-positions and
    P-positions.
  \exercise
    In the subtraction game where players may subtract $1$, $2$, or $5$ chips
    on their turn, identify N-positions and P-positions.
  \exercise Two players one by one put bishops on the chessboard such that none
    of the bishops attack each other. Determine the winning strategy.
  \exercise Consider the following game: two players I and II are writing an
    $11$-digit number from left to right, one digit after another. Player I
    wins  if $7$ divides the number and player II wins otherwise.
    Determine who is the winner if player I makes the first move.
\end{chapterendexercises}
