\chapter{Propositional Logic}
In the previous part we studied the most important mathematical notation and
how to prove theorems, we gave several ``formal'' definitions; however, our
definition of the proof was not quite formal. It just allowed us to distinguish
between good arguments and bad arguments. It is not enough if we wish to study
proofs which is the core of mathematical logic.

\section{Propositional Formulas}
First, we need to define what precisely we mean by ``mathematical statement''.
In order to do this we define a class of formulas called propositional formulas.
\begin{definition}
  We say that $\phi$ is a propositional formula on the variables $x_1$, \dots,
  $x_n$ if
  \begin{itemize}
    \item either $\phi$ is equal to $x_i$ for some $i \in [n]$,
    \item or $\phi$ is equal to $(\psi_1 \land \psi_2)$ or
      $(\psi_1 \lor \psi_2)$, where $\psi_1$ and $\psi_2$ are propositional
      formulas on $x_1$, \dots $x_n$,
    \item or $\phi$ is equal to $\lnot \psi$, where $\psi$ is a propositional
      formula on $x_1$, \dots $x_n$.
  \end{itemize}
\end{definition}

We can also define the value of a formula.
\begin{definition}
  Let $\phi$ be a propositional formula on the variables $x_1$, \dots, $x_n$
  and $v_1, \dots, v_n \in \set{\text{T}, \text{F}}$.

  We say that the value $\phi\big\rvert_{x_1 = v_1, \dots, x_n = v_n}$ of the
  formula $\phi$ when $v_1$ is substituted as the value of $x_1$, \dots,
  and $v_n$  is substituted as the value of $x_n$ is equal to
  \begin{itemize}
    \item $v_i$ if $\phi$ is equal to $x_i$, and
    \item $\psi_1\big\rvert_{x_1 = v_1, \dots, x_n = v_n} \land
      \psi_2\big\rvert_{x_1 = v_1, \dots, x_n = v_n}$ if
      $\phi$ is equal to $(\psi_1 \land \psi_2)$, and
    \item $\psi_1\big\rvert_{x_1 = v_1, \dots, x_n = v_n} \lor
      \psi_2\big\rvert_{x_1 = v_1, \dots, x_n = v_n}$ if
      $\phi$ is equal to $(\psi_1 \lor \psi_2)$, and
    \item $\lnot\psi\big\rvert_{x_1 = v_1, \dots, x_n = v_n}$
      if $\phi$ is equal to $\lnot \psi$.
  \end{itemize}
\end{definition}

For example, the value of a formula $(x_1 \land x_2) \lor x_3$ when T is
substituted as the value of $x_1$, T is substituted as the value of $x_2$,
and F is substituted as the value of $x_3$ is equal to
$(\text{T} \land \text{T}) \lor \text{F} = \text{T}$.

\begin{theorem}
  For any function $f : \set{\text{T}, \text{F}}^n \to
  \set{\text{T}, \text{F}}$ there is a
  formula $\phi$ on the variables $x_1$, \dots, $x_n$ such that
  $\phi\big\rvert_{x_1 = v_1, \dots, x_n = v_n} = f(v_1, \dots, v_n)$ for all
  $v_1, \dots, v_n \in \set{\text{T}, \text{F}}$
\end{theorem}

\section{Truth Tables}
Let us start the discussion of mathematical logic from an example similar to
the proof we gave in the begining of the first chapter.
Assume that we know that if $x$ is a real number such that $x < -2$ or $x > 2$,
then $x^2 > 4$. We can derive that if $\lnot (x^2 > 4)$, then $\lnot (x < -2)$
and $\lnot (x > 2)$.

In order ot emphasize the logical structure of the argument let us denote
$x > 2$ by $p$, $x < -2$ by $q$, and $x^2 > 4$ by $r$. In this case the
argument is as follows. Assume that we know that $(p \lor q) \implies r$. We
can derive that $\lnot r \implies (\lnot p \land \lnot q)$.

How can we check that this argument is correct? The simplest way is to use a
truth table to check that whether the assumption is true, the consequence is
also true.
\begin{center}
  \begin{tabular}{c | c | c | c | c}
    $p$ & $q$ & $r$ & $(p \lor q) \implies r$ &
      $\lnot r \implies (\lnot p \land \lnot q)$ \\
    \hline
    T & T & T & T & T \\
    T & T & F & F & F \\
    T & F & T & T & T \\
    T & F & F & F & F \\
    F & T & T & T & T \\
    F & T & F & F & F \\
    F & F & T & T & T \\
    F & F & F & T & T
  \end{tabular}
\end{center}
It is easy to note that the argument is indeed correct i.e.
\[
  \bigl((p \lor q) \implies r\bigr) \iff
  \bigl(\lnot r \implies (\lnot p \land \lnot q) \bigr)
\]
is always true (we say
that this statement is a tautology). Such an argument is called the
\textit{contraposition} argument.

Let us now consider another argument. If we know that Joe was a good boy and we
know that if Joe is a good boy, then Santa gives a present to Joe. We may
conclude that Santa gives a present to Joe. We can similarly to the previous
example write this argument using variables and connectives.
If we know that $p$ and $p \implies q$, we may conclude that $q$ is true.
\begin{exercise}
  Show that this argument is correct.
\end{exercise}
Such an argument is called \textit{modus ponens}.

\section{Derivation Rules}
The problem of this method is that we need to consider \textbf{all} possible
values of the variables. Let us now consider a more complicated example.
Imagine that we know that $\lnot q$, $p \implies q$. Using the contraposition
argument and modus ponens we may derive $\lnot p$. Indeed, by contraposition
we may conclude that $\lnot q \implies \lnot p$ and modus ponens impies that
$\lnot p$ is true since $\lnot q$ is true.

In other words, we can combine several tautologies to prove another tautology.
Apparently it is enough to fix some small number of tautlogies to derive all
other tautologies, we call these tautlogies ``rules''.
There are several ways to write such proofs, we are going to use Fitch
notation. In this notation any proof is writen in several rows, each row in a
Fitch-style proof is either:
\begin{itemize}
  \item an assumption or subproof assumption.
  \item a sentence justified by the citation of (1) a rule of inference and (2)
    the prior line or lines of the proof that license that rule.
\end{itemize}
Using this scheme we may write the argument we jsut mentioned as follows.
\[
  \begin{nd}
    \have {1} {\lnot q}
    \have {2} {p \implies q}
    \have {3} {\lnot q \implies \lnot p} \by{contraposition}{2}
    \have {4} {\lnot p} \by{modus ponens}{1, 3}
  \end{nd}
\]
In the rest of the section we are going to list all the rules we use.

\paragraph{Conjunctions.}
In order to introduce a conjunction we can use the following rule.
\[
  \begin{nd}
    \have [m] {1} {A}
    \have [n] {3} {B}
    \have [~] {5} {A \land B} \ai{1, 3}
  \end{nd}
\]
This rule corresponds to the tautology $(A \land B) \implies (A \land B)$.

In order to eliminate conjunctions we can use the following two rules.
\begin{center}
  \begin{tabular}{c c}
    $\begin{nd}
      \have [m] {1} {A \land B}
      \have [~] {3} {A} \ae{1}
    \end{nd}$
    &
    $\begin{nd}
      \have [m] {1} {A \land B}
      \have [~] {3} {B} \ae{1}
    \end{nd}$
  \end{tabular}
\end{center}
These rules correspond to the tautologies $(A \land B) \implies A$ and
$(A \land B) \implies B$.

\paragraph{Disjuctions.}
In order to introduce a disjunction we can use the following two rules.
\begin{center}
  \begin{tabular}{c c}
    $\begin{nd}
      \have [m] {1} {A}
      \have [~] {3} {A \lor B} \oi{1}
    \end{nd}$
    &
    $\begin{nd}
      \have [m] {1} {A}
      \have [~] {3} {B \lor A} \oi{1}
    \end{nd}$
  \end{tabular}
\end{center}
These rules correspond to the tautologies $A \implies (A \lor B)$ and
$A \implies (B \lor A)$.

In order to eliminate a disjunction we can use the following rule.
\[
  \begin{nd}
    \have [m] {1} {A \lor B}
    \open
      \hypo [i] {3} {A}
      \have[j] {5} {C}
    \close
    \open
      \hypo [k] {6} {B}
      \have[l] {8} {C}
    \close
    \have[~] {9} {C} \oe{1, 3-5, 6-8}
  \end{nd}
\]
This rule corresponds to the tautology
$\bigl( (A \lor B) \land (A \implies C) \land (B \implies C) \bigr)
\implies C$.

\paragraph{Implications.}
In order to introduce an implication we can use the following two rules.
\[
  \begin{nd}
    \open
      \hypo [i] {3} {A}
      \have[j] {5} {B}
    \close
    \have[~] {9} {A \implies B} \ii{3-5}
  \end{nd}
\]
This rule corresponds to the tautology
$(A \implies B) \implies (A \implies B)$.

In order to eliminate an implication we can use the following rule.
\[
  \begin{nd}
    \have [m] {1} {A \implies B}
    \have [n] {2} {A}
    \have[~] {9} {B} \ie{1, 2}
  \end{nd}
\]
This rule corresponds to the tautology
$\bigl( (A \implies B) \land A \bigr)
\implies B$.

\paragraph{Negations.}
In order to introduce a negation we can use the following two rules ($\perp$ is
a special symbol representing a false statement).
\[
  \begin{nd}
    \open
      \hypo [i] {3} {A}
      \have[j] {5} {\perp}
    \close
    \have[~] {9} {\lnot A} \ni{3-5}
  \end{nd}
\]
This rule corresponds to the tautology
$\bigl( A \implies \perp \bigr)
\implies \lnot A$.

In order to eliminate a negation we can use the following rule.
\[
  \begin{nd}
    \have [m] {1} {A}
    \have [n] {2} {\lnot A}
    \have[~] {9} {\perp} \ne{1, 2}
  \end{nd}
\]
This rule corresponds to the tautology
$\bigl( A \land \lnot A \bigr)
\implies \perp$.

\paragraph{Truths and falsities.}
Additionally, we have the following two rules.
\begin{center}
  \begin{tabular}{c c}
    $\begin{nd}
      \have [m] {1} {\perp}
      \have [~] {3} {A} \be{1}
    \end{nd}$
    &
    $\begin{nd}
      \open
        \hypo [i] {3} {\lnot A}
        \have[j] {5} {\perp}
      \close
      \have[~] {9} {A} \by{IP}{3, 5}
    \end{nd}$
  \end{tabular}
\end{center}
% Give a link to https://proofs.openlogicproject.org/

\begin{exercise}
  Check that all the tautologies we mentioned are indeed tautlogies.
\end{exercise}

\section{Examples of Derivations}
In this section we give several derivations using the rules we just introduced.

First, we prove that if we know that $A \implies \lnot A$ we can derive that
$\lnot A$.
\[
  \begin{nd}
    \hypo {1} {A \implies \lnot A}
    \open
      \hypo {2} {A}
      \have {3} {\lnot A} \ie{1, 2}
      \have {4} {\perp} \ne{2, 3}
    \close
    \have {5} {\lnot A} \ni{2-4}
  \end{nd}
\]

Another statement we are going to prove is that if
$A \implies (A \land \lnot A)$ is true, then $\lnot A$ is also true.
\[
  \begin{nd}
    \hypo {1} {A \implies (A \land \lnot A)}
    \open
      \hypo {2} {A}
      \have {3} {A \land \lnot A} \ie{1, 2}
      \have {4} {\lnot A} \ae{3}
      \have {5} {\perp} \ne{2, 4}
    \close
    \have {6} {\lnot A} \ni{2-5}
  \end{nd}
\]

A bit more complicated is the proof of the law of excluded middle:
$A \lor \lnot A$.
\[
  \begin{nd}
    \hypo {1} {}
    \open
      \hypo {2} {\lnot (A \lor \lnot A)}
      \open
        \hypo {3} {A}
        \have {4} {A \lor \lnot A} \oi{3}
        \have {5} {\perp} \ne{2, 4}
      \close
      \have {6} {\lnot A} \ni{3-5}
      \have {7} {A \lor \lnot A} \oi{6}
      \have {8} {\perp} \ne{2, 8}
    \close
    \have {9} {A \lor \lnot A} \by{IP}{2-8}
  \end{nd}
\]

\section*{End of The Chapter Exercises}
\begin{exercises}
  \exerciseitem Derive $(A \land B) \implies C$ from
    $A \implies (B \implies C)$.
  \exerciseitem Derive $A \lor C$ from
    $(A \land B) \lor C$.
  \exerciseitem Derive $B \lor C$ from
    $A \implies B$ and $\lnot A \implies C$.
\end{exercises}
