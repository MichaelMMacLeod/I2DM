\chapter{Predicate Logic}
In the previous chapter we defined natural deductions for propositional logic.
But in real mathematics there are many formulas that are not propositional. For
example we may wish to prove that if a relation $R$ on $M$ is transitive, then
\[
  (R(w, x) \land R(x, y) \land R(y, z)) \implies R(w, z)
\]
is true for any $w, x, y, z \in M$. In this chapter we define a logical
system that allows us to formally prove such statements.

\section{Predicate Formulas}
Let us formalize the previous statement:
\begin{multline*}
  (\forall x, y, z \in M ~ (R(x, y) \land R(y, z)) \implies R(x, z)) \implies \\
  (\forall w, x, y, z \in M ~ (R(w, x) \land R(x, y) \land R(y, z)) \implies
  R(x, z))
\end{multline*}
Note that there are several things we need to explain if we wish to define
formally formulas like this:
\begin{itemize}
  \item we need to explain what kind of sets we can use (in this case we need
    to define $M$),
  \item we need to explain what kind of relations we can use (in this case we
    need to define $R$),
\end{itemize}

Another example of a formula we may wish to prove is saying that if
$f : M \to M$ is an inverse of itself (i.e. $f(f(x)) = f(x)$ for any $x \in M$),
then $f(f(f(x))) = f(x)$ for any $x \in M$; more formaly, we may wish to prove a
statement
\[
  (\forall x \in M ~ f(f(x)) = x) \implies
    (\forall x \in M ~ f(f(f(x))) = f(x)).
\]
In order to expalin what we mean by such formulas
\begin{itemize}
  \item we need to explain what kind functions we can use (in this case we need
    to define $f$).
\end{itemize}

\paragraph{Signature.}
In predicate logic, formula uses just symbols for all these objects. We specify
these symbols only when we wish to compute actual truth value of the formula.
We also assume that all the quantifiers are over the same set so we do not need
a symbol for the set $M$.

Signature is the way to define the list of all these symbols, it consists of
three objects:
\begin{itemize}
  \item the set of symbols for relations,
  \item the set of symbols for functions,
  \item arities of these functions and relations (i.e. how many arguments they
    may take).
\end{itemize}
An example of a signature is a triple $(\set{\text{``R''}}, \set{\text{``f''}}, \mathrm{ar})$, where
\[
  \mathrm{ar}(s) = \begin{cases}
    2 & \text{if } s =  \text{``R''} \\
    1 & \text{if } s =  \text{``f''}
  \end{cases}.
\]
This signature is enough to define the formulas we discussed. Now we are ready
to defin the predicate formulas.

\begin{definition}
  Let $\mathcal{S} = (S_\mathrm{rel}, S_\mathrm{fun}, a)$ be a signature.

  We say that $t$ is a term in the signature $\mathcal{S}$ if
  \begin{itemize}
    \item either $t$ is equal to a variable $x_i$
    \item or $t$ is equal to $f(t_1, \dots, t_\ell)$, where
      $f \in S_\mathrm{fun}$, $\ell = a(f)$, and $t_1$, \dots, $t_\ell$
      are terms in the signature $\mathcal{S}$.
  \end{itemize}

  We say that $\phi$ is a predicate formula in the signature $\mathcal{S}$ if
  \begin{itemize}
    \item either $\phi$ is equal to $R(t_1, \dots, t_\ell)$, where
      $R \in S_\mathrm{rel}$, $\ell = a(R)$, and $t_1$, \dots, $t_\ell$
      are terms in the signature $\mathcal{S}$.
    \item or $\phi$ is equal to $(\psi_1 \land \psi_2)$ or
      $(\psi_1 \lor \psi_2)$, where $\psi_1$ and $\psi_2$ are predicate
      formulas in the signature $\mathcal{S}$,
    \item or $\phi$ is equal to $\lnot \psi$, where $\psi$ is a predicate
      formula in the signature $\mathcal{S}$,
    \item or $\phi$ is equal to $\exists x~\psi$ or $\forall x~\psi$ where
      $\psi$ is a predicate formula in the signature $\mathcal{S}$.
  \end{itemize}

\end{definition}



\section{Natural Deduction}
\section{Soundness and Completeness}
