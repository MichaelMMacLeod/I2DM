\chapter{Generating Function}
\label{chapter:generating-functions}
In this chapter we discuss the basics of one of the most general methods we have
in combinatorics, the method is called ``generating functions''. The core idea
of this method is to use knowledge we have about mathematical analysis in
combinatorics.

\section{Easy Two Term Recurrences}
Let us start from the following problem.
Sasha took an insane credit in a bank: he took $100$\$ at the begining and
his debt is growing twofold every year. At the begining of each year John is
paing $100$\$ to the bank. How big will be his debt in 5 years?

It is easy to see that the answer for this and simialr questions can be answered
using a recurrent formula. Indeed, if $a_i$ denotes his debt on $i$th year,
then $a_0 = 100$, and $a_{n + 1} = 2 a_n - 100$. Using this, one may compute
all the values of $a_i$. However, the question became tricky if we want to
find an explicit formula for $a_i$.

To solve this kind of questions we can use beforementioned generating functions.
\begin{definition}
  Let $\set{c_n}_{n \ge 0}$ be a sequence of real numbers. Then
  the generating function for this sequence is the power series
  $F(x) = \sum_{n \ge 0} c_n x^n$.
\end{definition}
Note that these power series may not converge for $x \neq 0$.
In this chapter, we will not discuss this problem and always pretend that
they are converging, for a formal explanation of how to deal with this issue see
Appendix~\ref{chapter:formal-power-series}.

Let us use the definition of $a_i$ to find the generating function $G(x)$
for this sequence. Note that
$a_{n + 1} x^{n + 1} = 2 a_n x^{n + 1} - 100 x^{n + 1}$. Thus
\[
  \sum_{n \ge 0} a_{n + 1} x^{n + 1} =
    \sum_{n \ge 0} 2 a_n x^{n + 1} - 100 \sum_{n \ge 0} x^{n + 1}.
\]
The left-hand side is equal to $G(x) - a_0$ and the right-hand side is
equal to $2xG(x) - \frac{100 x}{1 - x}$. So we can derive the equality
\[
  G(x) - 100 = 2xG(x) - \frac{100 x}{1 - x}.
\]
Using this equality we can find explicitly a formula for $G(x)$,
\[
  G(x) = \frac{100}{1 - 2x} - \frac{100x}{(1 - x)(1 - 2x)}.
\]
Let us simplify the formula a bit.
\[
  G(x) = \frac{100}{1 - 2x} + \frac{100}{1 - x} - \frac{100}{1 - 2x} =
  \frac{100}{1 - x}.
\]
Thus $G(x) = \sum_{n \ge 0} 100 x^n$. As a result, $a_n = 100$.

\begin{exercise}
  Find a formula for $a_n$ in the case when $a_0 = 200$.
\end{exercise}

Let us consider another, more complicated, example. Consider a sequence
$\set{a_n}_{n \ge 0}$ such that $a_{n + 1} = 2a_n + n$ for $n \ge 0$ and
$a_0 = 1$. As in the previous case let us write an equation for the
generating function $G(x)$.
\[
  G(x) - a_0 = 2xG(x) + \sum_{n \ge 0} n x^{n + 1}.
\]
First, we find a formula for $\sum_{n \ge 0} n x^n$,
\[
  \sum_{n \ge 0} n x^{n + 1} = \sum_{n \ge 0} x^2 \cdot \frac{d x^n}{dx} =
  x^2 \cdot \frac{d \sum_{n \ge 0} x^n}{dx} = x^2 \left(\frac{1}{1 - x}\right)'
  = \frac{x^2}{(1 - x)^2}.
\]
Therefore,
\[
  G(x) = \frac{1 - 2x + 2x^2}{(1 - x)^2 (1 - 2x)}.
\]
So we need to find a more appropriate formula for $G(x)$.
Let us try to find a formula in the form
\[
  \frac{1 - 2x + 2x^2}{(1 - x)^2 (1 - 2x)} =
    \frac{A}{(1 - x)^2} + \frac{B}{1 - x} + \frac{C}{1 - 2x}.
\]
To find $A$, $B$, and $C$ we mutliply both sides by $(1 - x)^2$ and set $x = 1$.
We get that $A = -1$. We can also multiply by $1 - 2x$ and substitute
$x = 1 / 2$ and derive that $C = 2$. Now we need to find $C$, we substitute $0$
to the equation and get $B = 0$. As a result,
$G(x) = \frac{-1}{(1 - x)^2} + \frac{2}{1 - 2x}$. Using simple equalities from
calculus we can derive $G(x) = \sum_{n \ge 0} -(n + 1) + 2^{n + 1} x^n$.
So $a_n = -(n + 1) 2^{n + 1}$.

\section{Recurrences With Two Variables}
Two illustrate how to deal with recurrent relation in cases when we have
more than one variable, we prove a verstion of the binomial theorem and
derive a formula for binomial coefficients. In order to do it,
we consider the reccurent relation
\[
  \binom{n + 1}{k} = \binom{n}{k} + \binom{n}{k - 1}.
\]
Let us denote $\sum_{k \ge 0} \binom{n}{k} x^k$ by $B_n(x)$.
It is clear that
\[
  B_{n + 1}(x) - 1 = (B_n(x) - 1) + x B_n(x).
\]
Therefore, $B_{n + 1}(x) = (1 + x) B_n(x)$. As a result, $B_n(x) = (1 + x)^n$;
i.e. $\sum_{k \ge 0} \binom{n}{k} x^k = (1 + x)^n$. To find a
formula for binomial coefficients, we just need to use Taylor's formula,
$\binom{n}{k} = \frac{d^k}{dx^k} B_n(x)\big\rvert_{x = 0}  / k!$. So
$\binom{n}{k} = n (n - 1) \dots (n - k + 1) / k!$.

\section{Products of Generating Functions}
Let us consider a new problem, how many ways to design a class consisting of
$n$ lectures with theoretical part and laboratory part (the first $k$ days of
the quarter form the theoretical part, note that $k$ is not fixed) such that
there are two midterms during the theoretical part and one exam during the
laboratory part.

Let $a_n$ be the answer. It is easy to see that
\[
  a_n = \sum_{k = 1}^{n - 2} k \binom{n - k}{2}.
\]
However, this formula does not suggest an explicit formula.
Let us write an equation for the generating function for $a_n$,
\[
  G(x) = \sum_{n \ge 0} \sum_{k = 1}^{n - 2} k \binom{n - k}{2} x^n.
\]
It is easy to see that this formula implies that
\[
  G(x) = \left( \sum_{k \ge 0} k x^k \right)
    \left( \sum_{k' \ge 0} \binom{k'}{2} x^{k'} \right).
\]
Thus
\[
  G(x) = \frac{x}{(1 - x)^2} \cdot \frac{x^2}{(1 - x)^3} =
  \frac{x^4}{(1 - x)^5} = x^3\sum_{n \ge 0} \binom{n + 4}{4} x^n.
\]
As a result, $a_n = \binom{n + 1}{4}$.

Using this example, we can formulate a general rule.
\begin{theorem}
  Let $a_n$ be the number of ways to build a certain structure on an $n$-element
  set, and let $b_n$ be the number of way to build another structure on an
  $n$-element set. Let $c_n$ be the number of ways to separate $[n]$ into two
  parts consisting of numbers $\set{1, \dots, k}$ and $\set{k + 1, \dots, n}$
  ($k \ge 0$), and then to build a structure of the first type on the first set,
  and a structure of the second type on the second set.

  Then $H(x) = F(x)G(x)$, where $F(x)$, $G(x)$, and $H(x)$ are generating
  functions for $\set{a_n}_{n \ge 0}$, $\set{b_n}_{n \ge 0}$, and
  $\set{c_n}_{n \ge 0}$, respectively.
\end{theorem}

To illustrate this theorem, let us solve another problem. A company
``bolshoy brat'' needs to finish two projects. To do this, a manager of the
company splits all the emploes into two projects and in each project she selects
product team and marketing team. How many ways to do this.
Let $c_n$ be the number of ways the manager can complete this task. Again, let
us split the problem into two parts. Let $A(x)$ be the generating function for
the number of ways to split people in the first project into marketing and
product teams. It is clear that $A(x) = \sum_{k \ge 0} 2^k x^k = 1 / (1 - 2x)$
since any $k$ element set has $2^k$ subsets. It is easy to see that the second
project has the same generating function. Thus the generating function for
$\set{c_n}_{n \ge 0}$, $C(x) = A(x) A(x) = 1 / (1 - 2x)^2$.
As a result,
\[
  C(x) = \frac{1}{2} \sum_{n \ge 1} n 2^n x^{n - 1} =
  \frac{1}{2} \sum_{n \ge 0} (n + 1) 2^{n + 1} x^n
\]
and $c_n = (n + 1) 2^{n + 1}$.

\begin{exercise}
  Find the number of ways to split an $n$-day semester into three parts, choose
  any number of holidays in the first part, an odd number of holidays in the
  second part, and an even number of holidays in the third part.
\end{exercise}

\section{Compositions of Generating Functions}
As ususal, we start the section from a problem.
All $n$ soldiers of a military squadron stand in a line. The officer in charge
splits the line at several places, forming (non-empty) squads. Then she names
one person in each unit to be the commander of that unit. Let $c_n$ be the
number of ways she can do this. Find an explicit formula for $c_n$.

If the officer splits the soldiers into $k$ squads, then there
are
\[
  \sum_{n_1, \dots, n_k : n = n_1 + \dots + n_k} n_1 \cdot n_2 \dots n_2
\]
ways to do this. Hence, the generating function for spliting into squads and
selecting commanders in all $k$ squads is equal to $A^k(x)$, where
$A(x) = \sum_{n \ge 0} n x^n = \frac{x}{(1 - x)^2}$. Therefore,
the generating function $C(x)$ for $\set{c_n}_{n \ge 0}$ is equal to
$\sum_{k \ge 1} A^k(x)$. As a result,
\[
  C(x) = \frac{1}{1 - A(x)} = 1 + \frac{x}{1 - 3x + x^2}.
\]
It is possible to note that the roots $\alpha$ and $\beta$ of $x^2 - 3x + 1$ are
equal to $(3 \pm \sqrt{5}) / 2$, respectively. We want to find $A$ and $B$ such
that
\[
  \frac{1}{1 - 3x + x^2} = \frac{A}{x - \alpha} - \frac{B}{x - \beta}.
\]
Thus $1 = (A - B) x - A\beta + B\alpha$. Therefore, we have
$A = B$ and $A(\alpha - \beta) = A\sqrt{5} = 1$; i.e.
$A = B = \frac{1}{\sqrt{5}}$.
By some simple calculations we may comclude that
\[
  \frac{1}{1 - 3x + x^2} =
  \frac{1}{\sqrt{5}}(\frac{\alpha}{1 - \alpha x} - \frac{\beta}{1 - \beta x}).
\]
Therefore
$C(x) = 1 + \frac{1}{\sqrt{5}}
  \sum_{n \ge 0} (\alpha^{n + 1} - \beta^{n + 1}) x^{n + 1}$.
Hence, $c_0 = 1$ and $c_n = \frac{1}{\sqrt{5}}(\alpha^n - \beta^n)$ for $n > 0$.

The following theorem generalises this obeservation.
\begin{theorem}
  Let $a_n$ be the number of ways to build a certain structure on an $n$-element
  set, and let us assume that $a_0 = 0$. Let $c_n$ be the number of ways to
  split the set $[n]$ into an unspecified number of disjoint non-empty
  intervals, then build a structure of the given type on each of these
  intervals. Set $h_0 = 1$. Denote $F(x) = \sum_{n \ge 0} a_n x^n$ and
  $G(x) = \sum_{n \ge 0} c_n x^n$. Then $G(x) = \frac{1}{1 - A(x)}$.
\end{theorem}

\begin{chapterendexercises}
  \exercise[recommended] Find the generating functions of each of the following sequences
    (in the simplest form):
    \begin{enumerate}[nolistsep]
      \begin{multicols}{2}
        \item $a_n = n$;
        \item $a_n = \alpha n + \beta$;
        \item $a_n = n^2$;
        \item $a_n = \alpha n^2 + \beta n + \gamma$;
        \item $a_n = 3^n$.
      \end{multicols}
    \end{enumerate}
  \exercise[recommended] Let $F(x)$ be a generating function for the sequence
    $\set{a_n}_{n \ge 0}$. Write, in terms of $F(x)$, the generating functions
    of the following sequences:
    \begin{enumerate}[nolistsep]
      \begin{multicols}{2}
        \item $\set{a_n + \alpha}_{n \ge 0}$;
        \item $\set{\alpha a_n + \beta}_{n \ge 0}$;
        \item $\set{n a_n}_{n \ge 0}$;
        \item $0$, $a_1$, \dots, $a_n$, \dots;
        \item $a_1$, \dots, $a_n$, \dots;
        \item $\set{a_{n + m}}_{n \ge 0}$ ($m$ is a constant).
      \end{multicols}
    \end{enumerate}
  \exercise Let $f(n)$ be the number of subsets of $[n]$ that contain no two
    consecutive elements, for integer $n$. Find the recurrence that is satisfied
    by these numbers, and then find an explicit formula for these numbers.
  \exercise Find an explicit formula for $a_n$ if $a_0 = 0$ and for any
    $n \ge 0$, $a_{n + 1} = a_n + 2^n$.
  \exercise[recommended] Let $a_n$ be the number of ways to pay $n$ dollars using ten-dollar
    bills, five-dollar bills, and one-dollar bills only. Find the generating
    function for $a_n$.
  \exercise[recommended] Let $x_1$ and $x_2$ be two different solutions of the equation
    $1 - bx - cx^2 = 0$. Show that a sequence $\set{f_n}_{n \ge}$ satisfies the
    recurrent relation $f_{n + 2} = b f_{n + 1} + c f_n$ iff
    $t_n = \alpha x_1^{-n} + \beta x_2^{-n}$ for some $\alpha, \beta \in \R$.
  \exercise[recommended] Let $\set{a_n}_{n \ge 0}$, $\set{b_n}_{n \ge 0}$ be two sequences
    such that $b_n = \sum_{k = 0}^n a_n$ and $F(x)$ be the generating function
    for $\set{a_n}_{n \ge 0}$. Find the generating function for
    $\set{b_n}_{n \ge 0}$ in terms of $F(x)$.
  \exercise Let $\set{a_n}_{n \ge 0}$, $\set{b_n}_{n \ge 0}$ be two sequences
    such that $b_n = a_{2n}$ and $F(x)$ be the generating function
    for $\set{a_n}_{n \ge 0}$. Find the generating function for
    $\set{b_n}_{n \ge 0}$ in terms of $F(x)$.
\end{chapterendexercises}
