\chapter*{Introduction}
This part, as it follows fromt the title, is devoted to mathematical logic,
a mathematical approach to a branch of philosophy called logic. Logic studies
reasoning and mathematical logic studies mathematical reasoning. As we have
mentioned in \Cref{chapter:proofs} proofs in mathematics consists of
\emph{sentences} of a certain structure that are connected by implications.
In addition,  as we discussed in \Cref{chapter:predicates}, we can build larger
sentences from smaller ones using connectives.

Note that in real life the sentences are written using common English wich is
amgious and therefore hard for analisis.
So to create a formal description of mathematics we need to create an
artificial formal language for mathematics.

First (\Cref{chapter:propositional-logic}) we will define a language for
propositional (sentential) logic; i.e. the logic which deals only with
propositions. Later (\Cref{chapter:predicate-logic}) we extend it to a logic
which also takes properties of individuals into account.
