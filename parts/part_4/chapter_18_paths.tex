\chapter{Paths in Graphs}
\section{Connectivity}
Imagine you are developing a game, where the map is generated automatocally.
In this gate there are several areas connected by portals. So you need to check
that all the areas in your map are reachable from one another.

First we need to somehow understand what we mean by ``reachable'', we say that
an area $A$ is reachable from an area $B$ if there is a path from $A$ to
$B$. To formalize this notion using graphs we need to intrduce a graph
corresponding to the map, consider a graph $G = (V, E)$ such that vertices of
the graph are areas in your map and $(A, B) \in E$ iff the areas $A$ and $B$
are connected by a portal. So a path from $A$ to $B$ is a sequence of areas
$A = C_1$, \dots, $C_\ell = B$ such that $C_i$ and $C_{i + 1}$ are connected by
a portal (i.e. $(C_i, C_{i + 1}) \in E$).
\begin{definition}
  Let $G = (V, E)$ be a graph. We say that a path from $u$ to $v$ is
  a sequence $w_1, \dots, w_\ell \in V$ such that
  \begin{itemize}
    \item $w_1 = u$, $w_\ell = v$, and
    \item $(w_i, w_{i + 1}) \in E$ for $i \in [\ell - 1]$.
  \end{itemize}

  We say that $u, v \in V$ are connected iff there is a path from $u$ to $v$.
  So the graph is connected iff any $u, v \in V$ are connected.
\end{definition}
So, using this notatation, we need to check whether the graph corresponding to
the map is connected.

There are numerous ways to do it, we consider a simple algorithm just to see how
it works.
\begin{algorithm}
  \begin{algorithmic}[1]
    \Function{Connected}{$n$, $E$}
      \State\Return{YES}
    \EndFunction
  \end{algorithmic}
  \caption{An algorithm checking whether the graph on $[n]$ with the set of
  edges $E$ is connected.}
  \label{algorithm:connectivity}
\end{algorithm}

\section{Eulerian Paths}
\section{Hameltonian Paths}
