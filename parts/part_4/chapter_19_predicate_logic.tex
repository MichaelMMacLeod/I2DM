\chapter{Predicate Logic}
\label{chapter:predicate-logic}
In the previous chapter we defined natural deductions for propositional logic.
But in real mathematics there are many formulas that are not propositional. For
example we may wish to prove that if a relation $R$ on $M$ is transitive, then
\[
  (R(w, x) \land R(x, y) \land R(y, z)) \implies R(w, z)
\]
is true for any $w, x, y, z \in M$. In this chapter we define a logical
system that allows us to formally prove such statements.

\section{Predicate Formulas}
\marginurl{%
  Predicate Formulas:\\\noindent
  Introduction to Mathematical Logic \#5
}{youtu.be/yb9NvmXyFfg}
Let us write the previous statement in a formula-like form:
\begin{multline*}
    \overbrace{
        (\forall x, y, z \in M ~ (R(x, y) \land R(y, z)) \implies R(x, z))
    }^{R \text{ is transitive}} \implies \\
    \underbrace{(
        \forall w, x, y, z \in M ~ (R(w, x) \land R(x, y) \land R(y, z))
        \implies
        R(x, z)
    )}_\text{the desired conclusion}.
\end{multline*}
Note that there are several things we need to explain if we wish to define
formally formulas like this:
\begin{itemize}
    \item we need to explain what kind of sets we can use (in this case we need
        to define $M$),
    \item we need to explain what kind of relations we can use (in this case we
        need to define $R$),
\end{itemize}

Another example of a statement we may wish to prove is saying that if
$f : M \to M$ is an inverse of itself (i.e. $f(f(x)) = f(x)$ for any $x \in M$),
then $f(f(f(x))) = f(x)$ for any $x \in M$; more formally, we may wish to prove a
statement
\[
    \underbrace{
        (\forall x \in M ~ f(f(x)) = x)}_{f \text{is an inverse of itself}}
    \implies
    \underbrace{
        (\forall x \in M ~ f(f(f(x))) = f(x))
    }_\text{the desired conclusion}.
\]
In order to explain what we mean by such formulas
\begin{itemize}
  \item we need to explain what kind functions we can use (in this case we need
    to define $f$).
\end{itemize}

\paragraph{Signature.}
In predicate logic, formula uses just symbols for all these objects. We specify
these symbols only when we wish to compute actual truth value of the formula.
We also assume that all the quantifiers are over the same set so we do not need
a symbol for the set $M$.

Signature is the way to define the list of all these symbols, it consists of
three objects:
\begin{itemize}
  \item the set (possibly empty) of symbols for relations,
  \item the set (possibly empty) of symbols for functions,
  \item arities of these functions and relations (i.e. how many arguments they
    may take).
\end{itemize}
An example of a signature is a triple $(\set{\text{``R''}}, \set{\text{``f''}},
\mathrm{ar})$, where
\[
  \mathrm{ar}(s) = \begin{cases}
    2 & \text{if } s =  \text{``R''} \\
    1 & \text{if } s =  \text{``f''}
  \end{cases}.
\]
This signature is enough to define the formulas we discussed. Now we are ready
to define the predicate formulas.

\begin{definition}
  Let $\mathcal{S} = (S_\mathrm{rel}, S_\mathrm{fun}, a)$ be a signature.

  We say that $t$ is a \emph{term} in the signature $\mathcal{S}$ over the variables
  $x_1$, \dots, $x_n$ if
  \begin{itemize}
    \item either $t$ is equal to a variable $x_i$
    \item or $t$ is equal to $f(t_1, \dots, t_\ell)$, where
      $f \in S_\mathrm{fun}$, $\ell = a(f)$, and $t_1$, \dots, $t_\ell$
      are terms in the signature $\mathcal{S}$.
  \end{itemize}

  We say that $\phi$ is a \emph{predicate formula} in the signature
  $\mathcal{S}$ over the variables $x_1$, \dots, $x_n$  if
  \begin{itemize}
    \item either $\phi$ is equal to $R(t_1, \dots, t_\ell)$, where
      $R \in S_\mathrm{rel}$, $\ell = a(R)$, and $t_1$, \dots, $t_\ell$
      are terms in the signature $\mathcal{S}$.
    \item or $\phi$ is equal to $(\psi_1 \land \psi_2)$, or
      $(\psi_1 \lor \psi_2)$, or $(\psi_1 \implies \psi_2)$,
      where $\psi_1$ and $\psi_2$ are predicate formulas in the signature
      $\mathcal{S}$,
    \item or $\phi$ is equal to $\lnot \psi$, where $\psi$ is a predicate
      formula in the signature $\mathcal{S}$,
    \item or $\phi$ is equal to $\exists x_i~\psi$ or $\forall x_i~\psi$ where
      $\psi$ is a predicate formula in the signature $\mathcal{S}$.
  \end{itemize}
\end{definition}

In order to compute the truth value of a predicate formula, we need to specify
the values of all the free variables and all the symbols from the signature.
The specification of the symbols from the signature is called structure; i.e.
a structure for a signature $\mathcal{S} = (S_\mathrm{rel}, S_\mathrm{fun}, a)$
is a triple $(M, F_\mathrm{rel}, F_\mathrm{fun})$ such that
\begin{itemize}
  \item $F_\mathrm{rel} : S_\mathrm{rel} \to \bigcup_{i = 0}^\infty 2^{M^i}$
    such that $F_\mathrm{rel}(R) \in 2^{M^{a(R)}}$ and
  \item $F_\mathrm{fun} : S_\mathrm{fun} \to \bigcup_{i = 0}^\infty M^{M^i}$
    such that $F_\mathrm{fun}(f) \in M^{M^{a(f)}}$.
\end{itemize}
The set $M$ in the structure is called the domain of the structure.


\begin{definition}
  Let $\mathcal{S} = (S_\mathrm{rel}, S_\mathrm{fun}, a)$ be a signature and
  $\mathcal{M} = (M, F_\mathrm{rel}, F_\mathrm{fun})$ be a structure for
  $\mathcal{S}$.

  Let $t$ be a term in the signature $\mathcal{S}$ over the variables
  $x_1$, \dots, $x_n$ and $v_1, \dots, v_n \in M$.
  The value of $t$ with $x_1 = v_1$, \dots, $x_n = t_n$ with respect to the
  structure $\mathcal{M}$ is equal
  \begin{itemize}
    \item either to $v_i$ when $t = x_i$,
    \item or $F_\mathrm{fun}(f)(\mu_1, \dots, \mu_{a(f)})$ when
      $t = f(t_1, \dots, t_{a(f)})$, where $\mu_i$ is equal to the value of
      $t_i$ with $x_1 = v_1$, \dots, $x_n = v_n$ with respect to the
      structure $\mathcal{M}$.
  \end{itemize}

  Let $\phi$ be a formula in the signature $\mathcal{S}$ over the variables
  $x_1$, \dots, $x_n$.
  \begin{itemize}
    \item Let $\phi$ be equal to $F_\mathrm{rel}(R)(t_1, \dots, t_{a(R)})$,
      where $t_1, \dots, t_n$ are some terms in $\mathcal{S}$. Then the value
      of $\phi$ with $x_1 = v_1$, \dots, $x_n = v_n$ with respect to
      $\mathcal{M}$ is equal to $R(\mu_1, \dots, \mu_{a(R)})$, where $\mu_i$
      is equal to the value of $t_i$ with $x_1 = v_1$, \dots, $x_n = v_n$ with
      respect to $\mathcal{M}$.
    \item Let $\phi$ be equal to $\psi_1 \# \psi_2$, where
      $\# \in \set{\lor, \land}$ and $\psi_1$, $\psi_2$ are predicate
      formulas. Then the value of $\phi$ with $x_1 = v_1$, \dots, $x_n = v_n$
      with respect to $\mathcal{M}$ is equal to $\beta_1 \# \beta_2$, where
      $\beta_i$ is equal to the value of $\psi_i$ with $x_1 = v_1$, \dots,
      $x_n = v_n$ with respect to $\mathcal{M}$.
    \item Let $\phi$ be equal to $\lnot \psi$, where
      $\psi$ is a predicate formula.
      Then the value of $\phi$ with $x_1 = v_1$, \dots, $x_n = v_n$ with
      respect to $\mathcal{M}$ is equal to $\lnot \beta$, where
      $\beta$ is equal to the value of $\psi$ with $x_1 = v_1$, \dots,
      $x_n = v_n$ with respect to $\mathcal{M}$.
    \item Let $\phi$ be equal to $\exists x_i ~ \psi$, where
      $\psi$ is a predicate formula.
      Then the value of $\phi$ with $x_1 = v_1$, \dots, $x_n = v_n$ with
      respect to $\mathcal{M}$ is equal to true iff there is $\mu \in M$
      such that the value of $\psi$ with $x_1 = v_1$, \dots,
      $x_{i - 1} = v_{i - 1}$, $x_i = \mu$, $x_{i + 1} = v_{i + 1}$, \dots
      $x_n = v_n$ with respect to $\mathcal{M}$.
    \item Let $\phi$ be equal to $\forall x_i ~ \psi$, where
      $\psi$ is a predicate formula.
      Then the value of $\phi$ with $x_1 = v_1$, \dots, $x_n = v_n$ with
      respect to $\mathcal{M}$ is equal to true iff for all $\mu \in M$,
      the value of $\psi$ with $x_1 = v_1$, \dots,
      $x_{i - 1} = v_{i - 1}$, $x_i = \mu$, $x_{i + 1} = v_{i + 1}$, \dots
      $x_n = v_n$ with respect to $\mathcal{M}$.
  \end{itemize}
\end{definition}

We say that $\mathcal{M}$ is a model of a formula $\phi$
(written $\mathcal{M} \models \phi$)\footnote{
  Sometimes ``$\mathcal{M}$ is a model of $\phi$'' is written
  as $\models_\mathcal{M} \phi$.
} over the variables
$x_1$, \dots, $x_n$ iff the value of $\phi$ with $x_1 = v_1$, \dots,
$x_n = v_n$ with respect to $\mathcal{M}$ is equal to $\ltrue$ for all
$v_1, \dots, v_n \in \set{\ltrue, \lfalse}$.

We also say that $\phi$ is true in $\mathcal{M}$ if $\mathcal{M} \models \phi$,
and we say that $\phi$ is false in $\mathcal{M}$ if $\mathcal{M} \not\models \phi$.

Let us consider an example:
\begin{itemize}
  \item First, we define a signature
    $\mathcal{S} = (\set{=, <}, \set{+, \cdot}, \mathrm{ar})$ (if the arities of
    the symbols are clear from the context, we can write $S = (=, <; +, \cdot)$),
    where $\mathrm{ar}(x) = 2$ for any $x \in \set{<, =, +, \cdot}$.
  \item After this we define a structure $\mathcal{M} = (\R, F_\mathrm{rel},
    F_\mathrm{fun})$, where
    \begin{gather*}
      F_\mathrm{fun}(f)(x, y) = \begin{cases}
        x \cdot y & \text{if } f \text{ is } \cdot \\
        x + y & \text{if } f \text{ is } +
      \end{cases} \\
      \text{and} \\
      F_\mathrm{rel}(R)(x, y) \begin{cases}
        x = y & \text{if } R \text{ is } = \\
        x < y & \text{if } R \text{ is } <
      \end{cases}
    \end{gather*}
    Note that such a definition is pretty cumbersome, especially considering the
    fact that we use standard $+$ instead of the symbol $+$,
    standard $=$ instead of the symbol $=$ etc. So in similar cases we
    write $\mathcal{M} = (\R; =, <; +, \cdot)$.
  \item Finally, we consider the formulas in the signature $\mathcal{S}$
    \begin{gather*}
      \forall x \ \forall y \ x + y = y + x \\
      \text{and} \\
      \forall x \ \forall y \ \forall z \ (x < y \implies x + z < y + z).
    \end{gather*}
    (Note that we write $a = b$ instead of $=(a, b)$ and $a + b$ instead of
    $a + b$, this is a common notation when the standard mathematical
    operations and relations are used in the signature.)
\end{itemize}
The first formula says that the operation $+$ is commutative, which is true, so
the value of the formula with respect to the structure $\mathcal{M}$ should be
true. (Note that we do not mention the values of the variables $x$ and $y$
since both of them are not free.) Indeed, consider $a, b \in \R$ note that the
value of $x + y = y + x$ with $x = a$ and $y = b$ and with respect to the
structure $\mathcal{M}$ is equal to $F_\mathrm{rel}(=)(F_\mathrm{fun}(+)(a, b),
F_\mathrm{fun}(+)(b, a))$ which is the same as $a + b = b + a$; thus, the first
formula is true.

The second formula says that the inequalities are additive, so it should be also
true with respect to the structure $\mathcal{M}$.
\begin{exercise}
  Show that the second formula is true with respect to the structure
  $\mathcal{M}$.
\end{exercise}

\begin{exercise}
  Let us consider a signature $(=; +, \cdot, 0, 1)$ and two models with this
  signature: $\mathfrak{R} = (\R; =; +, \cdot, 0, 1)$, and
  $\mathfrak{Q} = (\Q; =; +, \cdot, 0, 1)$.
  Find a predicate formula $\phi$ in this signature such that
  $\mathfrak{R} \models \phi$ but $\mathfrak{Q} \not\models \phi$.
\end{exercise}


\section{Natural Deduction}
\marginurl{%
  Natural Deduction for Predicate Logic:\\\noindent
  Introduction to Mathematical Logic \#6
}{youtu.be/GVht3ES2qqo}
By analogy with the tautology, in the predicate logic we wish to prove that a
formula is true, whenever the structure and the values of the variables we
choose. Such formulas are called \textit{logically valid}.

In addition, we may define semantic implication for predicate formulas.
We say that a set of predicate formulas $\Sigma$ in a signature $\mathcal{S}$
semantically implies a formula $\phi$ ($\Sigma \models \phi$) in the signature
iff any structure with the signature $\mathcal{S}$ modeling $\Sigma$ models
$\phi$ as well.

Natural deduction for the predicate formulas is defined in the same manner as
the natural deduction for the propositional formulas but now the lines are
predicate formulas and we can use four additional rules.

\paragraph{Universal quantifier.}
The first logically-valid formula we use as a rule is
$A(x) \implies (\forall y \ A(y))$,
this rule allows us to introduce a universal quantifier.
In order to use the following rule, $x$ should not be a free variable of
an open hypothesis.
\[
  \begin{nd}
    \have [m] {1} {A(x)}
    \have [~] {2} {\forall y \  A(y)} \Ai{1}
  \end{nd}
\]
The second logically-valid formula we use as a rule says that if a statement is
true for all the values of a variable, then it is also true when you substitute
some specific term instead of the variable, i.e. $(\forall x \ A(x)) \implies
A(t))$, this rule allows us to eliminate an universal quantifier.
\[
  \begin{nd}
    \have [m] {1} {\forall x \  A(x)}
    \have [~] {2} {A(t)} \Ae{1}
  \end{nd}
\]

\paragraph{Existential quantifier.}
The first formula for the exisential quantifier says that you can name any term
in the formula by a variable and formula is still true for some value of the
variable. The corresponding formula is $A(t) \implies (\exists x \ A(x))$.
\[
  \begin{nd}
    \have [m] {1} {A(t)}
    \have [~] {2} {\exists x \ A(x)} \Ei{1}
  \end{nd}
\]

The last rule says that if $A(x)$ is true for some $x$ and we know that $A(y)$
implies $B$, then we can derive $B$ (note that this is true only when
$y$ is not used in $B$). Thus we can apply the following rule when $y$ is not
be a free variable neither of $B$ nor of any open hypothesis.
\[
  \begin{nd}
    \have [m] {1} {\exists x \ A(x)}
    \open
      \hypo [i] {2} {A(y)}
      \have[j] {3} {B}
    \close
    \have [~] {4} {B} \Ee{1, 2-3}
  \end{nd}
\]
\section{Examples of Derivations}
First example $\forall x \ F(x) \lor \lnot(\forall x \ F(x))$ is a special form
of the law of excluded middle, which we proved
in the previous chapter. However, in order to emphasize that the propositional
logic can prove all the statements provable in the predicate case we present
the proof of this statement as well.

\noindent $
  \begin{nd}
    \hypo {1} {}
    \open
      \hypo {2} {\lnot (\forall x \  F(x) \lor \lnot (\forall x \  F(x)))}
      \open
        \hypo {3} {\forall x \  F(x)}
        \have {4} {\forall x \  F(x) \lor \lnot (\forall x \  F(x))} \oi{3}
        \have {5} {\perp} \ne{2, 4}
      \close
      \have {6} {\lnot (\forall x \  F(x))} \ni{3-5}
      \have {7} {\forall x \  F(x) \lor \lnot (\forall x \  F(x))} \oi{6}
      \have {8} {\perp} \ne{2, 8}
    \close
    \have {9} {\forall x \  F(x) \lor \lnot (\forall x \  F(x))} \by{IP}{2-8}
  \end{nd}
$

Unfortunately, this example just shows that a statement provable in the
propositional logic can be proven in the predicate logic. The next example is
an example that cannot be expressed in the propositional logic, we
prove that if we know that
$\forall x \forall y \ R(x, y) \implies R(y, x)$, the we can derive
$\forall x \forall y \ ((R(x, y) \implies R(y, x)) \land
  (R(y, x) \implies R(x, y)))$.


\noindent $
  \begin{nd}
    \hypo {1} {\forall x \forall y \ R(x, y) \implies R(y, x)}

    \have {2} {\forall y \ R(x', y) \implies R(y, x')} \Ae{1}
    \have {3} {R(x', y') \implies R(y', x')} \Ae{2}
    \have {4} {\forall y \ R(y', y) \implies R(y, y')} \Ae{1}
    \have {5} {R(y', x') \implies R(x', y')} \Ae{4}
    \have {6} {(R(x', y') \implies R(y', x')) \land R(y', x') \implies R(x', y')}
              \ai{3, 5}
    \have {6} {\forall y \ (R(x', y) \implies R(y, x')) \land
      (R(y, x') \implies R(x', y))}
              \Ai{6}
    \have {6} {\forall x \forall y \ (R(x, y) \implies R(y, x)) \land
      (R(y, x) \implies R(x, y))}
              \Ai{7}
  \end{nd}
$

\section{Soundness and Completeness}
Like in the propositional case, the most important properties of the natural
deduction are the following two theorems.

\begin{theorem}[completeness of natural deductions, G\"odel]
    Let $\phi$ be a predicate formula. If $\phi$ is logically valid, then
    there is a proof of $\phi$. Moreover, if $\Sigma \models \phi$,
    for some finite set of predicate formulas $\Sigma$, then there is a
    derivation of $\phi$ from $\Sigma$.
\end{theorem}

\begin{theorem}[soundness of natural deductions]
    Let $\phi$ be a predicate formula. If there is a proof of $\phi$, then
    $\phi$ is logically valid. Moreover, if there is a derivation of $\phi$ from
    $\Sigma$, for some finite set of predicate formulas $\Sigma$, then
    $\Sigma \models \phi$.
\end{theorem}

\begin{chapterendexercises}
  \exercise Give a natural deduction derivation of
    $\forall x \ A(x) \implies \forall x \ B(x)$ from
    $\forall x \ (A(x) \implies B(x))$.
  \exercise Give a natural deduction derivation of
    $\exists x \ ( A(x) \lor B(x))$ from
    $\exists x \  A(x) \lor \exists x \  B(x)$.
\end{chapterendexercises}
