\chapter{Trees}
\label{chapter:trees}
Let us consider the following problem. Given a network of several computers,
in this network if a computer $A$ receives some message from a computer $B$,
it broadcasts it to all the connected computers except $B$.
However, in such setting there is an issue known as broadcast radiation.
Assume we have three computers $A$, $B$, and $C$ such that they form a cycle.
If $A$ sends something to $B$ and $C$ both of them send received information to
$C$ and $B$, respectively; after that $B$ and $C$ send this information to $A$
and $A$ start sending this information again, which leads to an infinite
cycle.\footnote{%
  This problem is a simplified version of a problem that is solved by STP
  protocols in the modern networks.
}

Therefore to avoid such problem we need to disable some connecion so that the
graph of this network does not have cycles.
In this chapter we are going to study properties of the graphs without cycles.
\begin{definition}
  We say that a connected graph $G$ is a \emph{tree} iff $G$ does not have cycles.
\end{definition}

\section{Minimally Connected Graphs}
First we may make the following observation.
\begin{theorem}
  Let $G = (V, E)$ be a connected graph. Then the following statements are
  equivalent.
  \begin{itemize}
    \item $G$ is a tree.
    \item $G$ is minimally connected, that is, $G - e$ is not connected for any
      $e \in E$.
  \end{itemize}
\end{theorem}
\begin{proof}
  Assume that $G$ is minimally connected but $G$ has a cycle
  $v_1$, \dots, $v_k$. Consider $G' = G - (v_1, v_k)$, we claim that $G'$
  is still connected. Indeed, let $x$ and $y$ be some vertices of $G'$.
  Since $G$ is connected, there is a path $p$ from $x$ to $y$. If $p$ does not
  contain the edge $(v_1, v_k)$, then $x$ and $y$ are connected in $G'$. If
  $p$ contains $(v_1, v_k)$, then we replace this edge by the path $v_1$, \dots,
  $v_k$, so $x$ and $y$ are connected in $G'$. Therefore $G$ is not a minimally
  connected graph, which is a contradiction.

  Let us now assume that $G$ is not minimally connected, we wish to prove that
  it implies that $G$ is not a tree. Since $G$ is not minimally connected, there
  is an edge $(x, y) \in E$ such that $G - e$ is connected. Since $G - (x, y)$
  is connected, there is a path $x = v_1, \dots, v_k = y$ in $G - (x, y)$.
  Therefore $v_1, \dots, v_k$ is a cycle in $G$, which is a contradiction.
 \end{proof}

Therefore in order to get a tree from a graph, we just need to delete edges
in an arbitrary way until the moment when we cannot delete them anymore.
\begin{corollary}
  For any connected graph $G = (V, E)$, there is a tree $T = (V, E')$ such
  that $T$ is a subgraph of $G$. Such tree is called a spanning tree of $G$.
\end{corollary}

Another question we may ask is how many edges we need to delete in this process.
Aparenly, the answer is always $m - n + 1$, where $m$ is the number of edges in
the initial graph and $n$ is the number of vertices.
\begin{theorem}
\label{theorem:tree-is-minimally-connected}
  Let $G$ be a connected graph on $n$ vertices.
  If $G$ is a tree, then it has $n - 1$ edges. Moreover, if $G$ has $n - 1$
  edge, then it is a tree.
\end{theorem}

Before we prove the theorem, let us prove the following lemma.
\begin{lemma}
  If a tree $T$ has at least $2$ vertices, then it has at least two
  vertices whose degree is $1$.
\end{lemma}
\begin{proof}
  Let us choose a vertex $v$ of $T$ such that its degree is not $1$ (if such a
  vertex does not exist, then we found at least $2$ vertices whose degree is
  $1$). Let us start walking from $v$ to its neighbour, then to a new neighbor
  of this neighbor, and so on, never revisiting a vertex. As $T$ has finite
  number of vertices, we will eventually have to stop at a vertex $u$. We claim
  that the only reason for us to stop at $u$ could be that $u$ is of degree $1$.
  Indeed, the only possible other reason would be that $u$ has neighbors other
  than the neighbor $u'$ we reached $u$ from, but they have all been visited
  already. However, that would mean that there are at least two paths from $v$
  to $u$, and that cannot happen in a tree. So $u$ is of degree $1$. To get
  another vertex of degree $1$, remember that $v$ is of degree more than $1$.
  So take another neighbor of $v$, and repeat this argument. This will result in
  another vertex $w$ of degree $1$, and $u \neq w$ as that would again yield two
  paths from $v$ to $u$.
\end{proof}
The vertices of a tree that have degree $1$ are called \emph{leaves}.

\begin{proof}[Proof of Theorem~\ref{theorem:tree-is-minimally-connected}]
  We prove the statement using induction by $n$. If $n = 1$, the statement is
  clearly true.
  Assume that the statement is true for trees on $n$ vertices. Consider a tree
  $T$ on $n + 1$ vertices. Consider a leaf $\ell$ of $T$. Note that $T - \ell$
  is a tree as well, therefore by the induction hypothesis, it has $n - 2$
  edges. Hence, $T$ has $n - 1$ edges.

  Let us now prove that if a graph $G$ has $n - 1$ edges and is connected,
  then $G$ is a tree. Assume that it is not a tree, we start deleting edges as
  long as the graph is connected, we call the resulting graph $T$. Note that
  $T$ is minimally connected, so $T$ is a tree. Note that $T$ has $n$ vertices.
  Therefore, it has $n - 1$ eges, which implies that we removed $0$ edges and
  $T = G$. As a result, $G$ is a tree.
\end{proof}

\begin{exercise}
  A graph such that every connected component of this graph is a tree is called
  a \emph{forest}.
  Show that a forest with $k$ connected components has $n - k$ edges.
\end{exercise}

\section{Minimum-weight Spanning Trees}
In the initial example about the network, we missed an important detail: not
all the connections are equally fast. Let us label each connection
(edge in our graph) with the weight (the number that represents how slow is this
connection). So now we need to choose a spanning tree of
the graph of the network so that it has the minimal possible sum of weights.
\begin{definition}
  Let $G = (V, E)$ be a connected graph, and $w : E \to \R$ be weights of edges.
  Then we say that a spanning tree $T = (V, E')$ of $G$ is a
  \emph{minimum-weight spanning tree} of $G$ if
  $\sum_{e \in E'} w(e) \le \sum_{e \in E''} w(e)$ for any spanning tree
  $T' = (V, E'')$ of $G$.

  The number $\sum_{e \in E'} w(e)$ is called the \emph{weight} of $T$.
\end{definition}

It is obvious that such a tree exists. The question is
``how to find efficiently the minimum-weight spanning tree''.
\begin{exercise}
  Let $G = (V, E)$ be some graph and $w : E \to \R$ be a weight function such
  that $w(e) = 1$. How to to find efficiently the minimum-weight spanning tree
  of $G$?
\end{exercise}


Surprisingly, one may find such a minimum-weight spanning tree using a simple
greedy algorithm (Algorithm~\ref{algorithm:kruskal}).
\begin{algorithm}
  \begin{algorithmic}[1]
    \Function{MinimumSpanningTree}{$n$, $E$, $w$}
      \State Let $e_1$, \dots, $e_m$ be the edges from $E$ sorted in the
        ascending order with respect to $w$.
      \State $i \gets 1$
      \State Set $T$ to be an empty graph on $[n]$.
      \label{line:kruskal-while}
      \While{$i \le n$}
        \If{$T + e_i$ does not have cycles}
          \State $T \gets T + e_i$
        \EndIf
        \State Increase $i$ by $1$.
      \EndWhile
      \State \Return{$T$}
    \EndFunction
  \end{algorithmic}
  \caption{Kruskal's algorithm, the algorithm that returns a minimum-weight
  spanning tree of the graph on $[n]$ with the set of edges $E$.}
  \label{algorithm:kruskal}
\end{algorithm}

\begin{theorem}
\label{theorem:kruskal}
  If the graph $([n], E)$ is connected, then Algorithm~\ref{algorithm:kruskal}
  returns a minimum-weight spanning tree of the graph $([n], E)$.
\end{theorem}

To prove this statement we need a technical lemma.
\begin{lemma}
  Let $F_1$ and $F_2$ be forests on the same vertex set $V$. If $F_1$ has less
  edges than $F_2$, then $F_2$ has an edge $e$ not in $F_1$ so
  that the graph $F_1 + e$ is still a forest.
\end{lemma}
\begin{proof}
  Let $E_i$ be the set of edges of $F_i$.
  Assume that there such edge does not exist; i.e., $F_1 + e$ has a cycle for
  any edge $e \in E_2 \setminus E_1$.

  Therefore any edge of $F_2$ is between two vertices in the same component of
  $F_1$. Hence, $F_2$ has at least as many connected components as $F_1$.
  Indeed, consider two connected components $U_1$ and $U_2$ of $F_1$ we claim
  that they any $x \in U_1$ and $y \in U_2$ are not connected in $F_2$ since
  there are no edges going outside of $U_1$ and $U_2$ in $F_2$.

  However, $F_i$ has $n - |E_i|$ connected components, which contradicts to the
  fact that $|E_1| < |E_2|$.
\end{proof}

\begin{proof}[Proof of Theorem~\ref{theorem:kruskal}]
  Let $T_1$, \dots, $T_m$ be the states of $T$ after iterations of
  line~\ref{line:kruskal-while} of Algorithm~\ref{algorithm:kruskal}.
  Note that $T_i$ does not have cycles for $i \in \range{m}$. Therefore $T_i$ is a
  forest for $i \in \range{m}$.

  First we need to prove that Algorithm~\ref{algorithm:kruskal} returns a
  spanning tree; i.e. that $T_m$ is connected.
  Assume the opposite. Consider two vertices $x, y \in \range{n}$ such that
  they are not connected in $T$. Since $G = ([n], E)$ is connected there is a
  path $x = v_1, \dots, v_k = y$ in $G$. Consider the minimal $i \in [k - 1]$
  such that $(v_i, v_{i + 1})$ is not an edge of $T_m$. Let
  $e_j = (v_i, v_{i + 1})$. It is easy to see that $T_m + (v_i, v_{i + 1})$ does
  not have cycles so $T_{j - 1} + e_j$ does not have cycles as well and $T_j =
  T_{j - 1} + e_j$ which implies that $T_m$ has the edge $(v_i, v_{i + 1})$
  which is a contradiction.

  Before we start the second part of the proof note that if $w(e) < w(e_i)$,
  then $T_{i - 1} + e_i$ has a cycle.

  Now we need to prove that $T_m$ is a minimum-weight spanning tree. Assume
  that there is a spanning tree $H$ such that the weight of $H$ is less than
  the weight of $T_m$. Consider the edges $t_1$, \dots, $t_{n - 1}$ of $T_m$
  and the edges $h_1$, \dots, $h_{n - 1}$ of $H$ such that
  $w(t_1) \le w(t_2) \le \dots \le w(t_{n - 1})$ and
  $w(h_1) \le w(h_2) \le \dots \le w(h_{n - 1})$.
  Let us consider the first step when $H$ is better than $T_m$; i.e., the
  minimal $i$ so that $\sum_{j = 1}^i w(h_j) < \sum_{j = 1}^i w(t_j)$
  (obviously $i > 1$).

  It is easy to see that $h_i < t_i$. Let $e_j = t_i$ and
  $H_i = H[h_1, \dots, h_i]$. Since $H_i$ has more edges than $T_j$ there is an
  edge $h_{i'}$ ($i' < i$) such that $T_{j - 1} + h_{i'}$ does not have cycles.
  Wich is a contradiction since $h_{i'} < h_i < t_i$.
\end{proof}


\begin{chapterendexercises}
  \exercise[recommended] Let $G$ be a graph with $k$ connected components and
    $n - k$ edges. Show that $G$ is a forest.
  \exercise Prove that if $G$ is a simple graph on $[n]$, then at least one of
    $G$ and its complement is connected. Show an example when they are both
    connected. The complement $\bar{G}$ of $G$ has the same vertex set as $G$
    and $(x, y)$ is an edge in $\bar{G}$ if and only if it is not an edge in
    $G$.
  \exercise Let $H$ be a simple graph on $n$ vertices that has $m$ edges. Prove
    that $H$ contains at least $m - n - 1$ cycles.
\end{chapterendexercises}
