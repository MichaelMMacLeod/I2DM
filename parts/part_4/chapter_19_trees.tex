\chapter{Trees}
Let us consider the following problem. Given a network of several computers,
in this network if a computer $A$ receives some message from a computer $B$,
it broadcasts it to all the connected computers except $B$.
However, in such setting there is an issue known as broadcast radiation.
Assume we have three computers $A$, $B$, and $C$ such that they form a cycle.
If $A$ sends something to $B$ and $C$ both of them send received information to
$C$ and $B$, respectively; after that $B$ and $C$ send this information to $A$
and $A$ start sending this information again, which leads to an infinite cycle.

Therefore to avoid such problem we need to disable some connecion so that the
graph of this network does not have cycles.
In this chapter we are going to study properties of the graphs without cycles.
\begin{definition}
  We say that a connected graph $G$ is a \emph{tree} iff $G$ does not have cycles.
\end{definition}

\section{Minimally Connected Graphs}
First we may make the following observation.
\begin{theorem}
  Let $G = (V, E)$ be a connected graph. Then the following statements are
  equivalent.
  \begin{itemize}
    \item $G$ is a tree.
    \item $G$ is minimally connected, that is, $G - e$ is not connected for any
      $e \in E$.
  \end{itemize}
\end{theorem}

Therefore in order to get a tree from a graph, we just need to delete edges
in an arbitrary way until the moment when we cannot delete them anymore.
\begin{corollary}
  For any connected graph $G = (V, E)$, there is a tree $T = (V, E')$ such
  that $T$ is a subgraph of $G$. Such tree is called a spanning tree of $G$.
\end{corollary}

Another question we may ask is how many edges we need to delete in this process.
Aparenly, the answer is always $m - n + 1$, where $m$ is the number of edges in
the initial graph and $n$ is the number of vertices.
\begin{theorem}
  Let $G$ be a connected graph on $n$ vertices.
  If $G$ is a tree, then it has $n - 1$ edges. Moreover, if $G$ has $n - 1$
  edge, then it is a tree.
\end{theorem}

\section{Minimum-weight Spanning Trees}
In the initial example about the network, we missed an important detail: not
all the connections are equally fast. Let us label each connection
(edge in our graph) with the weight (the number that represents how slow is this
connection). So now we need to choose a spanning tree of
the graph of the network so that it has the minimal possible sum of weights.
\begin{definition}
  Let $G = (V, E)$ be a connected graph, and $w : E \to \R$ be weights of edges.
  Then we say that a spanning tree $T = (V, E')$ of $G$ is a
  \emph{minimum-weight spanning tree} of $G$ if
  $\sum_{e \in E'} w(e) \le \sum_{e \in E''} w(e)$ for any spanning tree
  $T' = (V, E'')$ of $G$.

  The number $\sum_{e \in E'} w(e)$ is called the \emph{weight} of $T$.
\end{definition}

It is obvious that such a tree exists. The question is
``how to find efficiently the minimum-weight spanning tree''.
\begin{exercise}
  Let $G = (V, E)$ be some graph and $w : E \to \R$ be a weight function such
  that $w(e) = 1$. How to to find efficiently the minimum-weight spanning tree
  of $G$?
\end{exercise}


Surprisingly, one may find such a minimum-weight spanning tree using the
following greedy algorithm.
\begin{algorithm}
  \begin{algorithmic}[1]
    \Function{Kruskal}{$n$, $E$, $w$}
      \State Let $e_1$, \dots, $e_m$ be the edges from $E$ sorted in the
        ascending order with respect to $w$.
      \State $i \gets 1$
      \State Set $T$ to be an empty graph.
      \While{$i \le n$}
        \If{$T + e_i$ does not have cycles}
          \State $T \gets T + e_i$
        \EndIf
        \State Increase $i$ by $1$.
      \EndWhile
      \State \Return{$T$}
    \EndFunction
  \end{algorithmic}
  \caption{Kruskal's algorithm, the algorithm that returns a minimum-weight
  spanning tree of the graph on $[n]$ with the set of edges $E$.}
  \label{algorithm:kruskal}
\end{algorithm}

\begin{theorem}
  If the graph $([n], E)$ is connected, then Algorithm~\ref{algorithm:kruskal}
  returns a minimum-weight spanning tree of the graph $([n], E)$.
\end{theorem}
