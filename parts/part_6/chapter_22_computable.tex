\chapter{Computable Functions}
In this part we study what computers can compute and what they cannot compute.
Usually study of this subject starts from a formal definition of an algorithm.
However, we believe that this is unnecessary nowadays because of rise of
computers. One may think about algorithms as programs on some programming
language such as C/C++, Java, Python etc.

An algorithms are taking several natural numbers $x_1$, \dots, $x_n$
as an input and either print another number $y$ as an output or never
terminates. In the first case we say $\Algorithm{A}(x_1, \dots, x_n) = y$
and in the second case we say $\Algorithm{A}(x_1, \dots, x_n)$ never terminates.

The first and the most basic definition in the computability theory is the
definition of a computable function.
\begin{definition}
    Let $S \subseteq \N^\ell$ and $f : S \to \N$.
    We say that $f$ is computable if there is an algorithm $\Algorithm{A}$
    \begin{enumerate}
        \item $\Algorithm{A}(x) = f(x)$ for any $x \in S$ and
        \item $\Algorithm{A}(x)$ never terminates for any $x \notin S$.
    \end{enumerate}
\end{definition}

It is important to note that nonetheless that we say that the algorithms are
said to take and print natural numbers, we could allow algorithm to work with
strings of bits (elements of the set
$\set{0, 1}^* = \bigcup_{n \in \N_0} \set{0, 1}^n$).
Moreover, these two definitions are  equivalent since there is a one-to-one
correspondence between natural numbers and strings:
$x \mapsto 2^n + \sum_{i = 1}^\ell 2^{i - 1} x[i]$, where $x[i]$ denotes the
$i$th symbol of the string $i$ and $n$ is the length of $x$.
It is also clear that using binary strings we may encode all sorts of objects
such as pairs of integers, integers, rational numbers etc. (However, in order
to encode real number we need more complicated definitions and we are not
going to discuss them in here.)

One may show that composition of computable functions is computable.
Moreover, one may prove the following a bit stronger statement.
\begin{theorem}
\label{theorem:composition-computable}
    Let $S \subseteq \N$, and let
    $f_1 : S \to \N$, $f_2 : S \to \N$, and $g : \N^2 \to \N$ be computable
    functions. Then the function $h$ such that $h(x) = g(f_1(x), f_2(x))$
    is also computable.
\end{theorem}

Another important notion is the notion of a decidable set.
\begin{definition}
    We say that a set $S \subseteq \N$ is decidable iff there is
    an algorithm $\Algorithm{A}$ such that
    $\Algorithm{A}(x) = 1$ if $x \in S$ and $\Algorithm{A}(x) = 0$ if
    $x \notin S$.
\end{definition}
It is easy to note that a set $S \subseteq \N$ is decidable iff the
characteristic function $\chi_S$ of $S$ is computable.
The function $\chi_S$ is defined as follows:
\[
    \chi_S(x) =
    \begin{cases}
        1 & \text{if } x \in S, \\
        0 & \text{otherwise}.
    \end{cases}
\]
