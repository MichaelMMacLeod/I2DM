\chapter{Decidable Sets}
\marginpar{%
  This part is mainly based on the amazing book
  ``Computable Functions''
  by Shen and Vereshchagin.
}
In this part we study what computers can compute and what they cannot compute.
Usually study of this subject starts from a formal definition of an algorithm.
However, we believe that this is unnecessary nowadays because of rise of
computers. One may think about algorithms as programs on some programming
language such as C/C++, Java, Python etc.

An algorithms are taking several natural numbers $x_1$, \dots, $x_n$
as an input and either print another number $y$ as an output or never
terminates. In the first case we say $\Algorithm{A}(x_1, \dots, x_n) = y$
and in the second case we say $\Algorithm{A}(x_1, \dots, x_n)$ never terminates.

\section{Computable Functions}
The first and the most basic definition in the computability theory is the
definition of a computable function.
\begin{definition}
    Let $S \subseteq \N^\ell$ and $f : S \to \N$.
    We say that $f$ is \emph{computable} if there is an algorithm
    $\Algorithm{A}$ such that
    \begin{enumerate}
        \item $\Algorithm{A}(x) = f(x)$ for any $x \in S$ and
        \item $\Algorithm{A}(x)$ never terminates for any $x \notin S$.
    \end{enumerate}

    We say that $\Algorithm{A}$ \emph{computes} $f$.
\end{definition}

It is important to note that nonetheless that we say that the algorithms are
said to take and print natural numbers, we could allow algorithm to work with
strings of bits (elements of the set
$\strings = \bigcup_{n \in \N_0} \set{0, 1}^n$).
Moreover, these two definitions are  equivalent since there is a one-to-one
correspondence between natural numbers and strings:
$x \mapsto 2^n + \sum_{i = 1}^\ell 2^{i - 1} x[i]$, where $x[i]$ denotes the
$i$th symbol of the string $i$ and $n$ is the length of $x$.
It is also clear that using binary strings we may encode all sorts of objects
such as pairs of integers, integers, rational numbers etc. (However, in order
to encode real number we need more complicated definitions and we are not
going to discuss them in here.)

\begin{exercise}
  Let $f : \strings \to \strings$ be the function such that $f(x)$ is
  reversed $x$. Show that $f$ is computable.
\end{exercise}

One may show that composition of computable functions is computable.
Moreover, one may prove the following a bit stronger statement.
\begin{theorem}
\label{theorem:composition-computable}
    Let $S \subseteq \N$, and let
    $f : \N \to \N$, and $g : S \to \N$ be computable
    functions. Then $f \circ g$ is also computable.
\end{theorem}
\begin{proof}
  Since $f$ and $g$ are computable, there are algorightms $\Algorithm{F}$ and
  $\Algorithm{G}$ computing f and g respectively.

  Let us consider the following algorithm.
  \begin{algorithm}
      \begin{algorithmic}[1]
          \Function{$\Algorithm{A}$}{$x$}
              \State{$y \gets \Algorithm{G}(x)$}
              \State{\Return{$\Algorithm{F}(y)$}}
          \EndFunction
      \end{algorithmic}
      \caption{The algorithm computing the composition of the functions
        computed by $\Algorithm{F}$ and $\Algorithm{G}$}
  \end{algorithm}
  It is clear that if $x \notin S$, then $\Algorithm{A}(x)$ never terminates.

  However, if $x \in S$, then $y = g(x)$ and therefore the algorithm prints
  $\Algorithm{F}(g(x)) = f(g(x)) = (f \circ g)(x)$.
\end{proof}

\section{Decidable Sets}
Another important notion is the notion of a decidable set.
\begin{definition}
    We say that a set $S \subseteq \N$ is \emph{decidable} iff there is
    an algorithm $\Algorithm{A}$ such that
    $\Algorithm{A}(x) = 1$ if $x \in S$ and $\Algorithm{A}(x) = 0$ if
    $x \notin S$.
\end{definition}
It is easy to note that a set $S \subseteq \N$ is decidable iff the
characteristic function $\chi_S$ of $S$ is computable.
The function $\chi_S$ is defined as follows:
\[
    \chi_S(x) =
    \begin{cases}
        1 & \text{if } x \in S, \\
        0 & \text{otherwise}.
    \end{cases}
\]
Similarly we may define decidable sets of strings, pairs of integers etc.

We illustrate this concept by proving that
$U_e = \set[q > e]{q \in \Q}$. It is known that
\begin{enumerate}
    \item $(1 + \frac{1}{n})^n < e$ and
        $\lim\limits_{n \to \infty} (1 + \frac{1}{n})^n = e$ and
    \item $(1 + \frac{1}{n})^{n + 1} > e$ and
        $\lim\limits_{n \to \infty} (1 + \frac{1}{n})^{n + 1} = e$.
\end{enumerate}
Hence, in order to check whether $q \in U_e$ or not, it is enough to either find
$n$ such that $(1 + \frac{1}{n})^{n + 1} < q$ or $(1 + \frac{1}{n})^n > q$.
In order to do this we can check all positive integers one after another
and at some point one of the inequalities became true.

\begin{exercise}
  Let $k \in \N$. Show that $[k]$ is decidable.
\end{exercise}

However, sometimes it is possible to show that some set is decidable without
presenting the algorithm explicitly.
For example, consider the $S \subseteq \N$ such that $n \in S$ iff base $10$
expansion of $\pi$ has $n$ consecutive $9$s. It is possible to show that $S$ is
decidable. Indeed, it is easy to see that either $S = \N$ or $S = [k]$ for some
$k \in \N$, however, in both cases the set is decidable.


It is easy to show that decidable sets have several good properties.
\begin{theorem}
\label{theorem:operations-decidable}
  Let $S_1, S_2 \subseteq \N^\ell$ be decidable sets. Then $S_1 \cup S_2$,
  $S_1 \cap S_2$, and $S_1 \setminus S_2$ are all decidable.
\end{theorem}

To prove the theorem, we generalize \Cref{theorem:composition-computable}.
\begin{theorem}
\label{theorem:composition-computable-generalization}
  Let $S \subseteq \N^\ell$ and let $f_1, f_2 : S \to \N$ and $g : \N^2 \to \N$
  be computable functions. Then the function $h : S \to \N$ such that
  $h(x) = g(f_1(x), f_2(x))$ is computable.
\end{theorem}

\begin{proof}[Proof of \Cref{theorem:operations-decidable}]
  Let us prove that $S_1 \cup S_2$ is decidable. Since $S_1$ and $S_2$ are
  decidable, there are algorithms $\Algorithm{A}_1$ and $\Algorithm{A}_2$
  computing characteristic functions $\chi_{S_1}$ and $\chi_{S_2}$
  for the sets $S_1$ and $S_2$, respectively.
  Consider the function $g : \N^2 \to \N$ such that
  \[
    g(x, y) = \begin{cases}
      0 & \text{if } x = 1 \text{ or } y = 1 \\
      1 & \text{otherwise}
    \end{cases}.
  \]
  It is clear that $h : \N \to \N$ such that
  $h(x) = g(\chi_{S_1}(x), \chi_{S_2}(x))$ is a characteristic function for
  $S_1 \cup S_2$. Hence, $S_1 \cup S_2$ is decidable by
  \Cref{theorem:composition-computable-generalization}.

  The proof of decidability of $S_1 \cap S_2$, and $S_1 \setminus S_2$
  is essentially the same.
\end{proof}

\section{Enumeratable Sets}
The algorithm constructed in the proof of decidability of $U_e$ consisted of
two important parts: first one allowed to show that $q$ is defenetely not in the
set $U_e$ and the second part allowed to show that $q$ is defenetely in the set.

This observation leads to the following definition.
\begin{definition}
  We say that a set $S \subseteq \N$ is \emph{(computably) enumerable},
  we also say that it is \emph{recursively enumerable} and \emph{semidecidable},
  iff there is an algorithm $\Algorithm{A}$ such that
  \begin{enumerate}
    \item $\Algorithm{A}(x) = 1$ for any $x \in S$ and
    \item either $\Algorithm{A}(x) = 0$ or $\Algorithm{A}$ never terminates for
      any $x \notin S$.
  \end{enumerate}

  We say that $S$ is \emph{semidecided} by $\Algorithm{A}$.
\end{definition}

We can easily show that both sets $L_e = \set[q < e]{q \in \Q}$ and
$U_e = \set[q > e]{q \in \Q}$ are enumerable. Indeed, we can try all possible
$n \in n$ until $q < (1 + \frac{1}{n})^n$ if we find such $n$, we know that
$q \in L_e$. Similarly, we can try all possible $n$ until
$q > (1 + \frac{1}{n})^{n + 1}$ and if we find such $n$, then $q \in U_e$.
The following allows us to show that the fact that $L_e$ is decidable is not
a coincedence.
\begin{theorem}[Post's Theorem]
\label{theorem:enumerable-to-decidable}
  Let $S \subseteq \N$. If $S$ is decidable, then $S$ is enumerable.
  Moreover, if $S$ and $\N \setminus S$ are enumerable, then $S$ is
  decidable.
\end{theorem}
\begin{proof}
  The first part is obvious. Let us prove the ``moreover'' part.
  Let $\Algorithm{A}_1$ and $\Algorithm{A}_2$ be the algorithms deciding
  $S$ and $\N \setminus S$ respectively. Then the algorithm $\Algorithm{A}$
  deciding $S$ is the following: on onput $x$ it runs $\Algorithm{A}_1(x)$ and
  $\Algorithm{A}_2(x)$ in parallel and if the first one prints $1$,
  $\Algorithm{A}$ prints $1$ as well;
  however, if the second prints $1$, $\Algorithm{A}$ prints $0$.

  We need to prove that the algorithm works correctly.
  \begin{itemize}
    \item If $x \in S$, then $\Algorithm{A}_1(x) = 1$ and $\Algorithm{A}_2(x)$
      never terminates. So $\Algorithm{A}(x)$ prints $1$.
    \item If $x \notin S$, then $\Algorithm{A}_1(x)$ never terminates and
      $\Algorithm{A}_2(x) = 1$. So $\Algorithm{A}(x)$ prints $0$.
  \end{itemize}
\end{proof}

The given definition of enumerable set does not explain the name. However,
there is an alternative definition that explains it.
\begin{theorem}
\label{theorem:enumerable-semidecidable}
  Let $S \subseteq \N$. The set $S$ is decidable iff there is an algorithm
  $\Algorithm{A}$ such that
  \begin{enumerate}
    \item $\Algorithm{A}(n)$ terminates for any $n \in \N$ and
    \item $\set[n \in  \N]{\Algorithm{A}(n)} = S$.
  \end{enumerate}

  We say that this $\Algorithm{A}$ is enumerating $S$.
\end{theorem}
\begin{proof}
  To prove this theorem, we generalize the idea of
  \Cref{theorem:enumeratable-to-decidable}. Assume that $S$ is infinite.
  Let $\Algorithm{A}'$ be the algorithm semideciding $S$ and
  let $\Algorithm{A}$ be \Cref{algorithm:enumerating-from-semideciding}.
  \begin{algorithm}
      \begin{algorithmic}[1]
          \Function{$\Algorithm{A}$}{$n$}
              \State{$i \gets 1$}
              \State{Let $V$ be a map from integers to $\set{0, 1}$}
              \While{$V$ has less than $n$ keys with the value $1$}
                \Parallel
                  \State{Let $y = \Algorithm{A}'(i)$}
                  \State{Put $(i, y)$ into $V$}
                \EndParallel
                \State{$i \gets (i + 1)$}
              \EndWhile
              \State{\Return{the $i$th key in $V$ with the value $1$}}
          \EndFunction
      \end{algorithmic}
      \caption{The algorithm enumerating the set that is semidecided by
        $\Algorithm{A}'$.}
      \label{algorithm:enumerating-from-semideciding}
  \end{algorithm}
  It is clear that $\Algorithm{A}$ satisfies the constraints of the theorem.

  Let us prove the statement in the opposite direction. Let us assume that
  there is an algorithm $\Algorithm{A}'$ enumerating $S$. Let $\Algorithm{A}$
  be \Cref{algorithm:semideciding-from-enumerating}.
  \begin{algorithm}
      \begin{algorithmic}[1]
          \Function{$\Algorithm{A}$}{$x$}
              \State{$n \gets 1$}
              \While{$\Algorithm{A}'(n) \neq x$}
              \label{line:while-semideciding-from-enumerating}
                \State{$n \gets n + 1$}
              \EndWhile
              \State{\Return{$1$}}
          \EndFunction
      \end{algorithmic}
      \caption{The algorithm semideciding the set that is enumerated by
        $\Algorithm{A}'$.}
      \label{algorithm:semideciding-from-enumerating}
  \end{algorithm}
  We need to prove that $\Algorithm{A}$ semidecide the set $S$.
  \begin{enumerate}
    \item Let us consider some $x \notin S$. In this case,
      $\Algorithm{A}'(n) \neq x$ for any $n \in \N$. Hence, $\Algorithm{A}$
      never terminates.
    \item Let $x \in S$. Then there is $n \in \N$ such that
      $\Algorithm{A}(n) \neq x$.
      Therefore, the number of iterations of
      \Cref{line:while-semideciding-from-enumerating} in
      \Cref{algorithm:semideciding-from-enumerating} is finite and the
      algorithm returns $1$.
  \end{enumerate}
\end{proof}

One may also establish a connection between computable functions and
enumerable sets.
\begin{theorem}
\label{theorem:enumerable-via-computable-funcitons}
  Let $S \subseteq \N$.
  \begin{enumerate}
    \item The set $S$ is enumerable iff there is a computable function
      $f : S \to \N$.
    \item The set $S$ is enumerable iff there is a computable function
      $f : \N \to \N$ so that $\Im{f} = S$.
  \end{enumerate}
\end{theorem}
\begin{proof}
  \begin{enumerate}
    \item To prove this part of the statement from right to left
        it is enough to note that
        the function $f : S \to \N$ such that $f(x) = 1$ is computable iff
        $S$ is enumerable. Indeed, if $\Algorithm{A}$ enumerates $S$, then
        it computes $f$; if $\Algorithm{A}$ computes $f$ it enumerates $S$.

        To prove it from left to right, we may notice that $g : \N \to \N$
        such that $g(x) = 1$ is computable.
        Therefore $(g \circ f) : S \to \N$ is
        computable. This implies that $S$ is enumerable since
        $(g \circ f)(x) = 1$ for all $x \in S$.
    \item This part directly follows from
        \Cref{theorem:enumerable-semidecidable}.
  \end{enumerate}
\end{proof}

In the rest of this part we refer to functions $f : S \to B$,
where $S \subseteq A$ as partial functions from $\N$ to $\N$. We say
that $S$ is preimage of $f$. Moreover, when we say that a function
from $\N$ to $\N$ is computable we mean that it is a partial computable
function. Using this notation \Cref{theorem:enumerable-via-computable-funcitons}
can be rephrased as follows.
\begin{corollary}
  \begin{enumerate}
    \item The set $S$ is enumerable iff there is a computable function
      $f : \N \to \N$ such that the preimmage of $f$ is equal to $S$.
    \item The set $S$ is enumerable iff there is a computable function
      $f : \N \to \N$ such that the immage of $f$ is equal to $S$.
  \end{enumerate}
\end{corollary}
Since the notation between functions and partial functions is that similar,
in the rest of this part we say that a partial function from $A$ to $B$ is
total iff the preimage of the function is equal to $A$.

\begin{chapterendexercises}
  \exercise[recommended] Let $S \subseteq \N$ be a nonempty set.
    Show that $S$ is decidable iff there is a function $f : \N \to \N$ such
    that $f$ is computable, $f$ is nondecreasing, and $\Im{f} = S$.
  \exercise Let $A, B \subseteq \N$ be enumerable sets. Show that
    $A \times B$ is enumerable.
  \exercise Let $F \subseteq \N^2$ be enumerable. Prove that exists a
    set $S \subseteq \N$ and a computable function $f: S \to \N$ such that
    $S = \set[(x, y) \in F]{x \in \N}$ and $(x, f(x)) \in F$ for any $x \in S$.
  \exercise
    Let $S =
      \set[
        x^n + y^n = z^n \text{ has an integer solution}
      ]{n \in \N}$. Show that $S$ is decidable.
      (You should not use Fermat's Last Theorem.)
\end{chapterendexercises}
