\chapter{Universal Functions}
It is known that we may write a program that gets another program as an argument
and run it. (Such programs are known as interpreters.) To use this observation
we give the following definition and theorem.
\begin{definition}
  We say that a function $U$ is \emph{universal function} (for the set of
  univariate computable functions) iff for each $n \in \N$,
  \[
    U_n : x \mapsto U(n, x)
  \]
  (we say that $U_n$ is a section of $U$)
  is computable and any univariate computable function is among $U_n$'s.
\end{definition}

\begin{theorem}
\label{theorem:universal-function-computable}
  There is a computable universal function $U$.
\end{theorem}

\begin{exercise}
  Assume that every section of a function $U$ is computable.
  Is it necessary true that $U$ is computable?
\end{exercise}

Similarly to the notion of universal function we may define the notion
of universal sets.
\begin{definition}
  Let $F \subseteq 2^\N$. We say that $W \subseteq \N^2$ is universal for $F$
  if $F = \set[n \in \N]{\set[(n, x) \in F]{x}}$.
\end{definition}

\begin{theorem}
  There is a enumerable set $W$ such that it is universal for the set of
  all enumerable subsets of $\N$.
\end{theorem}

\section{Enumeratable but Not Decidable Set}
\begin{theorem}
  There is a set $S \subseteq \N$ such that $S$ is enumerable but it is not
  decidable.
\end{theorem}
\begin{proof}
  Let $U$ be be a universal computable function, it exists by
  \Cref{theorem:universal-function-computable}.
  To prove the statement we are going to use the diagonalization method.
  Let us consider $S = \set[u_n(n) = U(n, n) \neq 1]{n \in \N}$.

  It is easy to see that $S$ is enumerable. Assume for the sake of
  contradiction that $S$ is decidable. Let $\Algorithm{A}$ be the algorithm
  deciding $S$. There is $n \in \N$ such that $\Algorithm{A}$ computes $U_n$
  since $U$ is universal. Let us now consider two following cases.
  \begin{enumerate}
    \item Assume that $n \in S$. In this case $\Algorithm{A}(n) = 1$ since
      $\Algorithm{A}$ decides $S$. However, $U_n(n) \neq 1$ by the definition of
      $S$. These two equalities together leads us to a contradiction since
      $\Algorithm{A}$ computes $U_n$.
    \item Assume that $n \notin S$. In this case $\Algorithm{A}(n) = 0$ since
      $\Algorithm{A}$ decides $S$. However, $U_n(n) = 1$ by the definition of
      $S$. These two equalities together leads us to a contradiction since
      $\Algorithm{A}$ computes $U_n$.
  \end{enumerate}
\end{proof}

Using the proof of this result we can make several surprising observations.

First observation states that there is no way to determine if a given
algorithm terminates or not.
\begin{theorem}
  Let $U$ be a universal function.
  Let $\Halting : \N^2 \to \set{0, 1}$ be the function such that
  $\Halting(n, x) = 1$ iff $U_n(x)$ is defined. Then $\Halting$ is not
  computable.
\end{theorem}

Another observation is that constructing the shortest program solving a given
problem is not feasible.
In other words let us consider the problem of producing the shortest algorithm
$\Algorithm{A}$ by a given $\Algorithm{B}$ such that
$\Algorithm{A}(x) = \Algorithm{B}(x)$ for any $x \in \N$. Apparently there is
no algorithm that can find such $\Algorithm{A}$.

To prove this we need to formalize what we mean by the shortest algorithm and
how we encode $\Algorithm{A}$.
\begin{theorem}
  Let $O : \N \to \N$ be the function such that
  $U_{O(n)}$ is the same as $U_n$ and $U_m$ and $U_n$ are different
  for any $m < O(n)$. Then $O$ is not computable.
\end{theorem}
