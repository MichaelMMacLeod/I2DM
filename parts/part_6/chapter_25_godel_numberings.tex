\chapter{G\"odel Universal Functions}
\Cref{theorem:composition-computable} proved that composition of two computable functions
is computbale. Moreover, it is easy to see that given the programs computing
funcitons $f$ and $g$ we can automatically obtain the function $g \circ f$.

However, we would like to avoid specifics of programin languages in our study of
computability theory. Our tool to do so is the notion of a universal function.
In this chapter we give a formulation and a proof of the statement saying that
there is an algorithm that assigns to numbers of any two computable functions a
number of their composition.

To do this we need a new class of universal functions.
\begin{definition}
  Let $U : \N^2 \to \N$ be a computable universal function for the class of
  univariate computable functions. We say that $U$ is \emph{G\"odel univeral
  function} if for any computable function $V : \N^2 \to \N$, there is a
  computable function $s : \N \to \N$ such that 
  \[
    V(n, x) = U(s(n), x)
  \]
  for all $n, x \in \N$.
\end{definition}

\begin{theorem}
  There is a G\"odel univeral function.
\end{theorem}

\begin{proof}
  We start the proof of the theorem from proving that there is a computable
  function $T : \N^3 \to \N$ that is universal for the set of bivariate
  computable functions. Let us fix a computable bijection 
  $\pair{\cdot}{\cdot} : \N^2 \to \N$. Let $R$ be a universal function for the set of
  univariate computable functions, and let $T(n, u, v) = R(n, \pair{u}{v})$.
  it is easy to see that $T$ is indeed a universal function for the set of
  bivariate computable functions.

  Let $U : \N^2 \to \N$ be the function such that $U(\pair{n}{u}, v) = 
  T(n, u, v)$. We need to show that $U$ is G\"odel univeral function.
  Let us consider some computable function $V : \N^2 \to \N$. There is $n \in
  \N$ such that $V(u, v) = T(n, u, v)$ for all $u, v \in \N$ since $T$ is
  universal. Therefore $U(\pair{n}{u}, v) = V(u, v)$ for all $u, v \in \N$.
  As a result, we can define $s(u)$ to be equal to $\pair{n}{u}$.
\end{proof}

\begin{exercise}
  Show that there is a computable bijection $\pair{\cdot}{\cdot} : \N^2 \to \N$.
\end{exercise}

\begin{theorem}
  Let $U$ be a G\"odel universal function for the set of univariate computable
  functions. Then there is a total computble function $c : \N^2 \to \N$ such
  that $U(c(p, q), x) = U(p, U(q, x))$ for any $p, q, x \in \N$.
\end{theorem}
\begin{proof}
  Let us consider a computable function $V : \N^2 \to \N$ such that
  $V(\pair{p}{q}, x) = U(p, U(q, x))$.There is a total computable function $s :
  \N \to \N$ such that $U(s(\pair{p}{q}), x) = V(\pair{p}{q}, x)$ since $U$ is a
  G\"odel universal function. Hence, if we define $c(p, q)$ to be equal to
  $s(\pair{p}{q})$, we get that $U(c(p, q), x) = U(p, U(q, x))$.
\end{proof}

Using the notion of G\"odel universal function we can prove that constructing
the shortest program solving a given problem is not feasible.
In other words let us consider the problem of producing the shortest algorithm
$\Algorithm{A}$ by a given $\Algorithm{B}$ such that
$\Algorithm{A}(x) = \Algorithm{B}(x)$ for any $x \in \N$. Apparently there is
no algorithm that can find such $\Algorithm{A}$.

To prove this we need to formalize what we mean by the shortest algorithm and
how we encode $\Algorithm{A}$.
\begin{theorem}
\label{theorem:program-optimization}
  Let $U$ be a G\"odel universal function, and let $\Optimize : \N \to \N$ be
  the function such that $U_{\Optimize(n)}$ is the same as $U_n$ and $U_m$ and
  $U_n$ are different for any $m < \Optimize(n)$. Then $O$ is not computable.
\end{theorem}

To prove this statement we need the following auxilary result.
\begin{theorem}
  Let $U$ be a G\"odel universal function. 
  Then the set 
  $S = \set[U(n, x) \text{ is not defined for all } x \in \N]{n \in \N}$ is
  not decidable.
\end{theorem}
\begin{proof}
  Let $K$ be a enumerable but undecidable set.
  Consider the partial function $V : \N^2 \to \N$ such that 
  \[
    V(n, x) = 
    \begin{cases}
      0 & \text{if } n \in K \\
      \text{undefined} & \text{otherwise}
    \end{cases}.
  \]
  Note that $V(n, x)$ terminates for some $x \in \N$ iff $n \in K$. 
  Since $U$ is G\"odel universal function there is a computable function $s$
  such that $V(n, x) = U(s(n), x)$. Hence, $s(n) \in S$ iff $n \in K$.
  As a result, $S$ is undecidable.
\end{proof}

\begin{proof}[Proof of \Cref{theorem:program-optimization}]
  Let $S$ be the set from the previous theorem.
  Assume, for the sake of contradiction, that $\Optimize$ is computable. Let
  $n_0$ be the smallest natural number $n$ such that $U(n, x)$ is not defined
  for all $x \in \N$. It is clear that $n \in S$ iff $\Optimize(n) = n_0$.
  Therefore $S$ is decidable, which is a contradiction.
\end{proof}
